\secnumbersection{INTRODUCCIÓN}
\setcounter{secnumdepth}{4}
\setcounter{tocdepth}{4}
\makeatletter
\renewcommand\paragraph{\@startsection{paragraph}{4}{\z@}%
  {1.5ex \@plus .5ex \@minus .2ex}%   % espacio antes
  {0.8ex \@plus .2ex}%                 % espacio después
  {\normalfont\normalsize\bfseries}}  % estilo
\makeatother

\subsection{Contexto y motivación}

Los Fast Radio Bursts (FRBs) son pulsos de radio transitorios de duración milisegundo y origen extragaláctico, descubiertos en 2007 \citep{Lorimer2007}, que representan uno de los fenómenos astronómicos más enigmáticos de las últimas dos décadas. Con energías isotrópicas de $10^{38}$--$10^{40}$ J y medidas de dispersión que evidencian orígenes cosmológicos \citep{Petroff_2022,Bochenek2020}, constituyen sondas del medio intergaláctico y ventanas hacia la física de magnetars \citep{Masui2015,Tendulkar2017}.

El descubrimiento de FRB repetidor (2016) \citep{Spitler2016} y la ráfaga del magnetar galáctico SGR 1935+2154 (2020) \citep{Bochenek2020,CHIME_SGR2020} establecieron conexiones directas con objetos compactos conocidos. Catálogos masivos como CHIME/FRB (miles de eventos desde 2018 \citep{Amiri2021,CHIMECat2_2024}) permiten estudios estadísticos de morfología, periodicidad y distribuciones energéticas.

El avance enfrenta un cuello de botella metodológico: pipelines tradicionales (dedispersión exhaustiva + umbrales SNR \citep{Ransom2011_PRESTO,Heimdall_Barsdell}) generan miles de candidatos con $>99\%$ falsos positivos \citep{Rajwade_2024_Review,Agarwal2020}, haciendo la curaduría manual inviable. El aprendizaje profundo ha emergido como solución \citep{Wagstaff2016,Connor2018,Agarwal2020}. DRAFTS combina detección (CenterNet) con clasificación (ResNet18) \citep{Zhang2024_DRAFTS}, pero permanece como prototipo: carece de integración coherente, gestión robusta de memoria, y generación de artefactos estandarizados.

Paralelamente, la frontera observacional se expande a frecuencias milimétricas. La detección de pulsos del magnetar PSR J1745--2900 con ALMA a $\sim$86 GHz \citep{veracasanova2025} demuestra viabilidad donde el scattering es despreciable \citep{Yang2020,Omand2022}. Sin embargo, a estas frecuencias el retardo dispersivo (patrón ``bow-tie'') se comprime a microsegundos, volviéndose imperceptible y eliminando la firma que sustenta algoritmos convencionales \citep{veracasanova2025,Torne2021}.

\subsection{Naturaleza y alcance del problema}

El problema abordado es doble. Primero, la \textbf{ausencia de pipelines astronómicos operativos} que integren modelos de aprendizaje profundo en flujos reproducibles y auditables. DRAFTS demuestra viabilidad conceptual pero carece de infraestructura operacional: gestión robusta de memoria, manejo multi-formato, continuidad temporal verificada, y artefactos con metadatos completos.

Segundo, la \textbf{brecha de extensión a alta frecuencia} (30--100 GHz). Pipelines existentes detectan firmas dispersivas en frecuencias centimétricas (0.3--3 GHz) donde el retardo temporal es resoluble. En régimen milimétrico, esta firma se comprime hasta volverse imperceptible, requiriendo estrategias diferenciadas: adaptación de rejillas DM, validación por polarización/coherencia multi-antena, y productos diagnósticos específicos \citep{veracasanova2025}.

\textbf{Alcance:} desarrollo de pipeline de inferencia y orquestación con modelos pre-entrenados, operando sobre formatos estándar (FITS/PSRFITS). No se contempla reentrenamiento, interfaces de telescopios, ni almacenamiento distribuido. Objetivo: transformar prototipo en sistema operativo extensible a alta frecuencia mediante parametrización.

\subsection{Propuesta de solución}

Esta memoria propone \textbf{DRAFTS++}, evolución de DRAFTS hacia un pipeline astronómico productivo, robusto y extensible, estructurado en dos componentes complementarios.

\textbf{Componente 1:} Pipeline end-to-end mediante ingeniería rigurosa: modularización con interfaces definidas, procesamiento en \textit{streaming} con gestión inteligente de memoria, integración automatizada de modelos CenterNet/ResNet18, ingesta multi-formato con análisis automático de headers, y generación de artefactos estandarizados.

\textbf{Componente 2:} Extensión a alta frecuencia mediante estrategias metodológicas adaptadas. Cuando el retardo dispersivo se vuelve irresoluble, el sistema activa detección alternativa: propuesta por umbral SNR con filtrado adaptado, seguida de clasificación binaria sobre patches dedispersados. Esta arquitectura híbrida mantiene coherencia metodológica mediante selección automática según características observacionales.

\subsection{Metodología general}

La metodología combina desarrollo de software, validación empírica y análisis científico en cinco fases: (1) análisis del prototipo DRAFTS identificando limitaciones arquitectónicas, (2) diseño de DRAFTS++ aplicando modularidad y separación de responsabilidades, (3) implementación en Python (PyTorch, Astropy), (4) validación mediante casos progresivos: datasets de entrenamiento (funcionalidad básica), observaciones de púlsares (robustez temporal), FRBs repetidores (escalabilidad), y ALMA a 86 GHz (estrategias de alta frecuencia), y (5) análisis de resultados con métricas estándar y confirmación científica independiente.

La metodología prioriza: portabilidad (operación multi-instrumento sin modificaciones), validez científica (precisión temporal, trazabilidad, coherencia física), y eficiencia operacional (procesamiento en tiempo casi-real).

\subsection{Estructura del documento}

La memoria sigue progresión lógica desde fundamentos hasta validación. \textbf{Capítulo 2 (Marco Conceptual):} FRBs, propiedades físicas, magnetars, alta frecuencia, métodos tradicionales, y aprendizaje automático. \textbf{Capítulo 3 (Definición del Problema):} contexto operacional, limitaciones de DRAFTS, brechas de alta frecuencia, y objetivos. \textbf{Capítulo 4 (Propuesta de Solución):} diseño e implementación de DRAFTS++ en dos componentes. \textbf{Capítulo 5 (Validación):} casos de uso progresivos y análisis comparativo. \textbf{Capítulo 6 (Conclusiones):} logros, implicaciones, limitaciones, y trabajo futuro. \textbf{Anexos:} tablas de detecciones, código representativo, y glosario técnico.

\subsection{Aportes de la memoria}

Las contribuciones principales son:

\begin{itemize}
\item Transformación de prototipo de investigación en pipeline astronómico operativo mediante ingeniería rigurosa: modularidad, streaming con gestión explícita de memoria, y trazabilidad completa.

\item Establecimiento de estrategias metodológicas para detección en régimen milimétrico, donde la firma dispersiva tradicional se comprime hasta volverse imperceptible.

\item Validación empírica demostrando capacidad de descubrimiento científico genuino en alta frecuencia, expandiendo el censo de pulsos miliméticos.

\item Código abierto, documentación de protocolos reproducibles, y conjuntos de candidatos para estudios de seguimiento.
\end{itemize}
