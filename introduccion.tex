\secnumbersection{INTRODUCCIÓN}
\setcounter{secnumdepth}{4}
\setcounter{tocdepth}{4}
\makeatletter
\renewcommand\paragraph{\@startsection{paragraph}{4}{\z@}%
  {1.5ex \@plus .5ex \@minus .2ex}%   % espacio antes
  {0.8ex \@plus .2ex}%                 % espacio después
  {\normalfont\normalsize\bfseries}}  % estilo
\makeatother

\subsection{Contexto y motivación}

Los Fast Radio Bursts (FRBs), o estallidos rápidos de radio, son pulsos de radio transitorios de duración milisegundo y origen predominantemente extragaláctico que representan uno de los fenómenos astronómicos más enigmáticos descubiertos en las últimas dos décadas. Desde la detección pionera del pulso de Lorimer en 2007 \citep{Lorimer2007}, estos eventos han capturado la atención de la comunidad astronómica internacional debido a sus propiedades extremas: energías isotrópicas de $10^{38}$--$10^{40}$ J concentradas en apenas unos milisegundos, medidas de dispersión que evidencian orígenes cosmológicos, y morfologías espectro-temporales que revelan entornos magnetizados extremos \citep{Petroff_2022,Bochenek2020}. Su relevancia científica trasciende el carácter exótico: constituyen sondas del medio intergaláctico, ventanas hacia la física de magnetars, y potenciales herramientas cosmológicas \citep{Masui2015,Tendulkar2017}.

El descubrimiento del primer FRB repetidor en 2016 \citep{Spitler2016} y la detección de una ráfaga análoga proveniente del magnetar galáctico SGR 1935+2154 en 2020 \citep{Bochenek2020,CHIME_SGR2020} establecieron conexiones directas entre estos fenómenos cosmológicos y objetos compactos conocidos, transformando el campo de especulación teórica a ciencia observacional robusta. La consolidación de catálogos masivos --CHIME/FRB ha detectado miles de FRBs desde 2018 \citep{Amiri2021,CHIMECat2_2024}-- ha permitido estudios estadísticos que revelan diversidad morfológica, periodicidades enigmáticas, y distribuciones de energía.

Sin embargo, el avance científico enfrenta un cuello de botella metodológico y computacional. Los pipelines tradicionales de búsqueda, basados en dedispersión exhaustiva y umbrales de SNR \citep{Ransom2011_PRESTO,Heimdall_Barsdell}, producen miles de candidatos por campaña, de los cuales más del 99\% son falsos positivos \citep{Rajwade_2024_Review,Agarwal2020}. La curaduría manual resulta inviable para procesamiento en tiempo real y compromete la capacidad de emitir alertas oportunas para seguimiento multi-longitud de onda \citep{Goode2022_DWF,vanLeeuwen2023_ARTS}.

En respuesta, la última década ha presenciado la incorporación de aprendizaje profundo para automatizar la clasificación de candidatos \citep{Wagstaff2016,Connor2018,Agarwal2020}. El desarrollo más reciente es DRAFTS, un pipeline que combina detección mediante visión computacional (CenterNet) con clasificación binaria (ResNet18) \citep{Zhang2024_DRAFTS}. No obstante, DRAFTS permanece como prototipo de investigación: carece de integración coherente entre etapas, depende de configuraciones codificadas manualmente, presenta restricciones de memoria que limitan su aplicación a observaciones prolongadas, y no genera artefactos estandarizados que permitan trazabilidad científica.

Paralelamente, la frontera observacional se expande hacia frecuencias inexploradas. La detección de pulsos del magnetar PSR J1745--2900 con ALMA a $\sim$86 GHz \citep{veracasanova2025} demuestra la viabilidad de búsquedas en el régimen milimétrico, donde el scattering es despreciable y ciertos progenitores en entornos densos podrían ser exclusivamente visibles \citep{Yang2020,Omand2022}. Sin embargo, a estas frecuencias el retardo dispersivo que genera el patrón ``bow-tie'' característico se reduce drásticamente --de segundos a microsegundos--, volviéndose casi imperceptible y eliminando la firma visual que sustenta los algoritmos convencionales \citep{veracasanova2025,Torne2021}.

\subsection{Naturaleza y alcance del problema}

El problema que aborda esta memoria es doble. Primero, la \textbf{ausencia de pipelines astronómicos operativos} que integren modelos de aprendizaje profundo en flujos reproducibles, eficientes y auditables. Aunque prototipos como DRAFTS demuestran viabilidad conceptual, carecen de infraestructura para operar en entornos reales: gestión robusta de memoria para observaciones prolongadas, manejo de formatos heterogéneos, continuidad temporal verificada en procesamiento por bloques, y generación de artefactos con metadatos completos.

Segundo, la \textbf{brecha de extensión a alta frecuencia} (30--100 GHz). Los pipelines existentes están optimizados para detectar firmas dispersivas en frecuencias centimétricas (0.3--3 GHz), donde el retardo temporal es fácilmente resoluble. En el régimen milimétrico, esta firma se comprime hasta volverse imperceptible, requiriendo estrategias algorítmicas diferenciadas: adaptación de rejillas de dispersión, criterios de validación alternativos basados en polarización y coherencia multi-antena, y productos diagnósticos específicos \citep{veracasanova2025}.

El alcance se delimita con precisión: desarrollo de un pipeline de inferencia y orquestación a partir de modelos ya entrenados, operando sobre datos en formatos estándar (FITS/PSRFITS). No se contempla reentrenamiento de arquitecturas neuronales, interfaces de control de telescopios, ni sistemas de almacenamiento distribuido. El objetivo es transformar un prototipo de investigación en un sistema operativo, extensible a alta frecuencia mediante parametrización adecuada y sin reentrenar modelos.

\subsection{Propuesta de solución}

Esta memoria propone \textbf{DRAFTS++}, una evolución del prototipo DRAFTS hacia un pipeline astronómico productivo, robusto y extensible. La propuesta se estructura en dos componentes complementarios.

El \textbf{Componente 1} construye un pipeline end-to-end mediante ingeniería de software rigurosa: modularización del sistema en etapas con interfaces bien definidas, procesamiento en \textit{streaming} con gestión inteligente de memoria que permite manejar observaciones de duración arbitraria, integración automatizada de modelos CenterNet y ResNet18 eliminando ejecución manual por etapas, sistema de ingesta multi-formato con análisis automático de headers, y generación de artefactos estandarizados que garantizan trazabilidad y reproducibilidad.

El \textbf{Componente 2} aborda la extensión a alta frecuencia mediante estrategias metodológicas adaptadas. Cuando el retardo dispersivo se vuelve irresoluble, el sistema activa una rama de detección alternativa inspirada en métodos tradicionales: propuesta de candidatos por umbral SNR con filtrado adaptado, seguida de clasificación binaria sobre patches dedispersados. Esta arquitectura híbrida mantiene coherencia metodológica entre regímenes frecuenciales mediante selección automática de estrategias según características observacionales.

\subsection{Metodología general}

La metodología empleada combina desarrollo de software, validación empírica y análisis científico en cinco fases secuenciales. Primero, análisis exhaustivo del prototipo DRAFTS para identificar limitaciones arquitectónicas y operativas. Segundo, diseño arquitectónico de DRAFTS++ aplicando principios de modularidad, separación de responsabilidades, y especificación de contratos formales entre etapas. Tercero, implementación en Python utilizando PyTorch para inferencia, Astropy para manejo de datos astronómicos, y bibliotecas especializadas para aceleración computacional. Cuarto, validación empírica mediante casos progresivos de complejidad creciente: datasets de entrenamiento para funcionalidad básica, observaciones de púlsares para robustez temporal, datos de FRBs repetidores para escalabilidad, y observaciones de ALMA a 86 GHz para validación de estrategias de alta frecuencia. Quinto, análisis de resultados mediante métricas estándar de detección y clasificación, documentación de casos representativos, y confirmación científica independiente de nuevos eventos.

La metodología prioriza tres criterios fundamentales: portabilidad (operación sobre diferentes instrumentos sin modificaciones de código), validez científica (precisión temporal, trazabilidad, coherencia física), y eficiencia operacional (latencia mínima para procesamiento en tiempo casi-real).

\subsection{Estructura del documento}

Esta memoria se organiza en capítulos que siguen una progresión lógica desde fundamentos hasta validación. El Capítulo 2 (Marco Conceptual) presenta los Fast Radio Bursts, sus propiedades físicas, casos históricos, vínculos con magnetars, observaciones en alta frecuencia, métodos tradicionales de búsqueda, y enfoques de aprendizaje automático. El Capítulo 3 (Definición del Problema) caracteriza el contexto operacional, diagnostica limitaciones del prototipo DRAFTS, analiza brechas específicas de alta frecuencia, y formula objetivos con precisión. El Capítulo 4 (Propuesta de Solución) presenta el diseño e implementación de DRAFTS++ en sus dos componentes: pipeline end-to-end productivo y extensión a alta frecuencia mediante estrategias metodológicas adaptadas. El Capítulo 5 (Validación de la Solución) verifica empíricamente el sistema mediante casos de uso progresivos y análisis comparativo de estrategias. El Capítulo 6 (Conclusiones) sintetiza logros, discute implicaciones, identifica limitaciones, y propone trabajo futuro. Finalmente, los Anexos incluyen tablas de detecciones completas, código representativo, y glosario técnico.

\subsection{Aportes de la memoria}

Las contribuciones principales de este trabajo son:

\begin{itemize}
\item Transformación de un prototipo de investigación en un pipeline astronómico operativo mediante aplicación rigurosa de principios de ingeniería de software: modularidad, streaming con gestión explícita de memoria, y trazabilidad completa.

\item Establecimiento de estrategias metodológicas sistemáticas para detección de transientes en el régimen milimétrico, donde la firma dispersiva tradicional se comprime hasta volverse imperceptible.

\item Validación empírica que demuestra capacidad de descubrimiento científico genuino en datos de alta frecuencia, expandiendo el censo conocido de pulsos miliméticos.

\item Código fuente de código abierto, documentación completa de protocolos reproducibles, y conjuntos de candidatos prometedores para estudios de seguimiento.
\end{itemize}
