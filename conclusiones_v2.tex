\secnumbersection{CONCLUSIONES}

\textit{DRAFTS++ demuestra que la ingeniería informada por física transforma un prototipo de detección de FRBs en un sistema operacional end-to-end que, además de procesar a escala con robustez, produce descubrimiento científico verificable.}

\subsection{Síntesis frente a Objetivos}

El sistema opera E2E sin intervención y procesa archivos multi-gigabyte con recuperación íntegra de eventos (Recall 100\% en FRB 121102 del telescopio Effelsberg); valida robustez temporal en fuentes periódicas (732/752 pulsos en B0355+54, 97.3\%); y extiende sensibilidad al régimen milimétrico (~86 GHz ALMA) con un enfoque híbrido inspirado en PRESTO (Recall 100\% y 44 pulsos nuevos confirmados del magnetar PSR J1745-2900). La mejora sostenida se evidencia también en curvas Precision-Recall y AUPRC, métricas adecuadas para clases desbalanceadas en detección de transientes astronómicos. El descubrimiento de 46 eventos confirmados independientemente (2 FRBs + 44 pulsos de magnetar) más 116 candidatos prometedores demuestra que el sistema amplía genuinamente conocimiento astronómico, no solo replica análisis previos.

\subsection{Significado e Impacto}

Operacionalmente, DRAFTS++ pasa de código de investigación a infraestructura científica reutilizable para re-análisis a escala y campañas near-real-time. Científicamente, amplía el censo (2 bursts nuevos de FRB 121102 y 44 pulsos del magnetar del Centro Galáctico) y fija un baseline cuantificado para transferencia a alta frecuencia. En HF, la caracterización rigurosa reveló que la existencia de pulsos persistentemente indetectables (Línea 1: Recall máximo 87.5\%) expone limitaciones arquitecturales irreductibles cuando el bow-tie colapsa, motivando estrategias especializadas; la estrategia híbrida (Línea 2: Recall 100\%) superó este límite apoyándose en estadísticos temporales en lugar de firmas 2D comprimidas. La mejora se refleja también en PR/AUPRC, métricas idóneas en datos desbalanceados.

\subsection{Calidad del Software y Reproducibilidad}

Evaluamos con ISO/IEC 25010 (2011, actualizado 2023): rendimiento, confiabilidad (0 pérdidas en bordes), mantenibilidad modular y portabilidad multi-instrumento; artefactos FAIR disponibles en \url{https://github.com/Kodamonkey/DRAFTS-UC} y orientados a ACM Artifact Badging.

\subsection{Limitaciones y Próximos Pasos}

\begin{itemize}
    \item HF con N reducido → ampliar dataset y polarimetría Q/U/V.
    \item Generalización → validar en VLA/MeerKAT/CHIME.
    \item Tiempo real → integración operacional (streaming y telemetría).
\end{itemize}

\subsection{Cierre}

La ruta física → algoritmos → arquitectura habilitó descubrimiento reproducible; proponemos este estándar para software astronómico productivo.
