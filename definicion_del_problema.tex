\secnumbersection{DEFINICIÓN DEL PROBLEMA}

% Se debe definir el problema, es importante no confundir definir el problema con describir la solución. Por ejemplo: ``diseñar una arquitectura e implementar una plataforma ...'' es una solución, no un problema.
% 
% Algunos elementos que podrían ir en este capítulo son (no es necesario que vayan todos):
% \begin{itemize}
%     \item Breve descripción del contexto donde se realizará la memoria (organización, línea dentro de la Informática en la que se basa, etc.)
%     \item ¿Qué y cómo se realiza actualmente la situación que mejorarás con tu memoria?
%     \item ¿Qué actores o usuarios están involucrados?
%     \item ¿Qué dificultades tienen esos actores actualmente? ¿cuántos son? (ideal si se pueden poner estadísticas para así saber si existe un mercado razonable para la solución que propondrás en tu memoria, en el fondo saber cuántas personas u organizaciones tienen el mismo problema que estás definiendo)
%     \item ¿Qué podría pasar si en el corto o mediano plazo no se solucionan esas dificultades (¿es decir, si no se hiciera tu memoria, qué pasaría?; en el fondo justificar por qué conviene hacer tu memoria, ¿cuál es la motivación o interés de hacerla?).
%     \item ¿Qué competencia existe actualmente? (a lo mejor ya existe una solución al problema, pero por qué no sirve, o por qué tu solución sería mejor, también se puede enfocar a si este problema existe en otras realidades y cómo ha sido solucionado allí).
%     \item Precisar los objetivos y alcances de la memoria (o solución al problema).
% \end{itemize}
% 
% En este capítulo, de ser necesario puede usar referencias bibliográficas (velar porque sean recientes), una cita de ejemplo \cite{schwab2002cure} y otras más \cite{georget1994study,beaumont1990patient}.
% 
% Recuerde poner notas al pie de página que sean explicativas \footnote{Este es un ejemplo de una nota al pie de página. Puede indicar alguna URL, definiciones, aclarar alguna información pertinente del texto, citar algunas referencias, etc..}.
% 
% \subsection{SUBSECCIÓN DE PRUEBA}
% 
% Sed ut perspiciatis unde omnis iste natus error sit voluptatem accusantium doloremque laudantium, totam rem aperiam, eaque ipsa quae ab illo inventore veritatis et quasi architecto beatae vitae dicta sunt explicabo.
% 
% \subsubsection{SUBSUBSECCIÓN DE PRUEBA}
% 
% Nemo enim ipsam voluptatem quia voluptas sit aspernatur aut odit aut fugit, sed quia consequuntur magni dolores eos qui ratione voluptatem sequi nesciunt. Neque porro quisquam est, qui dolorem ipsum quia dolor sit amet.
% 
% \subsubsection{OTRA SUBSUBSECCIÓN DE PRUEBA}
% 
% Nemo enim ipsam voluptatem quia voluptas sit aspernatur aut odit aut fugit, sed quia consequuntur magni dolores eos qui ratione voluptatem sequi nesciunt. Neque porro quisquam est, qui dolorem ipsum quia dolor sit amet.
\subsection{Contexto y caracterización del problema de detección}

En el capítulo anterior se presentó el marco conceptual de los Fast Radio Bursts, incluyendo sus fundamentos físicos y fenomenología observacional (secciones 2.1--2.2), casos históricos de FRBs repetidores (sección 2.3), su vínculo con magnetars (sección 2.4), las oportunidades de observación en alta frecuencia (sección 2.5), los métodos tradicionales de búsqueda (sección 2.6), y los enfoques modernos de aprendizaje automático para su detección (sección 2.7). En particular, la sección 2.7.1 describió DRAFTS, un pipeline de nueva generación basado en deep learning que combina detección mediante CenterNet y clasificación con ResNet, alcanzando tasas de recuperación cercanas al 100\% y detectando más del triple de ráfagas comparado con métodos tradicionales. Asimismo, la sección 2.8 identificó los desafíos específicos de la detección en el régimen milimétrico, incluyendo: sensibilidad reducida, curvatura de dispersión casi imperceptible, morfología de señales desconocida, aspectos instrumentales complejos, y la necesidad de adaptaciones en los pipelines para validación sin firma de dispersión clara.

\medskip

Partiendo de ese estado del arte, este capítulo define el problema específico que aborda esta memoria. Si bien se han desarrollado métodos y modelos capaces de detectar y clasificar FRBs mediante aprendizaje profundo, estas aproximaciones suelen estar limitadas a implementaciones específicas, de difícil reproducción y con escasa portabilidad. En particular, no se dispone de un flujo de procesamiento consolidado que pueda adaptarse a distintos entornos y extenderse a regímenes de alta frecuencia (mm-wave) sin necesidad de reentrenar los modelos.

\medskip

La búsqueda de FRBs en radioastronomía se desarrolla principalmente en dos regímenes espectrales con características y desafíos distintos. En las frecuencias tradicionalmente utilizadas (bandas decimétricas, aproximadamente 300 MHz - 3 GHz), se emplean los métodos tradicionales descritos en la sección 2.6 (dedispersión por ensayo de DM, filtrado por anchura de pulso, umbralización, agrupamiento y rechazo de RFI). A pesar de la madurez relativa de estos métodos, la etapa de depuración continúa demandando curaduría manual extensiva y el desarrollo de scripts \emph{ad hoc} específicos para cada equipo de investigación.

En contraste, el régimen de alta frecuencia (30--100 GHz, ondas milimétricas), particularmente en configuraciones de arreglo en fase (\emph{phased array}) como las implementadas en ALMA\footnote{Combinación coherente de señales de múltiples antenas para formar series temporales de alta resolución.}, presenta una ventana científica prometedora pero carece de \textit{pipelines} estandarizados. En estas frecuencias, el retardo dispersivo $\Delta t\propto \mathrm{DM}\,\nu^{-2}$ \citep{LorimerKramer2005} es menor, los patrones diagnósticos (\emph{bow-tie} en tiempo--DM) se atenúan, y los desarrollos actuales permanecen en estado de prototipo.

\medskip

Como se estableció en las secciones 2.6--2.7, se han desarrollado herramientas clásicas como PRESTO y Heimdall \citep{Ransom2011_PRESTO,Heimdall_Barsdell}, eficaces en bandas L y S pero limitadas para escenarios de alta frecuencia. En aprendizaje profundo, marcos como DRAFTS han explorado aproximaciones prometedoras mediante detección en mapas tiempo--DM y clasificación binaria sobre \emph{patches} tiempo--frecuencia. Sin embargo, la transición hacia un \textit{pipeline} operativo que incorpore principios sólidos de ingeniería de software y sea extensible entre regímenes frecuenciales permanece como una brecha crítica.

Los desafíos operativos comunes a ambos regímenes incluyen: fricción computacional (archivos grandes y procesamiento intensivo que genera latencia crítica para alertas de seguimiento), altas tasas de falsos positivos (RFI y artefactos instrumentales que exigen validación manual), y la ausencia de pipelines end-to-end que integren modelos de aprendizaje automático con ingeniería de datos, métricas y reportabilidad robustas.


\begin{table}[h]
\centering
\caption{\textbf{Comparación de características técnicas entre regímenes espectrales relevantes para el diseño del pipeline.}}
\vspace{0.5em}
\small
\begin{tabular}{l c c}
\toprule
\textbf{Aspecto} & \textbf{0.3--3\,GHz} & \textbf{30--100\,GHz} \\
\midrule
Retardo por dispersión & Alto; patrones claros & Bajo; patrones \\
& en tiempo--DM & atenuados \\
\midrule
RFI/sistemáticas & RFI ancha banda & Atmósfera, estabilidad \\
& & de fase, distinta RFI \\
\midrule
Productos útiles & Waterfalls, tiempo--DM & Tiempo--frecuencia, \\
& & polarización/diagnósticos \\
\midrule
Disponibilidad & Pipelines consolidados & Prototipos, sin \\
de software & & estándar general \\
\bottomrule
\end{tabular}

\vspace{0.3em}
\raggedright
\small{\textit{Nota: RFI = Radio Frequency Interference (Interferencia de Radiofrecuencia); DM = Dispersion Measure (Medida de Dispersión).}}
\end{table}




\subsection{Actores involucrados y alcance del problema}

La detección de FRBs involucra a múltiples actores en la comunidad astronómica internacional. Los principales usuarios y afectados por las limitaciones actuales incluyen:

\textbf{Observatorios y colaboraciones científicas:} Instalaciones como CHIME (Canadá), ASKAP (Australia), FAST (China), MeerKAT (Sudáfrica) y ALMA (Chile) \citep{Matthews2018_PASP} operan programas de detección de FRBs \citep{Amiri2021,Bannister2019,Niu2022,Bezuidenhout2022}. Solo CHIME ha detectado miles de FRBs desde 2018, generando volúmenes masivos de datos que requieren procesamiento sistemático \citep{Amiri2021,CHIMECat2_2024,NRC_CHORD_2025}. ALMA, por su parte, está iniciando búsquedas pioneras en el régimen milimétrico, sin contar aún con pipelines consolidados para este propósito \citep{veracasanova2025,Torne2021}.

\textbf{Grupos de investigación:} Se estiman decenas de equipos activos a nivel mundial trabajando en detección, caracterización y seguimiento de FRBs \citep{Petroff_2022}. Cada equipo desarrolla típicamente sus propios scripts de análisis, lo que resulta en esfuerzos duplicados y resultados difícilmente comparables entre campañas \citep{Rajwade_2024_Review}.

\textbf{Operadores de telescopios:} El personal técnico que gestiona las observaciones requiere herramientas robustas y automatizadas para validar detecciones en tiempo real y coordinar seguimientos multi-frecuencia. La curaduría manual actual puede tomar horas o días, comprometiendo la capacidad de respuesta ante eventos transitorios \citep{Rajwade_2024_Review}.

\textbf{Comunidad multi-mensajero:} La detección oportuna de FRBs es crítica para campañas de seguimiento en otras longitudes de onda (óptico, rayos X) que requieren alertas dentro de minutos u horas tras la detección inicial \citep{CHIMEFRBVOEvent2021,Seaman2011,vanLeeuwen2023_ARTS}.

Las estadísticas actuales indican que cada campaña de observación de FRBs genera entre miles y decenas de miles de candidatos, de los cuales típicamente $>99\%$ son falsos positivos \citep{Rajwade_2024_Review,Agarwal2020,Zhang2024_Tianlai}. La curaduría manual de estos candidatos puede tomar desde segundos hasta minutos por evento, resultando en cargas de trabajo de horas o días por campaña que son inviables para procesamiento en tiempo real \citep{Rajwade_2024_Review,Goode2022_DWF,Wagstaff2016}.

\subsection{Diagnóstico del pipeline DRAFTS}

Como se describió en la sección 2.7.1, DRAFTS (Deep Learning-based RAdio Fast Transient Search) representa el estado del arte en pipelines de búsqueda de FRBs basados en aprendizaje profundo \citep{Zhang2024_DRAFTS}. Este sistema combina un detector de objetos basado en CenterNet con un clasificador ResNet, logrando alta completitud y detectando más del triple de ráfagas comparado con pipelines tradicionales como Heimdall en el caso de FRB 20190520B. Sin embargo, si bien sus modelos neuronales son robustos y el enfoque conceptual es sólido, su arquitectura de software presenta limitaciones significativas que dificultan su adopción como herramienta operativa en entornos observacionales reales. A continuación, se presenta un análisis detallado de sus principales restricciones técnicas y operativas, que fundamentan la necesidad del problema abordado en esta investigación.

\textit{DRAFTS} es un “pseudo-pipeline”, más cercano a un prototipo experimental que a un software astronómico operativo. No obstante, sus modelos de detección y clasificación de FRBs son robustos y completamente funcionales; el problema surge en torno a la infraestructura que los rodea. \textit{DRAFTS} presenta una estructura monolítica con \textit{scripts} independientes, optimizados para condiciones de laboratorio controladas pero incapaces de manejar la variabilidad operacional de entornos observacionales reales. Más que limitaciones incrementales, el código original no constituye un \textit{pipeline} operativo: la búsqueda en datos reales se materializa en programas separados que el usuario debe ejecutar y parametrizar manualmente. No existe un punto de entrada unificado, una interfaz de línea de comandos ni un sistema de configuración versionado; el flujo exige editar variables en el código (rutas, umbrales, \textit{checkpoints}, tamaños de bloque, DM, etc.) y ejecutar etapas desconectadas sin orquestación. El propio \textit{README} instruye a “modificar la ruta de datos y de guardado y ejecutar el archivo”, lo que evidencia la ausencia de un flujo automatizado de extremo a extremo que, con un único \textit{input}, entregue resultados reproducibles y simples para el usuario astrónomo.

En la ingesta de datos se observan supuestos instrumentales rígidos. El lector de PSRFITS asume un esquema específico propio de FAST/GBT, y el repositorio indica explícitamente que para otros telescopios se deben “modificar” funciones internas. No hay detección automática de formato ni un análisis robusto de encabezados; los parámetros observacionales se derivan parcialmente con constantes implícitas y números mágicos (p.ej., factores de decimado temporal y espectral), además de correcciones \textit{ad hoc} como la inversión del eje de frecuencia o normalizaciones mín–máx. Tampoco se calculan marcas temporales precisas a partir de los \textit{headers}, por lo que los tiempos de arribo son relativos y no trazables de manera consistente dentro del archivo ni interoperables con análisis posteriores.

La gestión de datos y memoria es frágil para observaciones prolongadas. El procesamiento es monolítico sobre bloques grandes construidos concatenando archivos contiguos en memoria, y para completar las ventanas de dedispersión en los bordes se recurre a relleno sintético con ruido aleatorio cuando faltan muestras, alterando la estadística de fondo y comprometiendo la validez científica en condiciones reales. La dedispersión GPU (\texttt{numba.cuda}) opera sobre tensores completos de DM–tiempo sin un planificador de \textit{chunking} que respete un presupuesto de memoria; no hay telemetría ni control de uso de VRAM, limpieza sistemática, \textit{fallback} a CPU ante \textit{out-of-memory} o manejo robusto de errores. Esta combinación limita la estabilidad y escalabilidad del sistema cuando la duración, el ancho de banda o la resolución crecen.

La extracción y validación de candidatos carecen de un esqueleto unificado. La detección obtiene cajas en DM–tiempo, pero el filtrado posterior es heurístico (p.ej., exigir \texttt{DM > 20}) y la clasificación está desacoplada en un programa distinto que dedispersa a un DM fijo predefinido, sin realimentación del detector. No existe un validador físico común (coherencia por subbandas, verificación de DM positivo con incertidumbre acotada, consistencia temporal entre ventanas) ni mecanismos de desduplicación de eventos entre \textit{slices} y archivos. Las salidas se limitan a imágenes y algunos archivos \texttt{.npy} por bloque; no se generan artefactos estandarizados (CSV/Parquet con metadatos completos), ni resúmenes por ejecución, ni firmas de modelos y datos que habiliten trazabilidad y auditoría. La ausencia de \textit{logging} estructurado y de semillas controladas impide reproducir resultados de manera fiable.

En operación y extensibilidad, los modelos se cargan desde rutas relativas y su presencia es un prerrequisito tácito; si faltan, el programa simplemente falla sin orientación al usuario. No hay empaquetado ni instalación como librería, pruebas automatizadas, ni documentación de un flujo end–to–end; el entrenamiento existe, pero la integración con la búsqueda es manual. La compatibilidad multibanda y el tratamiento de polarización son limitados (se promedian canales y polarizaciones tempranamente), y no existe un mecanismo para seleccionar estrategias de búsqueda según condiciones físicas (resolución temporal, rango de frecuencias, SNR, disponibilidad de Stokes).

En suma, a pesar de contar con modelos de detección y clasificación bien entrenados, el prototipo carece de un software de \textit{pipeline} que permita, con un único \textit{input}, obtener resultados directos, sencillos y trazables en un entorno observacional real.

\subsection{Brecha específica: extensión a alta frecuencia}

Como se detalló en la sección 2.8, la detección de FRBs en el régimen milimétrico enfrenta desafíos únicos que actualmente limitan su viabilidad operativa. Estos desafíos no son meramente técnicos, sino que representan brechas científicas y de ingeniería críticas que requieren soluciones específicas. A continuación se analizan en detalle las limitaciones actuales y las capacidades faltantes en los pipelines existentes.

\subsubsection{Adaptaciones algorítmicas necesarias para pipelines en mm}

Los pipelines avanzados de detección desarrollados para frecuencias centimétricas \textbf{requieren modificaciones fundamentales} cuando se aplican al régimen milimétrico. El desafío principal radica en que, a frecuencias tan altas, la firma típica de dispersión se reduce considerablemente. Por ejemplo, un DM de $\sim1000$ pc cm$^{-3}$ genera un barrido de $\sim1.8$ s a 1.4 GHz, pero solo $\sim0.05$ s a 86 GHz \citep{veracasanova2025}. Esta reducción drástica del retardo dispersivo tiene implicaciones fundamentales para los algoritmos de búsqueda.

El único estudio exitoso de pulsos individuales con ALMA (Banda 3, $\sim86$ GHz) utilizó un pipeline basado en PRESTO adaptado manualmente para leer datos en formato PSRFITS \citep{veracasanova2025}. Dado que el DM del magnetar observado era conocido de antemano (1770 pc cm$^{-3}$), no fue necesario barrer múltiples valores de DM. Sin embargo, para búsquedas ciegas de FRBs --el caso de uso real-- esta simplificación no aplica, y \textbf{no existe actualmente} una estrategia estandarizada implementada en software que maneje eficientemente:

\begin{itemize}
\item Reducción automática de la granularidad de DM según la frecuencia observacional (p.ej. $\Delta$DM de 5--10 unidades en mm vs. $\sim1$ en cm)
\item Decisión adaptativa sobre cuándo aplicar integración directa en frecuencia vs. dedispersión completa
\item Cálculo dinámico del número óptimo de pruebas de DM basado en ancho de banda y frecuencia central
\item Configuración paramétrica de estos aspectos sin modificar código fuente
\end{itemize}

\subsubsection{Estrategias de validación sin curvatura de dispersión apreciable}

Uno de los desafíos más críticos en mm es que el indicador clave de una FRB --su trazo de dispersión curvado con $1/\nu^2$-- se vuelve casi imperceptible en estrechos rangos de alta frecuencia. Como se describió en la sección 2.8, a frecuencias centimétricas un candidato se valida exigiendo que muestre la forma de retardo frecuencial correcta (forma de ``bow tie'' en el plano DM-tiempo). En cambio, en una banda de 90--100 GHz, un FRB con DM elevado aparecería casi como un pulso simultáneo en todas las frecuencias, difícil de diferenciar de un pulso de DM=0.

Los experimentos iniciales con ALMA han demostrado estrategias alternativas de validación basadas en polarización: en el caso del magnetar galáctico, 5 de los 8 pulsos mostraron $>90\%$ de polarización circular o lineal \citep{veracasanova2025}. Sin embargo, \textbf{ningún pipeline actual implementa sistemáticamente}:

\begin{itemize}
\item Análisis automático de polarización (Stokes Q/U/V) como criterio de validación primario
\item Verificación de coherencia multi-antena para discriminar eventos celestes de RFI local
\item Estrategias de validación multi-banda (coordinación automática entre observaciones en cm y mm)
\item Umbrales adaptativos de SNR según la disponibilidad de información auxiliar (polarización, multi-antena, etc.)
\end{itemize}

Estas capacidades permanecen como implementaciones \emph{ad hoc} específicas de cada experimento, sin una infraestructura de software reutilizable.

\subsubsection{Exigencias computacionales del procesamiento en tiempo real}

El procesamiento en tiempo real de datos mm conlleva exigencias computacionales que exceden significativamente las de frecuencias tradicionales. A mayor frecuencia, los pulsos pueden ser intrínsecamente más breves, requiriendo muestreos de orden sub-milisegundo (incluso decenas de microsegundos). En la detección de pulsos del magnetar con ALMA, se trabajó con resoluciones temporales de $\sim8$ $\mu$s por muestra \citep{veracasanova2025}. Esto implica manejar flujos de datos voluminosos: un solo stream con $\Delta t \sim 10$ $\mu$s produce 100,000 muestras por segundo; con 32 canales por banda y dos polarizaciones, se procesan millones de muestras por segundo.

\textbf{Las limitaciones actuales} de pipelines como DRAFTS incluyen:

\begin{itemize}
\item Ausencia de control dinámico de memoria GPU (no hay telemetría de VRAM, \emph{fallback} a CPU ante out-of-memory, o limpieza sistemática)
\item Falta de arquitectura de \emph{chunking} adaptativa que ajuste el tamaño de bloques según recursos disponibles
\item Inexistencia de estrategias de \emph{downsampling} inteligente específicas para alta frecuencia
\item Carencia de mecanismos de paralelización multi-GPU o distribución entre nodos para manejar flujos de 10--100 Gbps
\end{itemize}

Estas limitaciones hacen inviable el procesamiento en tiempo real de observaciones largas (>1 hora) en mm con los pipelines actuales, comprometiendo la capacidad de generar alertas oportunas.

\subsubsection{Productos diagnósticos específicos para alta frecuencia}

A diferencia de las búsquedas en cm donde se priorizan diagramas DM--tiempo y espectrogramas tiempo-frecuencia estándar, el régimen mm \textbf{requiere productos diagnósticos diferenciados}:

\begin{itemize}
\item Análisis de polarización por pulso (fracciones de Stokes, ángulos de posición, RM si hay banda ancha)
\item Mapas de coherencia temporal multi-antena para validar origen celeste
\item Análisis espectral de ultra-alta resolución ($<10$ $\mu$s) para buscar microestructuras
\item Comparación automática con detecciones simultáneas en otras bandas (si disponibles)
\end{itemize}

Actualmente, \textbf{ningún pipeline genera estos productos de manera automática}. Los investigadores deben extraer manualmente datos de polarización, escribir scripts personalizados para cada análisis, y realizar comparaciones multi-banda manualmente. Esta falta de estandarización dificulta la comparabilidad entre campañas y ralentiza el análisis científico.

\medskip

En suma, la extensión a alta frecuencia no es solo una cuestión de ``cambiar un parámetro'': requiere reformulación de estrategias algorítmicas, implementación de nuevos criterios de validación, optimización computacional sustancial, y generación de productos diagnósticos específicos. Ningún pipeline actual --incluido DRAFTS en su forma original-- incorpora estas capacidades de manera sistemática, configurable y operativa. Esta brecha constituye la segunda línea de investigación de esta memoria.

\subsection{Formulación del problema}

Considerando el contexto, el diagnóstico de DRAFTS y las brechas identificadas en la extensión a alta frecuencia, se formula el problema central de esta memoria:

\medskip

\noindent\textbf{Dado} un flujo de datos de radioastronomía (FITS/PSRFITS u otros formatos) y dos modelos de aprendizaje profundo preentrenados (detección y clasificación de FRBs), \textbf{no existe actualmente} un \textit{pipeline} operativo, reproducible y portable que:

\begin{enumerate}
\item Procese datos en lotes y en línea con control eficiente de recursos computacionales y de memoria,
\item Reduzca la tasa de falsos positivos mediante una segunda criba basada en aprendizaje profundo,
\item Genere salidas completamente auditables (catálogo de candidatos con metadatos, recortes tiempo--frecuencia, figuras diagnósticas y métricas de rendimiento),
\item Sea extensible a regímenes de alta frecuencia (30--100 GHz) mediante parametrización adecuada de rejillas DM, escalas temporales y productos diagnósticos (incluida polarización cuando esté disponible),
\item Todo ello \textbf{sin requerir reentrenamiento} de los modelos base para cada nuevo instrumento o régimen espectral.
\end{enumerate}

\medskip

Esta ausencia constituye una barrera crítica para: (i) la operación eficiente de programas de detección de FRBs en tiempo casi-real, (ii) la exploración sistemática del régimen milimétrico con instrumentos como ALMA, y (iii) la comparabilidad y reproducibilidad de resultados entre diferentes campañas observacionales.

\medskip

El problema abordado en esta memoria se enfoca específicamente en dos líneas de investigación complementarias: \textbf{(1) Desarrollo de software end-to-end funcional y eficiente}, aplicando principios rigurosos de ingeniería de software para transformar un prototipo experimental en un sistema operativo robusto; y \textbf{(2) Extensión a alta frecuencia}, mediante la incorporación de estrategias algorítmicas, productos diagnósticos y mecanismos de validación específicos para el régimen milimétrico, todo ello sin requerir reentrenamiento de los modelos neuronales base.

\subsection{Consecuencias de no abordar el problema}

Si las limitaciones descritas no se solucionan en el corto y mediano plazo, se presentan las siguientes consecuencias:

\textbf{Pérdida de oportunidades científicas:} Los FRBs son fenómenos transitorios cuyo seguimiento multi-longitud de onda es crítico para comprender su naturaleza física \citep{Petroff_2022}. La latencia actual en la validación de candidatos (horas a días) impide coordinar observaciones complementarias en tiempo útil, resultando en que la mayoría de los FRBs detectados no reciben seguimiento oportuno \citep{Rajwade_2024_Review}.

\textbf{Subutilización de infraestructura de alta frecuencia:} Inversiones millonarias en instrumentos como ALMA modo \emph{phased array} no pueden aprovecharse plenamente para ciencia de transientes sin pipelines especializados \citep{veracasanova2025}. La comunidad está perdiendo la ventana crítica de primeras detecciones en el régimen mm, donde cada descubrimiento podría revolucionar nuestra comprensión de los mecanismos de emisión \citep{Torne2021,veracasanova2025}.

\textbf{Esfuerzo científico ineficiente:} Cada grupo de investigación continúa desarrollando sus propias herramientas \emph{ad hoc}, duplicando esfuerzos y generando resultados difícilmente comparables \citep{Rajwade_2024_Review}. Esto fragmenta el conocimiento y ralentiza el avance del campo.

\textbf{Limitación en catálogos y estadísticas:} La incompletitud introducida por pipelines subóptimos sesga las distribuciones observadas de propiedades de FRBs (energías, tasas, espectros), comprometiendo estudios cosmológicos y astrofísicos que dependen de muestras completas y bien caracterizadas \citep{Petroff_2022,Rajwade_2024_Review}.

\textbf{Exclusión de la comunidad:} La barrera técnica actual limita la participación de grupos con menor experiencia en procesamiento de datos, concentrando la ciencia en pocas instituciones con recursos dedicados a desarrollo de software \citep{Rajwade_2024_Review}.


\subsection{Objetivos y alcances de la memoria}

\textbf{Objetivo General}

Desarrollar un \textit{pipeline} astronómico operativo para detección y clasificación de FRBs basado en dos CNN preentrenadas, extensible a regímenes de alta frecuencia sin reentrenamiento de modelos.

\medskip

\noindent\textbf{Objetivos Específicos}

\begin{enumerate}
\item Integrar los modelos de detección y clasificación en un flujo robusto y reproducible: ingesta $\to$ preprocesamiento $\to$ inferencia $\to$ posprocesamiento $\to$ reporte auditable.
\item Implementar optimizaciones de rendimiento: procesamiento eficiente de archivos grandes, gestión de memoria, aceleración computacional y registro completo de operaciones.
\item Parametrizar el flujo para adaptación a alta frecuencia (rejillas DM, ventanas temporales, productos Stokes/polarización) sin reentrenamiento de modelos.
\item Validar el sistema sobre datos previamente analizados, igualando o superando recuentos reportados y caracterizando latencia, \emph{precision} y \emph{recall}.
\item Implementar productos diagnósticos específicos para ondas milimétricas y realizar análisis exploratorios para caracterizar las propiedades distintivas de FRBs en alta frecuencia.
\end{enumerate}

\medskip

\noindent\textbf{Alcances}

Esta memoria se enfoca en el desarrollo de un \textit{pipeline} de inferencia y orquestación a partir de modelos preentrenados y datos astronómicos en formatos estándar (FITS/PSRFITS). \textbf{No se contempla:}

\begin{itemize}
\item Reentrenamiento o desarrollo de nuevas arquitecturas de modelos neuronales
\item Desarrollo de \emph{backends} instrumentales o interfaces de control de telescopios
\item Implementación de sistemas de almacenamiento distribuido o bases de datos astronómicas
\item Integración con sistemas de alertas multi-mensajero externos (aunque se proporcionan las salidas necesarias para ello)
\end{itemize}

\medskip

\noindent\textbf{Resultados Esperados}

\begin{itemize}
\item \textit{Pipeline} end-to-end ejecutable por línea de comandos y/o servicio, con documentación completa y suite de pruebas automatizadas
\item Reporte comparativo de desempeño que incluya métricas de \emph{recall}, \emph{precision}, \emph{throughput} y latencia por GB/archivo procesado
\item Conjunto de figuras y \emph{notebooks} ilustrativos que demuestren la aplicación del pipeline en el régimen de ondas milimétricas
\item Código fuente publicado bajo licencia de código abierto para uso de la comunidad astronómica
\end{itemize}

\begin{figure}[h]
\centering
% \includegraphics[width=0.9\textwidth]{workflow_propuesto.pdf}
\caption{Flujo actual (dedispersión + candidatos + curaduría) versus flujo propuesto (detección DL + clasificación DL + reporte). Señalar puntos de latencia y reducción de falsos positivos.}
\end{figure}

\newpage