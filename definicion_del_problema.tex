\secnumbersection{Caracterización y formulación del problema}

% Se debe definir el problema, es importante no confundir definir el problema con describir la solución. Por ejemplo: ``diseñar una arquitectura e implementar una plataforma ...'' es una solución, no un problema.
% 
% Algunos elementos que podrían ir en este capítulo son (no es necesario que vayan todos):
% \begin{itemize}
%     \item Breve descripción del contexto donde se realizará la memoria (organización, línea dentro de la Informática en la que se basa, etc.)
%     \item ¿Qué y cómo se realiza actualmente la situación que mejorarás con tu memoria?
%     \item ¿Qué actores o usuarios están involucrados?
%     \item ¿Qué dificultades tienen esos actores actualmente? ¿cuántos son? (ideal si se pueden poner estadísticas para así saber si existe un mercado razonable para la solución que propondrás en tu memoria, en el fondo saber cuántas personas u organizaciones tienen el mismo problema que estás definiendo)
%     \item ¿Qué podría pasar si en el corto o mediano plazo no se solucionan esas dificultades (¿es decir, si no se hiciera tu memoria, qué pasaría?; en el fondo justificar por qué conviene hacer tu memoria, ¿cuál es la motivación o interés de hacerla?).
%     \item ¿Qué competencia existe actualmente? (a lo mejor ya existe una solución al problema, pero por qué no sirve, o por qué tu solución sería mejor, también se puede enfocar a si este problema existe en otras realidades y cómo ha sido solucionado allí).
%     \item Precisar los objetivos y alcances de la memoria (o solución al problema).
% \end{itemize}
% 
% En este capítulo, de ser necesario puede usar referencias bibliográficas (velar porque sean recientes), una cita de ejemplo \cite{schwab2002cure} y otras más \cite{georget1994study,beaumont1990patient}.
% 
% Recuerde poner notas al pie de página que sean explicativas \footnote{Este es un ejemplo de una nota al pie de página. Puede indicar alguna URL, definiciones, aclarar alguna información pertinente del texto, citar algunas referencias, etc..}.
% 
% \subsection{SUBSECCIÓN DE PRUEBA}
% 
% Sed ut perspiciatis unde omnis iste natus error sit voluptatem accusantium doloremque laudantium, totam rem aperiam, eaque ipsa quae ab illo inventore veritatis et quasi architecto beatae vitae dicta sunt explicabo.
% 
% \subsubsection{SUBSUBSECCIÓN DE PRUEBA}
% 
% Nemo enim ipsam voluptatem quia voluptas sit aspernatur aut odit aut fugit, sed quia consequuntur magni dolores eos qui ratione voluptatem sequi nesciunt. Neque porro quisquam est, qui dolorem ipsum quia dolor sit amet.
% 
% \subsubsection{OTRA SUBSUBSECCIÓN DE PRUEBA}
% 
% Nemo enim ipsam voluptatem quia voluptas sit aspernatur aut odit aut fugit, sed quia consequuntur magni dolores eos qui ratione voluptatem sequi nesciunt. Neque porro quisquam est, qui dolorem ipsum quia dolor sit amet.
\subsection{Contexto y caracterización del problema de detección}

El problema central es la ausencia de pipelines operativos para FRBs en régimen milimétrico, ya que la corrección de dispersión (DM) que funciona en frecuencias centimétricas se vuelve inútil a altas frecuencias. Si bien existen métodos y modelos capaces de detectar FRBs mediante aprendizaje profundo (como DRAFTS, que combina detección con CenterNet y clasificación con ResNet), estas aproximaciones permanecen como prototipos de investigación con limitaciones operativas críticas: implementaciones no reproducibles, escasa portabilidad, y ausencia de extensibilidad a regímenes de alta frecuencia sin reentrenar modelos.

En frecuencias decimétricas (300 MHz - 3 GHz), métodos tradicionales (dedispersión, filtrado, umbralización) son relativamente maduros pero demandan curaduría manual extensiva. En contraste, el régimen milimétrico, especialmente en configuraciones \emph{phased array} como ALMA\footnote{Combinación coherente de señales de múltiples antenas para formar series temporales de alta resolución.}, carece de \textit{pipelines} estandarizados. Herramientas clásicas (PRESTO, Heimdall \citep{Ransom2011_PRESTO,Heimdall_Barsdell}) funcionan en bandas L/S pero fallan en mm. DRAFTS ha explorado aproximaciones prometedoras mediante detección en tiempo--DM y clasificación binaria, pero la transición hacia un \textit{pipeline} operativo con ingeniería sólida y extensibilidad entre regímenes frecuenciales permanece como brecha crítica.

Desafíos operativos comunes incluyen: fricción computacional (archivos grandes generando latencia para alertas), altas tasas de falsos positivos (RFI\footnote{RFI: \textit{Radio Frequency Interference}, interferencia de radiofrecuencia de origen terrestre o instrumental.} exigiendo validación manual), y ausencia de pipelines end-to-end con ingeniería robusta, métricas y reportabilidad.


\begin{table}[h]
\centering
\caption{\textbf{Comparación de características técnicas entre regímenes espectrales relevantes para el diseño del pipeline.}}
\vspace{0.5em}
\small
\begin{tabular}{l c c}
\toprule
\textbf{Aspecto} & \textbf{0.3--3\,GHz} & \textbf{30--100\,GHz} \\
\midrule
Retardo por dispersión & Alto; patrones claros & Bajo; patrones \\
& en tiempo--DM & atenuados \\
\midrule
RFI/sistemáticas & RFI ancha banda & Atmósfera, estabilidad \\
& & de fase, distinta RFI \\
\midrule
Productos útiles & Waterfalls, tiempo--DM & Tiempo--frecuencia, \\
& & polarización/diagnósticos \\
\midrule
Disponibilidad & Pipelines consolidados & Prototipos, sin \\
de software & & estándar general \\
\bottomrule
\end{tabular}

\vspace{0.3em}
\raggedright
\small{\textit{Nota: RFI = Radio Frequency Interference (Interferencia de Radiofrecuencia); DM = Dispersion Measure (Medida de Dispersión).}}
\end{table}




\subsection{Actores involucrados y alcance del problema}

Múltiples radiotelescopios globales (CHIME, ASKAP, FAST, MeerKAT, ALMA, entre otros) generan volúmenes masivos de datos y enfrentan altos falsos positivos ($>99\%$), lo que hace inviable la revisión manual. La comunidad multi-mensajero requiere alertas rápidas (minutos/horas) para seguimiento multi-longitud de onda, mientras que decenas de grupos de investigación desarrollan scripts propios, resultando en esfuerzos duplicados y resultados difícilmente comparables \citep{Amiri2021,CHIMECat2_2024,veracasanova2025,Petroff_2022,Rajwade_2024_Review,CHIMEFRBVOEvent2021,vanLeeuwen2023_ARTS}.

\subsection{Diagnóstico del pipeline DRAFTS}

DRAFTS, el pipeline de investigación actual basado en deep learning \citep{Zhang2024_DRAFTS}, presenta limitaciones operativas severas: (i) Arquitectura no integrada (scripts aislados, sin CLI ni configuración reproducible), (ii) Ingesta inflexible (solo PSRFITS de ciertos telescopios, sin detección automática de formato), (iii) Manejo de memoria inestable (procesa bloques muy grandes sin control dinámico, causando fallos en observaciones largas), (iv) Validación y salidas poco robustas (filtros ad hoc, sin logging estructurado ni reporte reproducible), y (v) Extensibilidad limitada (no considera diferentes regímenes ni pruebas automáticas). Si bien sus modelos neuronales (CenterNet y ResNet) son robustos, estas restricciones arquitectónicas dificultan su adopción operativa.

%La Figura~\ref{fig:drafts-arquitectura-problema} ilustra esta limitación arquitectónica fundamental.

% \begin{figure}[htbp]
% \centering
% \resizebox{0.98\textwidth}{!}{%
% \begin{tikzpicture}[
%     node distance=0.8cm and 1.2cm,
%     script/.style={rectangle, draw=red!70, thick, fill=red!10, text width=2.2cm, text centered, minimum height=0.7cm, font=\scriptsize},
%     module/.style={rectangle, draw=blue!70, thick, fill=blue!10, text width=2.2cm, text centered, minimum height=0.7cm, font=\scriptsize},
%     manual/.style={rectangle, draw=orange!70, thick, fill=orange!20, text width=2.0cm, text centered, minimum height=0.5cm, font=\tiny, dashed},
%     auto/.style={rectangle, draw=green!70, thick, fill=green!10, text width=2.0cm, text centered, minimum height=0.5cm, font=\tiny},
%     arrow/.style={-Stealth, thick},
%     manual_arrow/.style={-Stealth, thick, dashed, orange},
%     auto_arrow/.style={-Stealth, thick, green!70!black}
% ]
%
% % ========== LADO IZQUIERDO: DRAFTS ORIGINAL ==========
% \node[font=\bfseries\normalsize, text=red!70!black] at (-5, 5.5) {DRAFTS Original (Prototipo)};
%
% % Scripts desacoplados
% \node[script] (s1) at (-7, 4) {Script 1:\\Leer FITS};
% \node[script] (s2) at (-5, 3) {Script 2:\\Dedispersión};
% \node[script] (s3) at (-3, 4) {Script 3:\\CenterNet};
% \node[script] (s4) at (-5, 1.5) {Script 4:\\ResNet18};
% \node[script] (s5) at (-7, 0.5) {Script 5:\\Guardar};
%
% % Intervenciones manuales
% \node[manual] (m1) at (-7, 2.5) {Editar\\rutas};
% \node[manual] (m2) at (-3, 2.5) {Configurar\\DM};
% \node[manual] (m3) at (-5, 0) {Mover\\archivos};
%
% % Conexiones manuales (discontinuas)
% \draw[manual_arrow] (s1) -- (m1);
% \draw[manual_arrow] (m1) -- (s2);
% \draw[manual_arrow] (s2) -- (m2);
% \draw[manual_arrow] (m2) -- (s3);
% \draw[manual_arrow] (s3) -- (s4);
% \draw[manual_arrow] (s4) -- (m3);
% \draw[manual_arrow] (m3) -- (s5);
%
% % Etiquetas de problemas
% \node[font=\tiny, text=red!70!black, align=center] at (-5, -0.8) {$\times$ Sin integración\\$\times$ Parámetros codificados\\$\times$ Sin trazabilidad};
%
% % ========== LADO DERECHO: DRAFTS++ PROPUESTO ==========
% \node[font=\bfseries\normalsize, text=blue!70!black] at (5, 5.5) {DRAFTS++ (Sistema Productivo)};
%
% % Pipeline modular
% \node[module] (p1) at (2.5, 4.2) {Ingesta\\Multi-formato};
% \node[module] (p2) at (5, 4.2) {Preproceso\\Adaptativo};
% \node[module] (p3) at (7.5, 4.2) {Modelos\\CenterNet/ResNet};
% \node[module] (p4) at (5, 2.5) {Validación\\Física};
% \node[module] (p5) at (5, 0.8) {Artefactos\\Estandarizados};
%
% % Servicios transversales
% \node[auto, rounded corners] (cfg) at (2.5, 1.5) {Config\\Auto};
% \node[auto, rounded corners] (mem) at (7.5, 2.5) {Memoria\\Dinámica};
% \node[auto, rounded corners] (log) at (7.5, 0.8) {Logging\\Completo};
%
% % Conexiones automatizadas (continuas, verdes)
% \draw[auto_arrow] (p1) -- (p2);
% \draw[auto_arrow] (p2) -- (p3);
% \draw[auto_arrow] (p3) -- (p4);
% \draw[auto_arrow] (p4) -- (p5);
%
% % Conexiones de servicios
% \draw[auto_arrow, dotted] (cfg) -- (p1);
% \draw[auto_arrow, dotted] (cfg) -- (p2);
% \draw[auto_arrow, dotted] (mem) -- (p3);
% \draw[auto_arrow, dotted] (mem) -- (p4);
% \draw[auto_arrow, dotted] (log) -- (p5);
%
% % Input/Output
% \node[rectangle, draw=black, thick, fill=gray!20, text width=1.8cm, text centered, font=\scriptsize] (in) at (2.5, 5.5) {Datos\\Astronómicos};
% \node[rectangle, draw=black, thick, fill=gray!20, text width=1.8cm, text centered, font=\scriptsize] (out) at (5, -0.5) {CSV + Plots\\+ Métricas};
%
% \draw[arrow] (in) -- (p1);
% \draw[arrow] (p5) -- (out);
%
% % Etiquetas de beneficios
% \node[font=\tiny, text=green!70!black, align=center] at (5, -1.3) {$\checkmark$ Flujo unificado\\$\checkmark$ Configuración externa\\$\checkmark$ Reproducible};
%
% % Separador vertical
% \draw[thick, gray, dashed] (0, -1.5) -- (0, 6);
% \node[font=\small, text=gray, rotate=90] at (0.5, 2.5) {vs.};
%
% \end{tikzpicture}%
% }
% \caption{Comparación arquitectónica entre DRAFTS original y DRAFTS++ propuesto. \textbf{Izquierda (DRAFTS original):} Conjunto de scripts independientes que requieren intervención manual para configuración, coordinación y transferencia de datos entre etapas. El flujo es discontinuo (flechas naranjas punteadas), con parámetros codificados en el código fuente y sin mecanismos de trazabilidad. \textbf{Derecha (DRAFTS++ propuesto):} Pipeline modular end-to-end con flujo automatizado (flechas verdes continuas), servicios transversales (configuración automática, gestión dinámica de memoria, logging estructurado) y artefactos estandarizados. Un único comando procesa datos astronómicos y genera salidas reproducibles sin intervención manual. Esta transformación arquitectónica constituye el núcleo del problema abordado en esta tesis: convertir un prototipo de investigación en un sistema operativo productivo.}
% \label{fig:drafts-arquitectura-problema}
% \end{figure}

\subsection{Brecha específica: extensión a alta frecuencia}

Más allá de las limitaciones arquitectónicas de DRAFTS, existe una segunda brecha crítica: la extensión a alta frecuencia. Como se detalló en la sección 2.8, el régimen milimétrico enfrenta desafíos únicos que representan brechas científicas y de ingeniería críticas. Las siguientes subsecciones caracterizan las capacidades faltantes en pipelines existentes.

\subsubsection{Adaptaciones algorítmicas necesarias para pipelines en mm}

Pipelines de frecuencias centimétricas \textbf{requieren modificaciones fundamentales} en mm. En mm, la curvatura dispersiva casi desaparece\footnote{Un DM de $\sim1000$ pc cm$^{-3}$ genera barrido de $\sim1.8$ s a 1.4 GHz versus solo $\sim0.05$ s a 86 GHz \citep{veracasanova2025}.}, por lo que se requieren rejillas DM mucho más gruesas y decisiones adaptativas (ej. cuándo integrar directamente en frecuencia en vez de dedispersar exhaustivamente). Para búsquedas ciegas de FRBs, \textbf{no existe actualmente} estrategia estandarizada que maneje: reducción automática de la granularidad de DM según la frecuencia observacional, decisión adaptativa sobre cuándo aplicar integración directa en frecuencia vs. dedispersión completa, cálculo dinámico del número óptimo de pruebas de DM basado en ancho de banda y frecuencia central, y configuración paramétrica de estos aspectos sin modificar código fuente. Además, sin la curva $1/\nu^2$ se requieren nuevos criterios de validación, como la polarización o coherencia entre antenas, dado que en mm no hay ``bow-tie'' para distinguir RFI. El régimen mm exige manejar datos con muestreos ultra-rápidos ($\sim\mu$s) y flujos de $>10$ Gbps, lo que demanda control dinámico de memoria, procesamiento paralelo (multi-GPU) y downsampling inteligente, capacidades ausentes en DRAFTS. Estas capacidades permanecen como implementaciones \emph{ad hoc} específicas de cada experimento, sin una infraestructura de software reutilizable.

\subsubsection{Productos diagnósticos específicos para alta frecuencia}

Finalmente, en mm se requieren productos de diagnóstico distintos a los típicos (tiempo-DM), como análisis de polarización de cada pulso, coherencia entre antenas para confirmar origen celeste, y espectrogramas de ultra-alta resolución para microestructuras; ningún pipeline actual los genera automáticamente. En suma, la extensión a mm requiere reformulación algorítmica, nuevos criterios de validación, optimización computacional, y productos diagnósticos específicos. Ningún pipeline actual --incluido DRAFTS-- incorpora estas capacidades de manera sistemática y configurable. Esta brecha constituye la segunda línea de investigación.

\subsection{Formulación del problema}

Con el diagnóstico completo de limitaciones arquitectónicas (DRAFTS) y brechas de alta frecuencia (régimen mm) establecido, se formula el problema central:

\medskip

\noindent No existe actualmente un \textit{pipeline} operativo, reproducible y portable que procese datos de radioastronomía (FITS/PSRFITS u otros formatos) utilizando modelos de aprendizaje profundo preentrenados (detección y clasificación de FRBs) y que:

\begin{enumerate}
\item Procese datos con control eficiente de recursos computacionales y de memoria,
\item Reduzca la tasa de falsos positivos mediante una segunda criba basada en aprendizaje profundo,
\item Genere salidas auditables con metadatos completos (catálogo de candidatos, recortes tiempo--frecuencia, figuras diagnósticas y métricas de rendimiento),
\item Sea extensible a 30--100 GHz mediante parametrización (rejillas DM, escalas temporales, etc.),
\item Todo ello \textbf{sin requerir reentrenamiento} de los modelos base para cada nuevo instrumento o régimen espectral.
\end{enumerate}

\medskip

Esta ausencia constituye una barrera crítica para: (i) la operación eficiente de programas de detección de FRBs en tiempo casi-real, (ii) la exploración sistemática del régimen milimétrico con instrumentos como ALMA, y (iii) la comparabilidad y reproducibilidad de resultados entre diferentes campañas observacionales.

\subsection{Consecuencias de no abordar el problema}

De persistir estas limitaciones, se perderían oportunidades científicas (alertas tardías impiden seguimientos oportunos), se desaprovecharía infraestructura puntera (ALMA en modo \emph{phased} no rendiría su potencial al no detectar FRBs en mm), y los catálogos resultantes serían incompletos o sesgados, comprometiendo estudios cosmológicos \citep{Petroff_2022,veracasanova2025,Torne2021}.


\subsection{Objetivos y alcances de la memoria}

Establecida la naturaleza del problema y sus consecuencias, se definen objetivos específicos y alcances de la solución propuesta.

\textbf{Objetivo General}

Desarrollar un \textit{pipeline} astronómico operativo para detección y clasificación de FRBs basado en dos CNN preentrenadas, extensible a regímenes de alta frecuencia sin reentrenamiento de modelos.

\medskip

\noindent\textbf{Objetivos}

\begin{enumerate}
\item Integrar modelos de detección/clasificación en flujo robusto end-to-end con optimizaciones de rendimiento, gestión de memoria y logging completo.
\item Desarrollar estrategias de detección adaptadas para alta frecuencia que superen las limitaciones de pipelines basados en firmas dispersivas, mediante enfoques híbridos que combinen técnicas clásicas (matched filtering) con clasificación mediante deep learning, sin reentrenamiento de modelos.
\item Validar sobre datos previamente analizados, igualando/superando recuentos y caracterizando latencia, \emph{precision} y \emph{recall}.
\end{enumerate}

\medskip

El alcance de esta memoria se estructura en dos componentes complementarios:

\textbf{Componente 1 - Pipeline Operativo Productivo:} Transformar DRAFTS prototipo en \textbf{DRAFTS++ operativo}: un pipeline end-to-end modular, reproducible, operando sobre formatos estándar (FITS/PSRFITS) sin requerir intervención manual entre etapas.

\textbf{Componente 2 - Extensión a Alta Frecuencia:} Se implementan y validan empíricamente dos líneas metodológicas para detección en régimen milimétrico: (1) validación del pipeline clásico DM-Time/CenterNet mediante adaptación paramétrica, y (2) estrategia híbrida SNR-threshold + clasificación CNN. Las Líneas 3 y 4 quedan propuestas como trabajo futuro.

\textbf{No se contempla:} reentrenamiento de modelos, \emph{backends} instrumentales, almacenamiento distribuido, ni integración con sistemas externos de alertas (aunque se proveen salidas necesarias).


\newpage