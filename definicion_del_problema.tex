\secnumbersection{DEFINICIÓN DEL PROBLEMA}

% Se debe definir el problema, es importante no confundir definir el problema con describir la solución. Por ejemplo: ``diseñar una arquitectura e implementar una plataforma ...'' es una solución, no un problema.
% 
% Algunos elementos que podrían ir en este capítulo son (no es necesario que vayan todos):
% \begin{itemize}
%     \item Breve descripción del contexto donde se realizará la memoria (organización, línea dentro de la Informática en la que se basa, etc.)
%     \item ¿Qué y cómo se realiza actualmente la situación que mejorarás con tu memoria?
%     \item ¿Qué actores o usuarios están involucrados?
%     \item ¿Qué dificultades tienen esos actores actualmente? ¿cuántos son? (ideal si se pueden poner estadísticas para así saber si existe un mercado razonable para la solución que propondrás en tu memoria, en el fondo saber cuántas personas u organizaciones tienen el mismo problema que estás definiendo)
%     \item ¿Qué podría pasar si en el corto o mediano plazo no se solucionan esas dificultades (¿es decir, si no se hiciera tu memoria, qué pasaría?; en el fondo justificar por qué conviene hacer tu memoria, ¿cuál es la motivación o interés de hacerla?).
%     \item ¿Qué competencia existe actualmente? (a lo mejor ya existe una solución al problema, pero por qué no sirve, o por qué tu solución sería mejor, también se puede enfocar a si este problema existe en otras realidades y cómo ha sido solucionado allí).
%     \item Precisar los objetivos y alcances de la memoria (o solución al problema).
% \end{itemize}
% 
% En este capítulo, de ser necesario puede usar referencias bibliográficas (velar porque sean recientes), una cita de ejemplo \cite{schwab2002cure} y otras más \cite{georget1994study,beaumont1990patient}.
% 
% Recuerde poner notas al pie de página que sean explicativas \footnote{Este es un ejemplo de una nota al pie de página. Puede indicar alguna URL, definiciones, aclarar alguna información pertinente del texto, citar algunas referencias, etc..}.
% 
% \subsection{SUBSECCIÓN DE PRUEBA}
% 
% Sed ut perspiciatis unde omnis iste natus error sit voluptatem accusantium doloremque laudantium, totam rem aperiam, eaque ipsa quae ab illo inventore veritatis et quasi architecto beatae vitae dicta sunt explicabo.
% 
% \subsubsection{SUBSUBSECCIÓN DE PRUEBA}
% 
% Nemo enim ipsam voluptatem quia voluptas sit aspernatur aut odit aut fugit, sed quia consequuntur magni dolores eos qui ratione voluptatem sequi nesciunt. Neque porro quisquam est, qui dolorem ipsum quia dolor sit amet.
% 
% \subsubsection{OTRA SUBSUBSECCIÓN DE PRUEBA}
% 
% Nemo enim ipsam voluptatem quia voluptas sit aspernatur aut odit aut fugit, sed quia consequuntur magni dolores eos qui ratione voluptatem sequi nesciunt. Neque porro quisquam est, qui dolorem ipsum quia dolor sit amet.
\subsection{Contexto y caracterización del problema de detección}

El capítulo anterior estableció el marco conceptual de los FRBs: desde su fenomenología observacional (dispersión, scattering, polarización), pasando por casos históricos y conexión con magnetars, hasta métodos de detección tradicionales y soluciones basadas en aprendizaje automático. Se destacó DRAFTS como estado del arte, logrando tasas de recuperación cercanas al 100\% mediante detección (CenterNet) y clasificación (ResNet). También se identificaron desafíos específicos del régimen milimétrico: sensibilidad reducida, compresión de la firma dispersiva, y necesidad de validación alternativa.

\medskip

Partiendo de ese contexto, este capítulo define el problema específico abordado. Si bien existen métodos y modelos capaces de detectar FRBs mediante aprendizaje profundo, estas aproximaciones permanecen como prototipos de investigación con limitaciones operativas críticas: implementaciones no reproducibles, escasa portabilidad, y ausencia de extensibilidad a regímenes de alta frecuencia sin reentrenar modelos.

\medskip

La búsqueda de FRBs opera en dos regímenes espectrales distintos. En frecuencias decimétricas (300 MHz - 3 GHz), métodos tradicionales (dedispersión, filtrado, umbralización) son relativamente maduros pero demandan curaduría manual extensiva. En contraste, el régimen milimétrico (30--100 GHz), especialmente en configuraciones \emph{phased array} como ALMA\footnote{Combinación coherente de señales de múltiples antenas para formar series temporales de alta resolución.}, carece de \textit{pipelines} estandarizados. Aquí, el retardo dispersivo $\Delta t\propto \mathrm{DM}\,\nu^{-2}$ se comprime, los patrones \emph{bow-tie} se atenúan, y los desarrollos permanecen como prototipos.

Herramientas clásicas (PRESTO, Heimdall \citep{Ransom2011_PRESTO,Heimdall_Barsdell}) funcionan en bandas L/S pero fallan en mm. DRAFTS ha explorado aproximaciones prometedoras mediante detección en tiempo--DM y clasificación binaria, pero la transición hacia un \textit{pipeline} operativo con ingeniería sólida y extensibilidad entre regímenes frecuenciales permanece como brecha crítica.

Desafíos operativos comunes incluyen: fricción computacional (archivos grandes generando latencia para alertas), altas tasas de falsos positivos (RFI exigiendo validación manual), y ausencia de pipelines end-to-end con ingeniería robusta, métricas y reportabilidad.


\begin{table}[h]
\centering
\caption{\textbf{Comparación de características técnicas entre regímenes espectrales relevantes para el diseño del pipeline.}}
\vspace{0.5em}
\small
\begin{tabular}{l c c}
\toprule
\textbf{Aspecto} & \textbf{0.3--3\,GHz} & \textbf{30--100\,GHz} \\
\midrule
Retardo por dispersión & Alto; patrones claros & Bajo; patrones \\
& en tiempo--DM & atenuados \\
\midrule
RFI/sistemáticas & RFI ancha banda & Atmósfera, estabilidad \\
& & de fase, distinta RFI \\
\midrule
Productos útiles & Waterfalls, tiempo--DM & Tiempo--frecuencia, \\
& & polarización/diagnósticos \\
\midrule
Disponibilidad & Pipelines consolidados & Prototipos, sin \\
de software & & estándar general \\
\bottomrule
\end{tabular}

\vspace{0.3em}
\raggedright
\small{\textit{Nota: RFI = Radio Frequency Interference (Interferencia de Radiofrecuencia); DM = Dispersion Measure (Medida de Dispersión).}}
\end{table}




\subsection{Actores involucrados y alcance del problema}

Habiendo caracterizado el contexto técnico, es importante identificar quiénes se ven afectados por las limitaciones actuales. Los principales actores incluyen:

\textbf{Observatorios y colaboraciones:} CHIME (Canadá), ASKAP (Australia), FAST (China), MeerKAT (Sudáfrica) y ALMA (Chile) operan programas de detección generando volúmenes masivos de datos \citep{Amiri2021,CHIMECat2_2024,veracasanova2025}. Solo CHIME ha detectado miles de FRBs desde 2018, requiriendo procesamiento sistemático.

\textbf{Grupos de investigación:} Decenas de equipos activos desarrollan scripts de análisis propios, resultando en esfuerzos duplicados y resultados difícilmente comparables \citep{Petroff_2022,Rajwade_2024_Review}.

\textbf{Operadores de telescopios:} Requieren herramientas automatizadas para validar detecciones en tiempo real. La curaduría manual (horas/días) compromete la respuesta ante transitorios \citep{Rajwade_2024_Review}.

\textbf{Comunidad multi-mensajero:} El seguimiento multi-longitud de onda (óptico, rayos X) requiere alertas dentro de minutos/horas \citep{CHIMEFRBVOEvent2021,vanLeeuwen2023_ARTS}.

\medskip

\textit{Magnitud del problema:} Cada campaña genera miles-decenas de miles de candidatos con $>99\%$ falsos positivos \citep{Rajwade_2024_Review,Agarwal2020}. La curaduría manual (segundos-minutos por evento) resulta en cargas de horas/días inviables para tiempo real \citep{Goode2022_DWF,Wagstaff2016}.

\subsection{Diagnóstico del pipeline DRAFTS}

Conocidos los actores afectados, el siguiente paso es diagnosticar las limitaciones específicas del estado del arte. DRAFTS representa el pipeline más avanzado basado en aprendizaje profundo \citep{Zhang2024_DRAFTS}, combinando detección (CenterNet) y clasificación (ResNet) con alta completitud. Si bien sus modelos neuronales son robustos, su arquitectura de software presenta restricciones críticas que dificultan adopción operativa.

\textbf{DRAFTS es un "pseudo-pipeline" de investigación, no un sistema operativo.} Problemas clave:

\textbf{Arquitectura desacoplada:} Scripts independientes requieren ejecución manual por etapas. No existe punto de entrada unificado, CLI, ni configuración versionada. El README instruye "modificar rutas en código", evidenciando ausencia de automatización end-to-end.

\textbf{Ingesta rígida:} Lector PSRFITS asume esquema específico FAST/GBT. Otros telescopios requieren "modificar funciones internas". Sin detección automática de formato, análisis robusto de headers, ni timestamps precisos. Parámetros con números mágicos y correcciones \textit{ad hoc}.

\textbf{Gestión de memoria frágil:} Procesamiento monolítico sobre bloques grandes. Relleno sintético con ruido aleatorio en bordes (altera estadística). Dedispersión GPU sin planificador de \textit{chunking}, telemetría VRAM, ni \textit{fallback} CPU. Inestable en observaciones prolongadas.

\textbf{Validación desacoplada:} Detección y clasificación en programas separados. Filtrado heurístico sin validador físico común. Sin desduplicación entre \textit{slices}/archivos. Salidas limitadas (imágenes + .npy); sin CSV/Parquet con metadatos, logging estructurado, ni reproducibilidad.

\textbf{Extensibilidad limitada:} Modelos en rutas relativas (falla si ausentes). Sin empaquetado, pruebas automatizadas, ni selección automática de estrategias según condiciones físicas.

%La Figura~\ref{fig:drafts-arquitectura-problema} ilustra esta limitación arquitectónica fundamental.

% \begin{figure}[htbp]
% \centering
% \resizebox{0.98\textwidth}{!}{%
% \begin{tikzpicture}[
%     node distance=0.8cm and 1.2cm,
%     script/.style={rectangle, draw=red!70, thick, fill=red!10, text width=2.2cm, text centered, minimum height=0.7cm, font=\scriptsize},
%     module/.style={rectangle, draw=blue!70, thick, fill=blue!10, text width=2.2cm, text centered, minimum height=0.7cm, font=\scriptsize},
%     manual/.style={rectangle, draw=orange!70, thick, fill=orange!20, text width=2.0cm, text centered, minimum height=0.5cm, font=\tiny, dashed},
%     auto/.style={rectangle, draw=green!70, thick, fill=green!10, text width=2.0cm, text centered, minimum height=0.5cm, font=\tiny},
%     arrow/.style={-Stealth, thick},
%     manual_arrow/.style={-Stealth, thick, dashed, orange},
%     auto_arrow/.style={-Stealth, thick, green!70!black}
% ]
%
% % ========== LADO IZQUIERDO: DRAFTS ORIGINAL ==========
% \node[font=\bfseries\normalsize, text=red!70!black] at (-5, 5.5) {DRAFTS Original (Prototipo)};
%
% % Scripts desacoplados
% \node[script] (s1) at (-7, 4) {Script 1:\\Leer FITS};
% \node[script] (s2) at (-5, 3) {Script 2:\\Dedispersión};
% \node[script] (s3) at (-3, 4) {Script 3:\\CenterNet};
% \node[script] (s4) at (-5, 1.5) {Script 4:\\ResNet18};
% \node[script] (s5) at (-7, 0.5) {Script 5:\\Guardar};
%
% % Intervenciones manuales
% \node[manual] (m1) at (-7, 2.5) {Editar\\rutas};
% \node[manual] (m2) at (-3, 2.5) {Configurar\\DM};
% \node[manual] (m3) at (-5, 0) {Mover\\archivos};
%
% % Conexiones manuales (discontinuas)
% \draw[manual_arrow] (s1) -- (m1);
% \draw[manual_arrow] (m1) -- (s2);
% \draw[manual_arrow] (s2) -- (m2);
% \draw[manual_arrow] (m2) -- (s3);
% \draw[manual_arrow] (s3) -- (s4);
% \draw[manual_arrow] (s4) -- (m3);
% \draw[manual_arrow] (m3) -- (s5);
%
% % Etiquetas de problemas
% \node[font=\tiny, text=red!70!black, align=center] at (-5, -0.8) {$\times$ Sin integración\\$\times$ Parámetros codificados\\$\times$ Sin trazabilidad};
%
% % ========== LADO DERECHO: DRAFTS++ PROPUESTO ==========
% \node[font=\bfseries\normalsize, text=blue!70!black] at (5, 5.5) {DRAFTS++ (Sistema Productivo)};
%
% % Pipeline modular
% \node[module] (p1) at (2.5, 4.2) {Ingesta\\Multi-formato};
% \node[module] (p2) at (5, 4.2) {Preproceso\\Adaptativo};
% \node[module] (p3) at (7.5, 4.2) {Modelos\\CenterNet/ResNet};
% \node[module] (p4) at (5, 2.5) {Validación\\Física};
% \node[module] (p5) at (5, 0.8) {Artefactos\\Estandarizados};
%
% % Servicios transversales
% \node[auto, rounded corners] (cfg) at (2.5, 1.5) {Config\\Auto};
% \node[auto, rounded corners] (mem) at (7.5, 2.5) {Memoria\\Dinámica};
% \node[auto, rounded corners] (log) at (7.5, 0.8) {Logging\\Completo};
%
% % Conexiones automatizadas (continuas, verdes)
% \draw[auto_arrow] (p1) -- (p2);
% \draw[auto_arrow] (p2) -- (p3);
% \draw[auto_arrow] (p3) -- (p4);
% \draw[auto_arrow] (p4) -- (p5);
%
% % Conexiones de servicios
% \draw[auto_arrow, dotted] (cfg) -- (p1);
% \draw[auto_arrow, dotted] (cfg) -- (p2);
% \draw[auto_arrow, dotted] (mem) -- (p3);
% \draw[auto_arrow, dotted] (mem) -- (p4);
% \draw[auto_arrow, dotted] (log) -- (p5);
%
% % Input/Output
% \node[rectangle, draw=black, thick, fill=gray!20, text width=1.8cm, text centered, font=\scriptsize] (in) at (2.5, 5.5) {Datos\\Astronómicos};
% \node[rectangle, draw=black, thick, fill=gray!20, text width=1.8cm, text centered, font=\scriptsize] (out) at (5, -0.5) {CSV + Plots\\+ Métricas};
%
% \draw[arrow] (in) -- (p1);
% \draw[arrow] (p5) -- (out);
%
% % Etiquetas de beneficios
% \node[font=\tiny, text=green!70!black, align=center] at (5, -1.3) {$\checkmark$ Flujo unificado\\$\checkmark$ Configuración externa\\$\checkmark$ Reproducible};
%
% % Separador vertical
% \draw[thick, gray, dashed] (0, -1.5) -- (0, 6);
% \node[font=\small, text=gray, rotate=90] at (0.5, 2.5) {vs.};
%
% \end{tikzpicture}%
% }
% \caption{Comparación arquitectónica entre DRAFTS original y DRAFTS++ propuesto. \textbf{Izquierda (DRAFTS original):} Conjunto de scripts independientes que requieren intervención manual para configuración, coordinación y transferencia de datos entre etapas. El flujo es discontinuo (flechas naranjas punteadas), con parámetros codificados en el código fuente y sin mecanismos de trazabilidad. \textbf{Derecha (DRAFTS++ propuesto):} Pipeline modular end-to-end con flujo automatizado (flechas verdes continuas), servicios transversales (configuración automática, gestión dinámica de memoria, logging estructurado) y artefactos estandarizados. Un único comando procesa datos astronómicos y genera salidas reproducibles sin intervención manual. Esta transformación arquitectónica constituye el núcleo del problema abordado en esta tesis: convertir un prototipo de investigación en un sistema operativo productivo.}
% \label{fig:drafts-arquitectura-problema}
% \end{figure}

\subsection{Brecha específica: extensión a alta frecuencia}

Más allá de las limitaciones arquitectónicas de DRAFTS, existe una segunda brecha crítica: la extensión a alta frecuencia. Como se detalló en la sección 2.8, el régimen milimétrico enfrenta desafíos únicos que representan brechas científicas y de ingeniería críticas. Las siguientes subsecciones caracterizan las capacidades faltantes en pipelines existentes.

\subsubsection{Adaptaciones algorítmicas necesarias para pipelines en mm}

Pipelines de frecuencias centimétricas \textbf{requieren modificaciones fundamentales} en mm. El desafío principal: la firma dispersiva se reduce drásticamente. Un DM de $\sim1000$ pc cm$^{-3}$ genera barrido de $\sim1.8$ s a 1.4 GHz versus solo $\sim0.05$ s a 86 GHz \citep{veracasanova2025}, con implicaciones fundamentales para algoritmos de búsqueda.

El único estudio exitoso con ALMA (86 GHz) usó PRESTO adaptado manualmente con DM conocido (1770 pc cm$^{-3}$), evitando barrido multi-DM \citep{veracasanova2025}. Para búsquedas ciegas de FRBs, \textbf{no existe actualmente} estrategia estandarizada que maneje:

\begin{itemize}
\item Reducción automática de la granularidad de DM según la frecuencia observacional (p.ej. $\Delta$DM de 5--10 unidades en mm vs. $\sim1$ en cm)
\item Decisión adaptativa sobre cuándo aplicar integración directa en frecuencia vs. dedispersión completa
\item Cálculo dinámico del número óptimo de pruebas de DM basado en ancho de banda y frecuencia central
\item Configuración paramétrica de estos aspectos sin modificar código fuente
\end{itemize}

\subsubsection{Estrategias de validación sin curvatura de dispersión apreciable}

Complementando las adaptaciones algorítmicas, el segundo desafío es validación sin firma dispersiva. El indicador clave de FRBs --trazo curvado $1/\nu^2$-- se vuelve imperceptible en mm. A frecuencias centimétricas, candidatos se validan por forma de ``bow tie'' en DM-tiempo. En banda 90--100 GHz, un FRB con DM elevado aparecería casi simultáneo en frecuencias, indistinguible de RFI con DM=0.

Experimentos con ALMA han demostrado validación alternativa por polarización: 5 de 8 pulsos del magnetar mostraron $>90\%$ polarización \citep{veracasanova2025}. Sin embargo, \textbf{ningún pipeline implementa sistemáticamente}:

\begin{itemize}
\item Análisis automático de polarización (Stokes Q/U/V) como criterio de validación primario
\item Verificación de coherencia multi-antena para discriminar eventos celestes de RFI local
\item Estrategias de validación multi-banda (coordinación automática entre observaciones en cm y mm)
\item Umbrales adaptativos de SNR según la disponibilidad de información auxiliar (polarización, multi-antena, etc.)
\end{itemize}

Estas capacidades permanecen como implementaciones \emph{ad hoc} específicas de cada experimento, sin una infraestructura de software reutilizable.

\subsubsection{Exigencias computacionales del procesamiento en tiempo real}

El tercer desafío es computacional. Datos mm con pulsos breves requieren muestreos sub-milisegundo (hasta $\sim8$ $\mu$s en ALMA \citep{veracasanova2025}), implicando flujos voluminosos: 100,000 muestras/s × 32 canales × 2 polarizaciones = millones de muestras/s.

\textbf{Limitaciones actuales} de DRAFTS:

\begin{itemize}
\item Ausencia de control dinámico de memoria GPU (no hay telemetría de VRAM, \emph{fallback} a CPU ante out-of-memory, o limpieza sistemática)
\item Falta de arquitectura de \emph{chunking} adaptativa que ajuste el tamaño de bloques según recursos disponibles
\item Inexistencia de estrategias de \emph{downsampling} inteligente específicas para alta frecuencia
\item Carencia de mecanismos de paralelización multi-GPU o distribución entre nodos para manejar flujos de 10--100 Gbps
\end{itemize}

Estas limitaciones hacen inviable el procesamiento en tiempo real de observaciones largas (>1 hora) en mm con los pipelines actuales, comprometiendo la capacidad de generar alertas oportunas.

\subsubsection{Productos diagnósticos específicos para alta frecuencia}

Finalmente, el régimen mm \textbf{requiere productos diagnósticos diferenciados} respecto a búsquedas en cm (diagramas DM--tiempo estándar):

\begin{itemize}
\item Análisis de polarización por pulso (fracciones de Stokes, ángulos de posición, RM si hay banda ancha)
\item Mapas de coherencia temporal multi-antena para validar origen celeste
\item Análisis espectral de ultra-alta resolución ($<10$ $\mu$s) para buscar microestructuras
\item Comparación automática con detecciones simultáneas en otras bandas (si disponibles)
\end{itemize}

Actualmente, \textbf{ningún pipeline genera estos productos automáticamente}. Investigadores extraen datos de polarización, escriben scripts personalizados, y comparan multi-banda manualmente, dificultando comparabilidad y ralentizando análisis científico.

\medskip

En suma, la extensión a mm no es ``cambiar un parámetro'': requiere reformulación algorítmica, nuevos criterios de validación, optimización computacional, y productos diagnósticos específicos. Ningún pipeline actual --incluido DRAFTS-- incorpora estas capacidades de manera sistemática y configurable. Esta brecha constituye la segunda línea de investigación.

\subsection{Formulación del problema}

Con el diagnóstico completo de limitaciones arquitectónicas (DRAFTS) y brechas de alta frecuencia (régimen mm) establecido, se formula el problema central:

\medskip

\noindent\textbf{Dado} un flujo de datos de radioastronomía (FITS/PSRFITS u otros formatos) y dos modelos de aprendizaje profundo preentrenados (detección y clasificación de FRBs), \textbf{no existe actualmente} un \textit{pipeline} operativo, reproducible y portable que:

\begin{enumerate}
\item Procese datos en lotes y en línea con control eficiente de recursos computacionales y de memoria,
\item Reduzca la tasa de falsos positivos mediante una segunda criba basada en aprendizaje profundo,
\item Genere salidas completamente auditables (catálogo de candidatos con metadatos, recortes tiempo--frecuencia, figuras diagnósticas y métricas de rendimiento),
\item Sea extensible a regímenes de alta frecuencia (30--100 GHz) mediante parametrización adecuada de rejillas DM, escalas temporales y productos diagnósticos (incluida polarización cuando esté disponible),
\item Todo ello \textbf{sin requerir reentrenamiento} de los modelos base para cada nuevo instrumento o régimen espectral.
\end{enumerate}

\medskip

Esta ausencia constituye una barrera crítica para: (i) la operación eficiente de programas de detección de FRBs en tiempo casi-real, (ii) la exploración sistemática del régimen milimétrico con instrumentos como ALMA, y (iii) la comparabilidad y reproducibilidad de resultados entre diferentes campañas observacionales.

\medskip

El problema abordado en esta memoria se enfoca específicamente en dos líneas de investigación complementarias: \textbf{(1) Desarrollo de software end-to-end funcional y eficiente}, aplicando principios rigurosos de ingeniería de software para transformar un prototipo experimental en un sistema operativo robusto; y \textbf{(2) Extensión a alta frecuencia}, mediante la incorporación de estrategias algorítmicas, productos diagnósticos y mecanismos de validación específicos para el régimen milimétrico, todo ello sin requerir reentrenamiento de los modelos neuronales base.

\subsection{Consecuencias de no abordar el problema}

¿Qué sucede si estas limitaciones persisten? Las consecuencias incluyen:

\textbf{Pérdida de oportunidades científicas:} Latencia actual en validación (horas-días) impide coordinar seguimiento multi-longitud de onda en tiempo útil, perdiendo observaciones complementarias críticas \citep{Petroff_2022,Rajwade_2024_Review}.

\textbf{Subutilización de infraestructura mm:} Inversiones millonarias en ALMA \emph{phased array} no se aprovechan plenamente. Se pierde la ventana de primeras detecciones en mm donde cada descubrimiento podría revolucionar comprensión de mecanismos de emisión \citep{veracasanova2025,Torne2021}.

\textbf{Esfuerzo científico ineficiente:} Cada grupo desarrolla herramientas \emph{ad hoc}, duplicando esfuerzos y generando resultados incomparables, fragmentando conocimiento \citep{Rajwade_2024_Review}.

\textbf{Catálogos sesgados:} Incompletitud de pipelines subóptimos sesga distribuciones observadas (energías, tasas, espectros), comprometiendo estudios cosmológicos que requieren muestras completas \citep{Petroff_2022}.

\textbf{Exclusión comunitaria:} Barrera técnica limita participación de grupos con menor experiencia, concentrando ciencia en pocas instituciones \citep{Rajwade_2024_Review}.


\subsection{Objetivos y alcances de la memoria}

Establecida la naturaleza del problema y sus consecuencias, se definen objetivos específicos y alcances de la solución propuesta.

\textbf{Objetivo General}

Desarrollar un \textit{pipeline} astronómico operativo para detección y clasificación de FRBs basado en dos CNN preentrenadas, extensible a regímenes de alta frecuencia sin reentrenamiento de modelos.

\medskip

\noindent\textbf{Objetivos Específicos}

\begin{enumerate}
\item Integrar modelos de detección/clasificación en flujo robusto: ingesta $\to$ preprocesamiento $\to$ inferencia $\to$ posprocesamiento $\to$ reporte auditable.
\item Implementar optimizaciones: procesamiento eficiente, gestión de memoria, aceleración computacional y logging completo.
\item Parametrizar el flujo para alta frecuencia (rejillas DM, ventanas temporales, Stokes/polarización) sin reentrenamiento.
\item Validar sobre datos previamente analizados, igualando/superando recuentos y caracterizando latencia, \emph{precision} y \emph{recall}.
\item Implementar productos diagnósticos para mm y caracterizar propiedades distintivas de FRBs en alta frecuencia.
\end{enumerate}

\medskip

\noindent\textbf{Alcances}

El alcance de esta memoria se estructura en dos ejes complementarios que definen precisamente qué se implementará y validará:

\textbf{Eje 1 - Pipeline Operativo Productivo:} El alcance central es transformar DRAFTS prototipo en \textbf{DRAFTS++ operativo}: un pipeline de inferencia end-to-end completamente funcional, listo para usar mediante CLI, con documentación completa, gestión robusta de memoria, streaming de datos masivos, integración automatizada de modelos CenterNet/ResNet18, y generación de artefactos estandarizados. Este pipeline debe operar sobre formatos estándar (FITS/PSRFITS) sin requerir intervención manual entre etapas. \textit{Este es el producto entregable principal de la memoria.}

\textbf{Eje 2 - Extensión a Alta Frecuencia (4 líneas exploratorias):} Se \textit{exploran} cuatro líneas metodológicas para detección en régimen milimétrico donde la firma dispersiva se comprime. Sin embargo, \textbf{solo se implementan y validan empíricamente dos líneas} en esta memoria:
\begin{itemize}
\item \textbf{Línea 1 (implementada):} Validación del pipeline clásico DM-Time/CenterNet mediante adaptación paramétrica
\item \textbf{Línea 2 (implementada):} Estrategia híbrida SNR-threshold + clasificación CNN como suplente del detector
\item \textbf{Líneas 3 y 4 (propuestas, NO implementadas):} Representaciones 2D alternativas (polarimetría, tiempo-ancho, tiempo-RM) y estrategias avanzadas de Zhang (DM-expand, fishing DM$\approx$0) quedan expresadas arquitectónicamente pero \textit{sin validación empírica}, constituyendo trabajo futuro para nivel de posgrado.
\end{itemize}

\textbf{No se contempla:} reentrenamiento de modelos, \emph{backends} instrumentales, almacenamiento distribuido, ni integración con sistemas externos de alertas (aunque se proveen salidas necesarias).

\medskip

\noindent\textbf{Resultados Esperados}

\begin{itemize}
\item \textit{Pipeline} dockerizado end-to-end ejecutable por CLI con documentación 
\item Estrategia validada para detección en régimen milimétrico, incluida la validación con polarización cuando esté disponible. 
\item Código abierto para la comunidad astronómica
\end{itemize}


\newpage