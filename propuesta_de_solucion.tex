\secnumbersection{PROPUESTA DE SOLUCIÓN}

% Se debe desarrollar la solución propuesta. Los subcapítulos por poner aquí son propios del autor. Se sugiere mencionar metodología usada. Es conveniente incorporar figuras y tablas para aclarar la solución, que deben indicar el número de la figura, su nombre y su autor o fuente (si las diseñas tú, la fuente es ``Elaboración propia''). Ver ejemplos en esta página y en la siguiente.

% Cabe mencionar que aquí está la esencia del trabajo en lo que se refiere al aporte creativo del memorista, es el momento de demostrar que usted es un destacado profesional que creó, diseñó y/o llevó a cabo la solución propuesta.

Esta propuesta presenta la evolución de \textbf{DRAFTS} \cite{zhang2024drafts} desde un prototipo de investigación hacia un \textit{pipeline} \textbf{productivo, robusto y eficiente} diseñado para operaciones observacionales a gran escala. La transformación arquitectónica no se limita a mejoras incrementales, sino que establece una \textbf{extensión fundamental para regímenes de alta frecuencia} que expande significativamente las capacidades de detección en el espectro milimétrico.

\subsection{Vista General y Mapa de las 4 Contribuciones}

Esta propuesta se organiza en cuatro componentes articulados: (1) \textbf{DRAFTS++} (productización y robustecimiento del pipeline), (2) \textbf{Extensión a alta frecuencia (HF)} mediante sustitución de la detección por una etapa SNR determinística y validación DM-aware, (3) \textbf{Ruta alternativa sugerida por el autor de DRAFTS} (DM-range expand y fishing en DM $\approx 0$ con chequeos estrictos), y (4) \textbf{Análisis exploratorios de nuevos productos diagnósticos} (p.ej., TWL) para una futura red de detección en mm-wave. Las Figuras \ref{fig:pipeline-end-to-end} y \ref{fig:arquitectura-unificada} muestran el flujo E2E y la arquitectura unificada por \emph{Strategy}. \textit{Fuente de las figuras: Elaboración propia}.

\subsection{DRAFTS++: Pipeline Productivo, Robusto y Eficiente}

\subsubsection{Arquitectura Modular y CLI Unificada}

Se refactoriza a módulos claros (I/O, preprocesamiento, propuesta de candidatos, clasificación, validador, visualización) y una \textit{CLI} unificada con archivos de configuración validados (YAML/JSON). Se fijan semillas, versiones y \textit{hashes} de datos/modelos para trazabilidad.

La evolución de DRAFTS desde prototipo de investigación hacia sistema productivo requiere una transformación arquitectónica fundamental que aborde las limitaciones inherentes del código base original. El prototipo inicial presenta una estructura monolítica con scripts independientes (\texttt{d-center-main.py}, \texttt{d-resnet-main.py}) optimizados para condiciones de laboratorio controladas pero incapaces de manejar la variabilidad operacional de entornos observacionales reales.

\noindent\textbf{Refactorización arquitectónica sistemática.} La transformación implementa una arquitectura modular con 11 componentes especializados que establecen separación clara de responsabilidades y interfaces estandarizadas. Esta reestructuración elimina las dependencias circulares del prototipo original y establece un flujo de datos unidireccional que garantiza consistencia operacional y facilita mantenimiento y extensibilidad del sistema.

\subsubsection{Ingesta y Streaming por Chunks}

Soporte FITS/PSRFITS/Filterbank; \emph{chunking} con \emph{overlap} parametrizable, lectura perezosa y control de memoria/latencia. Artefactos intermedios: \textit{waterfalls}, mapas tiempo--DM (cuando aplica), parches tiempo--frecuencia.

El sistema procesa archivos FITS/PSRFITS mediante algoritmos de normalización por canal que compensan variaciones instrumentales y efectos de ganancia. Se implementa enmascarado adaptativo de interferencia de radiofrecuencia (RFI) basado en análisis estadístico de outliers temporales y espectrales. La arquitectura de \emph{chunking} con solapamiento controlado permite procesamiento eficiente de observaciones de larga duración mediante streaming de datos con gestión automática de memoria.

\subsubsection{Aceleración y Control de Recursos}

Vectorización, uso opcional de GPU, paralelización por bloques, \textit{caches} por canal y reducción de I/O. Métricas por etapa (tiempos, throughput, picos de memoria).

Vectorización de operaciones matemáticas críticas, integración opcional de procesamiento GPU para algoritmos intensivos en cómputo, y paralelización de bloques de procesamiento para maximizar la utilización de recursos computacionales disponibles.

\subsubsection{Registro, Auditoría y Salidas Estandarizadas}

\textit{Logging} estructurado, \textit{run-ids} reproducibles, reportes CSV/Parquet, figuras consistentes (tiempo--DM, dispersado/dedispersado, parches) y \textit{composites}.

Sistema de \emph{logging} estructurado que registra todas las operaciones críticas, implementación de semillas fijas para algoritmos estocásticos, generación de firmas criptográficas para datos y modelos, y producción de resúmenes de métricas por ejecución para garantizar auditoría completa.

\begin{table}[h] 
\centering 
  \caption{\label{tab:mejoras} Resumen de mejoras de ingeniería (DRAFTS++). \textit{Fuente: Elaboración propia}.}
\begin{tabular}{p{0.28\textwidth} p{0.32\textwidth} p{0.32\textwidth}} 
\toprule 
    \textbf{Aspecto} & \textbf{Antes (prototipo)} & \textbf{Después (DRAFTS++)} \\
\midrule 
Estructura & Scripts sueltos & Módulos + CLI + configs validadas \\ 
    Archivos grandes & Lectura monolítica & \emph{Chunking} + \emph{overlap} + control de memoria \\
    Detección & Sólo DL (DM--Time) & \emph{Strategy} intercambiable (DM--Time/TWL/SNR) \\
Auditoría & Log mínimo & Log estructurado + semillas + hashes \\ 
    Salidas & Figuras ad hoc & CSV/Parquet + plots estándar + composites \\
\bottomrule 
\end{tabular} 
\end{table}

\subsection{Extensión a Alta Frecuencia: Detección por SNR + Validación DM-aware}

\subsubsection{Criterio Físico para Activar el Modo HF}

Se reemplaza un corte fijo en GHz por el retardo dispersivo esperado frente a la resolución temporal:
\[
\Delta t_{\mathrm{ms}} = 4.148808 \times 10^{3}\ \mathrm{DM}\,(\nu_{\mathrm{low}}^{-2}-\nu_{\mathrm{high}}^{-2}) \, .
\]
Se activa HF si $\Delta t_{\mathrm{ms}} < \alpha\, t_{\mathrm{samp}}$ (p.ej., $\alpha\!=\!1.5$), indicando que el \textit{bow-tie} no sería resoluble.

La detección de FRBs en el régimen milimétrico (30--100 GHz) presenta desafíos fundamentales que requieren aproximaciones metodológicas diferenciadas. A estas frecuencias, la dispersión temporal se atenúa significativamente debido a la dependencia cuadrática inversa con la frecuencia, resultando en firmas dispersivas que pueden ser indistinguibles del ruido instrumental en resoluciones temporales típicas.

\subsubsection{Detección por SNR (Sustituto de la Red de Detección)}

\begin{enumerate}
  \item Perfil temporal $s(t)$ \(\rightarrow\) normalización robusta (mediana/MAD) \(\rightarrow\) \textbf{SNR}(t).
  \item Hallazgo de máximos locales $\ge T$ con separación mínima $\Delta t_{\min}$.
  \item Para cada pico $t_i$: generar \textbf{caja sintética} $[t_i-w,\, t_i+w]$ (sobre todo el ancho o por sub-bandas).
\end{enumerate}

\paragraph{Flujo detallado del Pipeline de Alta Frecuencia.} El proceso de detección híbrida sigue una secuencia específica que garantiza robustez y eficiencia:

\begin{enumerate}
\item \textbf{Evaluación de disponibilidad TWL:} El sistema verifica si \texttt{TWL\_HYBRID\_DETECTION} está habilitado y si los datos de polarización (Stokes Q, U) están disponibles.
\item \textbf{Generación de mapa t-W:} Si la detección híbrida está activa, se genera el mapa tiempo-ancho de ocupación de banda mediante \texttt{generate\_twl\_map\_for\_window}.
\item \textbf{Conversión a tensor RGB:} El mapa 2D se convierte a un tensor RGB de 512×512 píxeles usando \texttt{twl\_occupancy\_to\_detection\_tensor} para compatibilidad con CenterNet.
\item \textbf{Detección con CenterNet:} El tensor se procesa con la red de detección para identificar candidatos potenciales.
\item \textbf{Evaluación de candidatos:} Si se encuentran candidatos, se procede a clasificación binaria; si no, se activa el fallback SNR.
\item \textbf{Fallback SNR:} En caso de fallo de la detección híbrida o si está deshabilitada, se ejecuta \texttt{compute\_snr\_profile} y \texttt{\_find\_snr\_peaks} para detección tradicional.
\item \textbf{Clasificación final:} Todos los candidatos (híbridos o SNR) pasan por \texttt{classify\_patch} para determinar si son BURST o NO BURST.
\item \textbf{Guardado de resultados:} Los candidatos válidos se almacenan con métricas completas y se generan visualizaciones correspondientes.
\end{enumerate}

\begin{center}
\begin{tikzpicture}[
    node distance=1.2cm and 1.5cm,
    box/.style={rectangle, draw, fill=blue!20, text width=2.2cm, text centered, minimum height=0.7cm, font=\small},
    decision/.style={diamond, draw, fill=yellow!20, text width=1.8cm, text centered, minimum height=0.7cm, font=\small},
    process/.style={rectangle, draw, fill=green!20, text width=2.2cm, text centered, minimum height=0.7cm, font=\small},
    arrow/.style={-Stealth, thick}
]

% Nodos principales
\node[box] (A) {Pipeline Alta Frecuencia};
\node[decision, below=of A] (B) {TWL Híbrido?};
\node[process, below left=of B] (C) {Generar Mapa t-W};
\node[process, below=of C] (D) {Convertir a Tensor RGB};
\node[process, below=of D] (E) {Red de Detección};
\node[decision, below=of E] (F) {Candidatos?};
\node[process, below left=of F] (G) {Clasificación Binaria};
\node[process, below right=of B] (H) {Fallback SNR};
\node[process, below=of H] (I) {Detección SNR Tradicional};
\node[process, below=of G] (J) {Guardar Candidatos};

% Conexiones
\draw[arrow] (A) -- (B);
\draw[arrow] (B) -- node[left] {Sí} (C);
\draw[arrow] (C) -- (D);
\draw[arrow] (D) -- (E);
\draw[arrow] (E) -- (F);
\draw[arrow] (F) -- node[left] {Sí} (G);
\draw[arrow] (B) -- node[right] {No} (H);
\draw[arrow] (H) -- (I);
\draw[arrow] (I) -- (F);
\draw[arrow] (G) -- (J);

\end{tikzpicture}
\end{center}

\vspace{0.5cm}
\noindent\textbf{Figura \ref{fig:hf-pipeline}:} Flujo del Pipeline de Alta Frecuencia: detección híbrida TWL con fallback automático a SNR tradicional. El diagrama muestra la decisión que determina si usar detección híbrida (mapa t-W + CenterNet) o fallback a SNR tradicional. Fuente: Elaboración propia.

\label{fig:hf-pipeline}

\begin{figure}[h] 
\centering 
% \includegraphics[width=0.9\textwidth]{hf_mode_snr_boxes.pdf} 
\caption{\label{fig:hf} Modo HF: del perfil SNR a cajas sintéticas, dedispersión local y clasificación binaria. Fuente: Elaboración propia.} 
\end{figure}

\subsubsection{Dedispersión Local + Clasificación Binaria}

Con cada caja, construir el parche tiempo--frecuencia; dedispersar en rejilla local de DM; \textbf{clasificación binaria} (BURST/NO BURST); retener la mejor DM y SNR del parche dedispersado.

\paragraph{Pseudocódigo (detección SNR).}
\begin{algorithm}[h]
\caption{Detección SNR para Alta Frecuencia}
\begin{algorithmic}[1]
\Input{$s(t)$, umbral $T$, separación mínima $\Delta t_{min}$}
\Output{Lista de picos candidatos $\{t_i\}$}
\Function{DetectarPicosSNR}{$s$, $T$, $\Delta t_{min}$}
    \State $s_{norm} \leftarrow (s - median(s)) / MAD(s)$
    \State $picos \leftarrow maxima\_locales(s_{norm})$
    \State $candidatos \leftarrow [\ ]$
    \For{$t$ \textbf{in} $picos$}
        \If{$s_{norm}[t] \geq T$ \textbf{and} $dist\_minima(t, candidatos) \geq \Delta t_{min}$}
            \State $candidatos.append(t)$
        \EndIf
    \EndFor
    \State \textbf{return} $candidatos$
\EndFunction
\end{algorithmic}
\end{algorithm}

\subsubsection{Validación DM-aware y Consistencia}

Exigir DM$^\ast\!>\!0$ con incertidumbre acotada; verificar consistencia por sub-bandas y coherencia entre \emph{chunks} solapados; descartar candidatos que fallen estos chequeos.

\subsubsection{Productos Diagnósticos TWL-maps}

En el régimen de alta frecuencia, donde las firmas dispersivas tradicionales se atenúan significativamente, los mapas tiempo--ancho--polarización lineal (TWL) proporcionan información diagnóstica complementaria que puede compensar la pérdida de evidencia dispersiva. Estos productos especializados aprovechan la información de polarización disponible en observaciones con datos de Stokes completos.

\noindent\textbf{Análisis de polarización lineal.} Cuando se dispone de datos de Stokes $I,Q,U$ \cite{hamaker1996understanding}, se puede calcular la \textbf{polarización lineal} $L=\sqrt{Q^2+U^2}$, que cuantifica la intensidad de la componente polarizada de la señal electromagnética. Esta medida proporciona información adicional sobre la naturaleza física de las fuentes y puede revelar patrones que no son evidentes en el análisis de intensidad total únicamente.

\begin{figure}[h] 
\centering 
% \includraphics[width=0.9\textwidth]{twl_map.pdf} 
\caption{\label{fig:twl} Ejemplo de TWL-map (tiempo vs. ancho con intensidad de $L$). Fuente: Elaboración propia.} 
\end{figure}

\begin{table}[h]
  \centering
  \caption{\label{tab:param_hf} Parámetros del modo HF (valores iniciales). \textit{Fuente: Elaboración propia}.}
  \begin{tabular}{p{0.35\textwidth} p{0.55\textwidth}}
    \toprule
    \textbf{Parámetro} & \textbf{Descripción} \\
    \midrule
    $T$ (umbral SNR) & 5--7; ajustar por FDR y condiciones de ruido \\
    $\Delta t_{\min}$ & Múltiplo del ancho de pulso esperado \\
    Rejilla DM local & Centrada en 0; pasos gruesos y posterior refinamiento \\
    Sub-bandas & 2--4 particiones para coherencia \\
    Criterio DM & Requerir DM$^\ast>0$ (con error acotado) \\
    \bottomrule
  \end{tabular}
\end{table}

\subsection{Vía Alternativa del Autor de DRAFTS: DM-expand \& Fishing}

\subsubsection{DM-range Expand \& Step Coarse}

Ampliar el rango y el \textit{step} de DM hasta "abrir" el \textit{bow-tie}; una vez detectado, exigir centro con DM$>0$ significativamente mayor que cero.

La metodología propuesta integra dos estrategias complementarias sugeridas por Yong–Kun Zhang \cite{zhang2024drafts}: (i) \emph{Expansión del rango/step de DM} para maximizar la visibilidad de firmas dispersivas residuales y verificar estadísticamente que DM$>0$; (ii) \emph{Detección de candidatos cerca de DM$\approx 0$} mediante algoritmos de detección de picos adaptativos seguida de validación rigurosa que exige evidencia de DM$>0$ y consistencia entre sub-bandas/\emph{chunks}.

\subsubsection{Fishing en DM$\approx 0$ + Chequeos Estrictos}

\textit{Pescar} candidatos con clasificador o detector simple cerca de DM$\approx 0$; luego validar con DM$>0$, consistencia por sub-bandas y coherencia entre \emph{chunks}. Útil para eventos débiles sin \textit{bow-tie} claro.

\subsubsection{Integración con el Pipeline}

Ambas tácticas se exponen como \textbf{estrategias} de propuesta de candidatos (alternativas a SNR/TWL/DM--Time) y se someten al \textbf{mismo} validador DM-aware, clasificación y visualización.

\begin{table}[h] 
\centering 
\caption{\label{tab:estrategias} Estrategias de propuesta de candidatos y criterios de uso. Fuente: Elaboración propia.} 
\begin{tabular}{p{0.24\textwidth} p{0.38\textwidth} p{0.30\textwidth}} 
\toprule 
\textbf{Estrategia} & \textbf{Cuándo usarla} & \textbf{Pros / Contras} \\ 
\midrule 
CenterNet en DM--Time & $\Delta t \gg t_{\rm samp}$ (bow-tie resoluble) & + Precisa en L/S-band; -- Pierde poder en mm-wave \\ 
TWL-Híbrido + CenterNet & Stokes disponibles en mm-wave & + Usa polarización/anchos; -- Coste extra de features \\ 
SNR-only + Clasificador & mm-wave sin bow-tie claro & + O(N), simple; -- Más FP sin validación DM-aware \\ 
DM$\approx 0$ fishing + validar DM$>0$ & Para "pescar" candidatos débiles & + Sensible; -- Requiere validaciones estrictas \\ 
\bottomrule 
\end{tabular} 
\end{table}

\subsection{Análisis Exploratorios: Plots Característicos para Futura Red}

\subsubsection{TWL-maps (Tiempo-Ancho-Polarización Lineal)}

Cuando existan Stokes $I,Q,U$, se genera $L=\sqrt{Q^2+U^2}$ y se construyen mapas tiempo--ancho--$L$; vistas agregadas (máximo/mediana por ancho) sirven como evidencia complementaria en mm-wave.

El análisis TWL explora un espacio tridimensional \textbf{tiempo--ancho--$L$}, donde cada coordenada representa una combinación específica de tiempo de llegada, ancho de pulso temporal, e intensidad de polarización lineal. Este volumen de información se procesa mediante algoritmos de compactación que generan vistas útiles (máximo/mediana por ancho) optimizadas para detección automática y análisis visual.

\subsubsection{Caracterización y Features}

Medidas de ocupación/coherencia por ancho, alineación temporal de picos, variación de $L$; exportables como tensores para un detector basado en \textit{learning}.

Estos mapas no solo complementan la evidencia reducida en análisis tiempo--DM a mm-wave, sino que establecen una nueva dimensión de detección basada en propiedades de polarización. Constituyen candidatos prometedores para futuras \emph{features} de detección mediante aprendizaje automático, proporcionando información física adicional que puede mejorar la discriminación entre señales genuinas y artefactos instrumentales.

\subsubsection{Plan para Entrenamiento Futuro}

Curación de parches etiquetados, \emph{data augmentation} (líneas casi verticales, bow-ties débiles), \emph{split} por fuente/banda y \emph{ablation} contra validador DM-aware.

\subsection{Arquitectura Unificada (Strategy + Validator + Visualizer)}

Se adopta un patrón \textbf{Strategy} para la \textbf{propuesta de candidatos} (intercambiable: DM--Time/CenterNet, TWL-híbrido, SNR, DM-expand, fishing en DM$\approx0$), con un \textbf{validador DM-aware} común y un \textbf{visualizador desacoplado}. La función de proceso se unifica como \texttt{process\_slice(..., strategy)}.

\begin{figure}[h] 
\centering 
% \includegraphics[width=0.96\textwidth]{arquitectura_unificada.pdf} 
\caption{\label{fig:arquitectura-unificada} Arquitectura unificada: Strategy (propuesta de candidatos) + Validador DM-aware + Visualizador. \textit{Fuente: Elaboración propia}.} 
\end{figure}

\paragraph{Bucle unificado (pseudocódigo).} 
\begin{algorithm}[h]
\caption{Pipeline Unificado con Strategy Pattern}
\begin{algorithmic}[1]
\Input{$config$, $obs\_params$, $data\_files$}
\Output{Candidatos validados y métricas}
\Function{RunPipeline}{$config$, $obs\_params$}
    \State $strategy \leftarrow select\_strategy(config, obs\_params)$
    \For{$chunk$ \textbf{in} $stream\_data(\ldots)$}
        \State $cands \leftarrow strategy.propose(chunk, meta)$
        \For{$c$ \textbf{in} $cands$}
            \State $patch \leftarrow extract\_patch\_and\_dedisperse(chunk, c, local\_dm\_grid)$
            \State $y \leftarrow classifier.predict(patch)$
            \If{$y.is\_burst$ \textbf{and} $dm\_validate(patch)$ \textbf{and} $subband\_consistency(patch)$}
                \State $save\_candidate(patch, y, metrics)$
            \EndIf
        \EndFor
    \EndFor
    \State $visualize\_and\_summarize(run\_id)$
\EndFunction
\end{algorithmic}
\end{algorithm}

\subsection{Diagrama End-to-End del Pipeline}

La Figura \ref{fig:pipeline-end-to-end} sintetiza el flujo desde \texttt{main.py} hasta los resultados, con ramas clásica (DM--Time/CenterNet) y de alta frecuencia (TWL-híbrido, SNR) y \textit{fallbacks}.

\begin{center}
\begin{tikzpicture}[
    node distance=1.0cm and 1.5cm,
    box/.style={rectangle, draw, fill=blue!20, text width=2.0cm, text centered, minimum height=0.6cm, font=\tiny},
    decision/.style={diamond, draw, fill=yellow!20, text width=1.6cm, text centered, minimum height=0.6cm, font=\tiny},
    process/.style={rectangle, draw, fill=green!20, text width=2.0cm, text centered, minimum height=0.6cm, font=\tiny},
    output/.style={rectangle, draw, fill=orange!20, text width=2.0cm, text centered, minimum height=0.6cm, font=\tiny},
    arrow/.style={-Stealth, thick}
]

% FLUJO SIMPLIFICADO
\node[box] (A) {main.py};
\node[box, below=of A] (B) {run\_pipeline};
\node[box, below=of B] (C) {Cargar Modelos};

% ENTRADA DE DATOS
\node[box, right=of B] (D) {find\_data\_files};
\node[decision, below=of D] (E) {Tipo Archivo?};
\node[box, below left=of E] (F) {fits\_handler};
\node[box, below right=of E] (G) {filterbank\_handler};

% PROCESAMIENTO
\node[box, below=of E] (H) {streaming\_orchestrator};
\node[box, below=of H] (I) {Procesar Chunk};
\node[decision, below=of I] (J) {Frecuencia ≥ 8GHz?};

% PIPELINES
\node[box, below left=of J] (K) {Pipeline Alta Frecuencia};
\node[box, below right=of J] (L) {Pipeline Clásico};

% DETECCIÓN
\node[decision, below=of K] (M) {TWL\_HYBRID?};
\node[box, below left=of M] (N) {TWL + CenterNet};
\node[box, below right=of M] (O) {SNR Tradicional};
\node[process, below=of L] (P) {CenterNet DM-Time};

% CLASIFICACIÓN
\node[process, below=of N] (Q) {classify\_patch};
\node[process, below=of O] (R) {classify\_patch};
\node[process, below=of P] (S) {classify\_patch};

% RESULTADOS
\node[output, below=of Q] (T) {Guardar Candidatos};
\node[output, below=of R] (U) {Guardar Candidatos};
\node[output, below=of S] (V) {Guardar Candidatos};

% VISUALIZACIÓN
\node[output, below=of T] (W) {Plots + CSV};
\node[output, below=of U] (X) {Plots + CSV};
\node[output, below=of V] (Y) {Plots + CSV};

% Conexiones principales
\draw[arrow] (A) -- (B);
\draw[arrow] (B) -- (C);
\draw[arrow] (B) -- (D);
\draw[arrow] (D) -- (E);
\draw[arrow] (E) -- node[left] {FITS} (F);
\draw[arrow] (E) -- node[right] {Filterbank} (G);
\draw[arrow] (F) -- (H);
\draw[arrow] (G) -- (H);
\draw[arrow] (H) -- (I);
\draw[arrow] (I) -- (J);
\draw[arrow] (J) -- node[left] {Sí} (K);
\draw[arrow] (J) -- node[right] {No} (L);

% Detección
\draw[arrow] (K) -- (M);
\draw[arrow] (M) -- node[left] {Sí} (N);
\draw[arrow] (M) -- node[right] {No} (O);
\draw[arrow] (L) -- (P);

% Clasificación
\draw[arrow] (N) -- (Q);
\draw[arrow] (O) -- (R);
\draw[arrow] (P) -- (S);

% Resultados
\draw[arrow] (Q) -- (T);
\draw[arrow] (R) -- (U);
\draw[arrow] (S) -- (V);

% Visualización
\draw[arrow] (T) -- (W);
\draw[arrow] (U) -- (X);
\draw[arrow] (V) -- (Y);

\end{tikzpicture}
\end{center}

\vspace{0.5cm}
\noindent\textbf{Figura \ref{fig:pipeline-end-to-end}:} Diagrama end-to-end (ingesta $\to$ \emph{streaming}/\emph{chunking} $\to$ estrategia de candidatos $\to$ clasificación/validación $\to$ visualización y salida). \textit{Fuente: Elaboración propia}.

\label{fig:pipeline-end-to-end}

\subsection{Evaluación y Métricas}

Conjuntos ya analizados (control de \emph{recall}) y datos mm-wave piloto. Métricas: \emph{recall}, \emph{precision}, tasa de FP, latencia por GB, throughput, coherencia sub-bandas y \emph{chunks}. \emph{Ablation}: sin validación DM-aware, sin coherencia por sub-bandas, sin TWL.

\medskip\noindent \textbf{Síntesis de contribuciones y perspectivas futuras.} 

Esta propuesta establece una transformación fundamental en la metodología de detección de FRBs, evolucionando desde un prototipo de investigación hacia un sistema productivo de clase observacional. Las contribuciones presentadas abordan tanto aspectos de ingeniería de software como innovaciones metodológicas específicas para el régimen milimétrico.

\textbf{DRAFTS++} establece los fundamentos arquitectónicos para operaciones productivas mediante una refactorización sistemática que implementa modularización especializada, procesamiento eficiente por chunks, trazabilidad completa y artefactos de salida estandarizados. Esta transformación garantiza escalabilidad computacional, reproducibilidad experimental y mantenibilidad del sistema en entornos observacionales reales.

La \textbf{extensión a alta frecuencia} extiende las capacidades de detección hacia regímenes de alta frecuencia mediante la implementación de estrategias de detección híbrida que combinan análisis SNR adaptativo con procesamiento de mapas TWL especializados. Esta extensión metodológica compensa las limitaciones físicas inherentes a la detección dispersiva en el espectro milimétrico y establece nuevas vías de investigación basadas en propiedades de polarización.

Las figuras \ref{fig:pipeline-end-to-end}, \ref{fig:hf-pipeline}, \ref{fig:hf}, \ref{fig:twl} y \ref{fig:arquitectura-unificada} documentan la evolución arquitectónica completa, desde la conceptualización inicial hasta la implementación productiva. Juntas, demuestran no solo la transformación técnica del sistema, sino la creación de una plataforma extensible que puede adaptarse a futuros desarrollos en radioastronomía de transientes.

Esta investigación establece las bases para la próxima generación de sistemas de detección de FRBs, combinando robustez operacional con innovación metodológica. Los desarrollos futuros incluirán la integración de técnicas de aprendizaje automático avanzadas, la extensión a otros regímenes espectrales, y la aplicación a programas observacionales de gran escala (Fuente: Elaboración propia).