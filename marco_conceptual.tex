\secnumbersection{MARCO CONCEPTUAL}

\subsection{Introducción}
Los \textbf{Fast Radio Bursts (FRBs)} son explosiones brevísimas de emisión de radio (duraciones del orden de milisegundos) de origen extragaláctico, cuya naturaleza física permanece en debate. Desde la detección del primer FRB en 2007 \cite{Lorimer2007}, se han descubierto \textbf{millares} de estos eventos gracias a diversos radiotelescopios y búsquedas sistemáticas.

En la actualidad (2025) se conocen \textit{más de 4000 FRBs} individuales \cite{veracasanova2025}, de los cuales un \textasciitilde{}5\% manifiesta emisión repetitiva. Los FRBs destacan por su alta luminosidad intrínseca y exhiben el fenómeno de \textbf{dispersión}: los pulsos llegan retrasados a frecuencias más bajas debido a su propagación por el plasma interestelar/intergaláctico, caracterizándose mediante una medida de dispersión (DM) \cite{Lorimer2007,CordesMcLaughlin2003}.

Además, las observaciones están fuertemente afectadas por la \textbf{interferencia de radiofrecuencia (RFI)} terrestre, lo que dificulta la identificación de señales astrofísicas verdaderas. En este informe revisamos los métodos tradicionales y modernos para detectar FRBs en grandes volúmenes de datos, incluyendo pipelines clásicos (ej. \textbf{PRESTO}, \textbf{Heimdall}) y enfoques recientes basados en \textbf{aprendizaje automático}.

También se discuten los avances en la detección de FRBs a \textbf{frecuencias altas} (>1 GHz) con instrumentos como FAST y ALMA, y los últimos desarrollos en mitigación de RFI, técnicas de de-dispersión, clasificación de candidatos y localización automática de eventos.
\subsection{Métodos tradicionales de detección de FRBs}
Los primeros pipelines de búsqueda de FRBs derivan de técnicas empleadas en búsqueda de púlsares y pulsos individuales. El proceso clásico se describió en detalle en Cordes \& McLaughlin (2003) e implementaciones como \textbf{PRESTO} \cite{2011ascl.soft07017R} y \textbf{Heimdall} \cite{Barsdell_2012}.

En términos generales, estos métodos siguen una serie de pasos determinísticos:
\begin{enumerate}
\item \textbf{Filtrado de RFI inicial:} Se aplican umbrales estadísticos para identificar y enmascarar datos contaminados por RFI antes del análisis \cite{2011ascl.soft07017R}. Por ejemplo, la rutina rfifind de PRESTO calcula la distribución de intensidades por canal y marca tiempos o frecuencias con valores anómalos.
\item \textbf{Búsqueda en DM:} Se define una rejilla de valores de \textbf{Dispersion Measure (DM)} que cubre el rango de interés (e.g. DM = 0--3000 pc/cm³). Para cada DM, se \textbf{de-dispersa} el dinámico espectro (corrigiendo el retardo en frecuencia asumiendo ese DM) y se \textbf{integra en frecuencia} para obtener una serie temporal de potencia filtrada \cite{CordesMcLaughlin2003}.
\item \textbf{Barrido de anchura (boxcar):} Cada serie temporal dedispersada se convolve con filtros de ventana (boxcar) de distintas longitudes para ser sensible a pulsos de diversas duraciones (desde unos cientos de $\mu$s hasta varios ms) \cite{CordesMcLaughlin2003}. Tras esto, se mide la \textbf{relación señal-ruido (S/N)} máxima de cada serie temporal.
\item \textbf{Detección por umbral:} Se consideran \textit{candidatos} aquellos eventos con S/N por encima de un umbral predefinido (típicamente S/N > 6--8). Cada candidato queda caracterizado por su tiempo de llegada, DM y S/N, y opcionalmente anchura estimada.
\end{enumerate}
Herramientas ampliamente utilizadas como \textbf{PRESTO} \cite{2011ascl.soft07017R} implementan esta secuencia de forma eficiente, incorporando optimizaciones para el cálculo de la dedispersión y filtros. \textbf{Heimdall}, por su parte, está diseñado para procesar volúmenes de datos muy grandes en tiempo real, usando algoritmos paralelizables (FFT rápidas, dedispersión en árbol, etc.) y GPUs para acelerar la búsqueda incoherente \cite{Barsdell_2012}.

Estas soluciones tradicionales han sido pilares en descubrimientos de FRBs en las últimas dos décadas. Por ejemplo, el radiotelescopio de Parkes empleó pipelines de este tipo para detectar los primeros FRBs aislados \cite{Lorimer2007,FRB121102}, y los surveys de pulsar/FRB del telescopio de Arecibo y del GBT también utilizaron PRESTO o algoritmos semejantes para identificar eventos altamente dispersos \cite{2014ApJ...790..101S,Masui2015}.
\textbf{Limitaciones y rendimiento:} A pesar de su eficacia comprobada, los métodos tradicionales presentan varias limitaciones importantes. Un inconveniente es su \textbf{dependencia de parámetros y preprocesamiento de RFI}: la etapa de filtrado de interferencia (p. ej., eliminación de canales con RFI persistente, o sustracción de una curva de base) nunca es perfecta, de modo que \textit{residuos de RFI} pueden sobrevivir y producir \textit{falsos positivos}.

Irónicamente, el propio proceso de filtrado puede introducir artefactos artificiales en los datos (ej. ``sombra'' en la dinámica espectral tras eliminar un brote impulsivo de RFI), que luego pueden confundirse con señales reales \cite{zhang2024drafts}.

Además, la técnica de barrido de DM y boxcars conlleva una alta \textbf{complejidad computacional} y redundancia: la misma señal astrofísica puede disparar múltiples candidatos en DMs adyacentes y distintas anchuras, generando un conjunto duplicado de detecciones para un único evento físico. Esto implica que pipelines clásicos producen un número enorme de \textit{candidatos redundantes}.

Por ejemplo, al procesar datos con PRESTO usando un umbral S/N bajo (para mayor sensibilidad), es común obtener millones de \textit{triggers} candidatos, de los cuales la vasta mayoría son espúreos (RFI o ruido) y sólo una fracción minúscula corresponden a verdaderos pulsos de origen celeste \cite{zhang2024drafts}.

En la práctica, este ``diluido'' de verdaderos positivos entre un mar de falsos obliga a realizar \textbf{filtrado manual} o con heurísticas \textit{ad hoc} para descartar interferencia (por ejemplo, descartando candidatos con DM\textasciitilde{}0 pc/cm³ que suelen indicar RFI terrestre, o agrupando eventos coincidentes en tiempo en múltiples DMs en un único candidato). Todo este proceso manual dificulta la escalabilidad.

Estudios de rendimiento han mostrado que la \textit{completitud} (o sensibilidad efectiva) de los métodos clásicos puede verse comprometida: para mantener baja la tasa de falsos, se fijan umbrales S/N relativamente altos que inevitablemente hacen perder eventos débiles. Por ejemplo, en un dataset de prueba, PRESTO con umbral S/N $\approx$7 recuperó \textasciitilde{}80\% de los pulsos simulados, mientras que bajar el umbral a \textasciitilde{}5 elevó la recuperación a \textasciitilde{}86\% pero al costo de incrementar drásticamente los falsos positivos \cite{zhang2024drafts}.

Además, el procesamiento es lento: analizando unos segundos de datos grabados a alta resolución puede tomar del orden de \textbf{minutos} por archivo con PRESTO/Heimdall dependiendo de la densidad de DM probados \cite{zhang2024drafts}.

En resumen, los pipelines tradicionales tienden a ser \textbf{ineficientes y propensos a falsos positivos}, por lo que pueden quedarse cortos frente a los enormes volúmenes de datos que producen los nuevos radiotelescopios (p. ej. el array CHIME genera petabytes al día). De hecho, se ha señalado que este paradigma difícilmente podrá manejar de forma óptima los \textit{tens of thousands} de futuros FRBs que se esperan descubrir \cite{zhang2024drafts}.

Esto ha motivado la exploración de enfoques más sofisticados, en particular basados en \textbf{aprendizaje automático}, para mejorar la velocidad, sensibilidad y autonomía de la detección de transientes rápidos.
\subsection{Métodos modernos basados en aprendizaje automático}
En años recientes, la comunidad ha adoptado técnicas de \textit{machine learning} y \textit{deep learning} para abordar las deficiencias antes mencionadas. Existen dos estrategias principales: (1) aplicar modelos de aprendizaje para \textbf{clasificar candidatos} generados por los métodos tradicionales (filtrado de falsos), y (2) desarrollar pipelines capaces de \textbf{detectar FRBs directamente} en los datos crudos o ligeramente procesados, reduciendo la dependencia en las etapas manuales de los algoritmos clásicos.
\textbf{Clasificación de candidatos (post-búsqueda):} Reconociendo que los buscadores clásicos producen una sobreabundancia de candidatos espurios, varios trabajos implementaron clasificadores automáticos (basados en \textit{machine learning supervisado}) para discernir entre \textit{verdaderos FRBs} y \textit{falsos positivos} entre los candidatos detectados.

Connor \& van Leeuwen (2018) introdujeron uno de los primeros enfoques de este tipo, entrenando una red neuronal convolucional para asignar a cada candidato una probabilidad de ser una señal astrofísica real. De modo similar, Agarwal \textit{et al}. (2020) desarrollaron el sistema \textbf{FETCH} (\textit{Fast Extragalactic Transient Candidate Hunter}), que consiste en una serie de modelos de deep learning (redes neuronales profundas) entrenados con ejemplos de FRBs simulados y datos reales de RFI.

FETCH logra \textbf{precisiones y exhaustividades mayores al 99.5\%} en la clasificación de candidatos en sus pruebas, filtrando casi completamente los falsos positivos generados por pipelines tradicionales \cite{Agarwal_2020}. Estos clasificadores suelen operar sobre \textit{imágenes} o \textit{plantillas} del candidato: por ejemplo, usan la imagen tiempo vs. frecuencia dedispersada (``waterfall'') y/o la serie de S/N vs. DM del evento como entrada para la red neuronal \cite{Agarwal_2020}.

Al aprender las características morfológicas típicas de un FRB real (p. ej. dispersión con cierta curva característica, ancho temporal acotado, emisión amplia en frecuencia, etc.) frente a artefactos de RFI (p. ej. señales confinadas en una frecuencia estrecha, o pulsos no dispersos), estos métodos han reducido sustancialmente la carga de inspección manual.

No obstante, es importante notar que este enfoque \textit{no soluciona} otros problemas de los algoritmos clásicos: la búsqueda sigue pudiendo ser incompleta (puede no detectar señales muy débiles o con formas inusuales), y sigue existiendo redundancia en la detección inicial. En otras palabras, la clasificación automática optimiza la \textit{etapa final de verificación}, pero no reinventa el proceso de búsqueda en sí.
\textbf{Detección directa con redes neuronales:} La segunda dirección innovadora ha sido intentar que redes neuronales profundas \textit{reemplacen parcial o totalmente las etapas tradicionales} de búsqueda de pulsos. En vez de generar candidatos mediante dedispersión exhaustiva y umbral de S/N, se entrena un modelo para que analice directamente los datos dinámicos (tiempo-frecuencia) y \textit{detecte la presencia de un FRB}.

Un ejemplo pionero fue el trabajo de Zhang \textit{et al}. (2018), quienes aplicaron una CNN (red convolucional) a las observaciones crudas de FRB121102 en banda C (4--8 GHz), encontrando decenas de pulsos adicionales que los algoritmos convencionales no habían marcado. En ese estudio, la red aprendió a identificar las características ``parabólicas'' que deja la dispersión de un FRB en la matriz tiempo-frecuencia, y logró mayor sensibilidad y menor tasa de falsos que un algoritmo de dedispersión de fuerza bruta en el mismo dataset \cite{Gajjar2018}.

Trabajos más recientes, como Liu \textit{et al}. (2022), han llevado esta idea a pipelines operativos: estos autores implementaron una red neuronal profunda entrenada con \textit{FRBs simulados} inmersos en ruido real, capaz de escanear datos de un radiotelescopio (26-m de Nanshan) \textbf{sin pasar por la dedispersión por rejilla}.

Su pipeline, denominado DDSS (\textit{Dispersed Dynamic Spectrum Search}), detectó exitosamente ráfagas del repetidor FRB 20201124A en banda L, mostrando una \textbf{precisión >99\%} en la identificación de pulsos individuales y destacando su sensibilidad para captar incluso señales débiles por encima del ruido \cite{Liu_2022}.

En principio, al eliminar la necesidad de probar multitud de DM y usar filtros de ancho, estos métodos de detección con \textit{deep learning} pueden ser mucho más \textbf{rápidos} y tratar directamente el problema como uno de reconocimiento de patrones en imágenes (2D), aprovechando los avances en \textit{computer vision}.

Sin embargo, también presentan desafíos: (i) las \textit{señales débiles y difusas} (con S/N bajo, o muy ensanchadas temporalmente) pueden pasar inadvertidas para una red entrenada principalmente con ejemplos más brillantes, pues en la imagen de entrada se ven muy poco contrastadas \cite{zhang2024drafts}.

(ii) La diversidad de curvaturas de dispersión: cada FRB tiene una DM distinta, lo que significa que la pendiente de su traza tiempo-frecuencia varía; una red que analice una ventana de datos limitada podría ``perder'' eventos cuyo arco de dispersión no quepa enteramente en dicha ventana o tenga geometría distinta a la aprendida.

(iii) Al detectar directamente en datos no dedispersados, típicamente estos métodos sólo entregan la \textit{ubicación temporal} del burst, pero \textit{no} estiman automáticamente su DM óptima (que es crucial para caracterizar el evento). En consecuencia, algunos investigadores han optado por enfoques híbridos donde la red neuronal se integra con cierta dedispersión previa o posterior.
Un enfoque muy exitoso dentro de esta categoría es el pipeline \textbf{DRAFTS} (``\textit{Deep Learning--based RAdio Fast Transient Search}'') propuesto por Zhang \textit{et al}. (2024). Este sistema combina dos etapas basadas en \textit{deep learning} para lograr tanto la detección como la confirmación de FRBs en tiempo casi real.

En DRAFTS, primero se aplica un modelo de \textbf{detección de objetos} en la imagen de datos dedispersados para varios DM, identificando regiones rectangulares donde podría haber una señal (similar a cómo se detectan rostros u objetos en una foto) -- en esencia, localiza el tiempo de llegada y DM de una posible ráfaga. Luego, cada recorte candidato es alimentado a un modelo \textbf{clasificador binario} que decide si corresponde a un FRB verdadero o a ruido/RFI.

Técnicamente, la etapa 1 utiliza una arquitectura \textbf{CenterNet} \cite{zhou2019objects}, una red neuronal \textit{anchor-free} que aprende a señalar el \textit{punto central} de cada objeto de interés en la imagen (aquí, el ``objeto'' es la firma dispersada de un FRB en el plano tiempo vs DM). Esta elección es ideal porque evita tener que predefinir un conjunto de formas/anchos de ventana; CenterNet simplemente produce un mapa de calor donde cada máximo indica el centro de un transiente y además regresa el tamaño (extensión en tiempo y en DM) del mismo \cite{zhang2024drafts}.

La etapa 2 emplea una red \textbf{ResNet} \cite{He_2015_ResNet} (una ResNet-18 entrenada específicamente) que toma la sub-imagen del candidato (p. ej. el dinámico espectro dedispersado alrededor de la señal) y devuelve la probabilidad de que sea real. Esta arquitectura de \textit{red residual} es muy potente en clasificación de imágenes, habiendo demostrado alta exactitud en diversos dominios, y en este contexto ayuda a filtrar falsos que superen la primera etapa.

En la práctica, DRAFTS realiza también la \textbf{dedispersión} de forma acelerada: utiliza GPU para computar de manera paralela las sumas dedispersadas en un rango amplio de DM en fragmentos de datos grandes \cite{zhang2024drafts}. Gracias a esto, el pipeline completo puede examinar datos voluminosos con rapidez.
\textit{Figura 1.} Esquema simplificado del flujo de procesamiento en DRAFTS++ (versión extendida de DRAFTS). El pipeline realiza la dedispersión de la señal en GPU y luego aplica un detector de objetos basado en CenterNet para localizar ``cajas'' de posibles FRBs en la matriz tiempo--DM dedispersada. Cada detección es recortada y evaluada por un clasificador ResNet para determinar si es un FRB real (paso True) o un falso positivo (paso False). Finalmente, los eventos confirmados se guardan y generan sus productos (estadísticas, gráficas). \textit{Fuente: adaptado de Zhang} et al\textit{., 2024.}
\textbf{Rendimiento de DRAFTS y comparación:} Los resultados reportados para este pipeline muestran mejoras notables frente a los métodos tradicionales. En pruebas con datos simulados con inyecciones de pulsos, la fase de detección con CenterNet logró una \textit{tasa de recuperación} (recall) de \textasciitilde{}96--97\% con una tasa de falsas detecciones muy baja (\textasciitilde{}20--23 falsos en total), superando ampliamente el recall de \textasciitilde{}85\% de PRESTO en condiciones equivalentes \cite{zhang2024drafts}.

Además, el procesamiento es más eficiente: analizar un segmento de \textasciitilde{}6 segundos de datos tomó \textbf{\textasciitilde{}4--5 segundos} con DRAFTS (en GPU con modelos ResNet-18), comparado a \textbf{\textasciitilde{}120 segundos} usando PRESTO sobre la misma ventana \cite{zhang2024drafts}. Esto sugiere que el enfoque de \textit{deep learning} no sólo mantiene o mejora la sensibilidad, sino que también acelera la búsqueda.

Un caso de estudio destacado es el de la fuente repetitiva \textbf{FRB 20190520B} (descubierta por FAST \cite{Niu2022_FRB20190520B}). Al reanalizar \textasciitilde{}5 meses de datos de FAST de esta fuente, DRAFTS detectó \textbf{258 ráfagas} en total, mientras que el pipeline tradicional Heimdall originalmente había encontrado sólo 75 eventos \cite{zhang2024drafts}.

Es decir, la red neuronal \textit{triplicó} el número de FRBs hallados, recuperando \textbf{\textasciitilde{}183 bursts adicionales} que pasaron inadvertidos en el análisis inicial con métodos clásicos. Esto elevó la tasa de evento máxima observada de \textasciitilde{}4.5 a \textasciitilde{}28.6 ráfagas por hora, modificando sustancialmente la interpretación de la actividad intrínseca de la fuente \cite{zhang2024drafts}.

Este experimento demuestra cómo los nuevos pipelines basados en aprendizaje profundo pueden mejorar la \textbf{completitud} de las búsquedas, descubriendo pulsos más débiles o eclipsados por RFI que antes se perdían. Actualmente, se están aplicando variantes de DRAFTS (y otros similares) a datos de distintos telescopios para validar su desempeño en condiciones diversas.

Cabe mencionar que otras aproximaciones \textit{anchor-based} (e.g. usando algoritmos de la familia YOLO o RCNN) también podrían aplicarse a la detección de FRBs, pero DRAFTS optó por CenterNet por su robustez en detección de objetos pequeños y numerosos en imágenes \cite{zhou2019objects}.

En general, la comunidad ve con optimismo la incorporación de \textbf{redes neuronales entrenadas} en la tubería de análisis: no sólo para detección/clasificación, sino también para tareas como sustracción de RFI (ver siguiente sección) y incluso estimación de parámetros físicos.
En resumen, las técnicas modernas basadas en \textit{deep learning} están logrando reducir drásticamente los falsos positivos, automatizar la selección de eventos astrofísicos y aumentar la sensibilidad de detección. La combinación de detectores de patrones 2D y clasificadores de alta precisión permite aspirar a pipelines de FRB \textbf{totalmente autónomos}, capaces de procesar en tiempo real los flujos masivos de datos de instrumentos nuevos (como CHIME, ASKAP, MeerKAT, FAST) y de activar alertas inmediatas para seguimiento multi-frecuencia.

No obstante, es importante continuar evaluando estos sistemas en distintos entornos observacionales, asegurando que generalicen bien y que se comprendan sus sesgos (por ejemplo, evitar que una red \textit{sobre-entrenada} descarte FRBs reales inusuales por no parecerse a los del conjunto de entrenamiento).

A medida que crece el volumen de datos y la necesidad de rapidez (p. ej. para capturar contrapartes ópticas o de rayos X de un FRB en tiempo real), estos avances en pipelines inteligentes serán cruciales.

\subsubsection{Revisión sistemática de pipelines modernos de detección de FRBs}

Los Fast Radio Bursts (FRBs) son impulsos de radio de duración milimétrica con origen extragaláctico, capaces de sondear el contenido bariónico del universo mediante su medida de dispersión (DM). La detección oportuna y localización precisa exigen software de muy baja latencia y alta eficiencia, capaz de operar en tiempo real con grandes volúmenes de datos y entornos con fuerte interferencia de radiofrecuencia (RFI). Este informe sintetiza y reescribe, en español, el contenido central del artículo de revisión de Rajwade \& van Leeuwen (2024), manteniendo su estructura y énfasis técnicos \cite{Rajwade2024}.

Como motivación científica, las ráfagas localizadas permiten estimar el contenido de bariones a partir de la DM integrada a lo largo de la línea de visión, aportando evidencia independiente sobre los bariones ``faltantes'' en el medio intergaláctico \cite{Macquart2020}.

\paragraph{Pasos clave de un pipeline de búsqueda de FRBs}

De extremo a extremo, los pipelines modernos implementan una secuencia que incluye: mitigación de RFI, dedispersión, filtrado por coincidencia (matched filtering), selección y clasificación de candidatos, y, cuando corresponde, buffering de voltajes, disparo de alertas y seguimiento multi-longitud de onda. A continuación se describen los componentes y técnicas tal como se sistematizan en la revisión base.

\textbf{Mitigación de RFI:} La RFI se clasifica habitualmente en banda ancha y banda angosta. Para reducir su impacto se usan enfoques de complejidad creciente: (i) máscaras estáticas de canales persistentemente contaminados; (ii) filtros de DM=0 (zero-DM) que sustraen la contribución de RFI de banda ancha; (iii) umbralizado adaptativo en tiempo real, con estadísticos locales por bloques; (iv) IQRM, que detecta outliers por canal usando métricas robustas y radio/umbral configurables; (v) Z-dot, que evita la sobre-sustracción canal a canal; (vi) curtosis espectral, para identificar no gaussianidades; y (vii) mitigación secundaria fuera de línea para refinar datos almacenados \cite{Rajwade2024,Morello2022}.

\textbf{Dedispersión:} El objetivo es corregir el retardo dependiente de la frecuencia introducido por el plasma interestelar/intergaláctico. Los enfoques incluyen: (a) fuerza bruta (sumar a lo largo de la curva cuadrática para cada DM, optimizando con sub-bandeo, downsampling y reutilización de datos); (b) algoritmos tipo árbol (tree dedispersion) que reducen complejidad mediante divide-and-conquer; (c) FDMT, que combina baja complejidad con seguimiento exacto de la curva de dispersión; y (d) dedispersión en el dominio de Fourier (FDD), donde los retardos se aplican como rotaciones de fase y la inversión se realiza eficientemente \cite{Rajwade2024,Zackay2017,Bassa2022}.

Para mejorar sensibilidad sin costos prohibitivos, es habitual un esquema semi-coherente: se aplica dedispersión coherente en una rejilla gruesa de DMs y luego una búsqueda fina incoherente alrededor de cada punto grueso \cite{Rajwade2024}.

\textbf{Matched Filtering:} Sobre cada serie de tiempo dedispersada se aplica filtrado por coincidencia con familias de boxcars (o aproximaciones gaussianas) de anchos geométricamente espaciados. Se maximizan las detecciones por S/N respetando límites de latencia con downsampling para anchos mayores. Los pipelines típicos exploran de \textasciitilde{}0.1 a \textasciitilde{}100 ms, con ampliaciones recientes hacia escalas más anchas o más angostas según la evidencia observacional \cite{Rajwade2024}.

\textbf{Selección y clasificación de candidatos:} El número de candidatos puede ser masivo incluso tras una buena mitigación de RFI. Se realizan `sifting \& clustering' por cercanía en tiempo--DM--ancho para agrupar múltiples detecciones del mismo evento y se aplican cortes por DM y ancho consistentes con señales astrofísicas. Luego, modelos de aprendizaje profundo ---p. ej., clasificadores CNN como FETCH--- reducen falsos positivos y priorizan candidatos, disminuyendo la inspección manual en entornos con alta RFI \cite{Rajwade2024,Agarwal_2020}.

\textbf{Buffering, triggering y alertas:} Para maximizar el retorno científico, muchos sistemas almacenan de forma circular datos de voltaje de alta resolución. Una detección robusta dispara el volcado de esa ventana al disco y el envío de alertas estandarizadas (VOEvent) a la comunidad para seguimiento rápido y localización interferométrica sub-arco-segundo \cite{Rajwade2024,Petroff2017}.

\paragraph{Búsqueda en la era SKAO/ngVLA: ampliar el espacio de parámetros y acelerar algoritmos}

La próxima generación de instrumentos (p. ej., SKA, ngVLA, DSA-2000, CHORD) incrementará exponencialmente los datos y exigirá algoritmos más rápidos y generalistas. Se discuten extensiones hacia ráfagas ultra-estrechas y ultra-anchas (mitigando sesgos de selección), así como transformadas de Radon/Hough aplicadas al plano transformado (t, 1/$\nu^2$) que convierten la curvatura de dispersión en líneas y facilitan la detección robusta a RFI. Se esbozan, además, ideas de cómputo cuántico para operaciones intrínsecamente paralelas (direcciones, DMs, anchos) con colapsos binarios (detección/no detección) \cite{Rajwade2024}.

\paragraph{Conclusiones sobre pipelines modernos}

Los pipelines modernos para FRBs combinan mitigación de RFI, dedispersión optimizada, matched filtering y clasificación automática para operar en tiempo real con alta sensibilidad y bajo falso positivo. La revisión de Rajwade \& van Leeuwen (2024) sintetiza buenas prácticas y rutas futuras: ampliar el espacio de búsqueda, adoptar algoritmos más eficientes (FDMT/FDD) y estandarizar disparos de seguimiento (VOEvent). A la par, los resultados cosmológicos ---como el censo bariónico con FRBs localizados--- refuerzan el valor de estos sistemas en la astrofísica de precisión \cite{Rajwade2024,Macquart2020}.
\subsection{Detecciones de FRBs a frecuencias altas (> 1 GHz)}
Históricamente, la mayoría de FRBs se han descubierto en \textbf{frecuencias relativamente bajas}, entre \textasciitilde{}400 MHz y 1.4 GHz, debido a que muchos radiotelescopios (Parkes, Arecibo, CHIME, UTMOST, ASKAP, etc.) operan en esas bandas y a que la emisión de FRBs típicamente presenta espectros decrecientes con la frecuencia (análogos a los púlsares, que suelen tener índices espectrales negativos).

Sin embargo, en años recientes ha habido un creciente interés por explorar \textbf{frecuencias más altas} --de varios GHz hasta el rango milimétrico-- por varias razones: (1) a frecuencias altas la \textbf{dispersión y el scattering} (dispersión multi-caminos) son menores, lo que podría permitir detectar pulsos que a bajas frecuencias quedan ensanchados o borrados; (2) la posibilidad de que ciertas fuentes de FRB emitan también a frecuencias milimétricas (lo que ofrecería nueva información sobre los procesos emisores y los entornos extremos donde se producen).
Las primeras incursiones exitosas en alta frecuencia provinieron de observar repetidores conocidos con receptores de microondas. Un hito fue logrado con la fuente repetitiva \textbf{FRB 121102}: en 2017, se le apuntó con el telescopio de Green Bank (GBT) usando un receptor de banda C (4--8 GHz). Sorprendentemente, se detectaron decenas de pulsos hasta \textbf{8 GHz}, demostrando que este FRB emite coherentemente hasta al menos esa frecuencia \cite{Gajjar2018}.

De hecho, \textit{se registraron por primera vez FRBs por encima de 5 GHz} en aquella campaña, con 72 nuevos pulsos hallados gracias a un análisis con aprendizaje automático (véase Zhang \textit{et al}., 2018) sumados a 21 que ya se habían reportado con métodos tradicionales en esos mismos datos. Este experimento elevó el total a 93 bursts en \textasciitilde{}5 horas y estableció el récord de frecuencia máxima para FRBs repetidores.

Los pulsos de FRB 121102 a 4--8 GHz mostraron estructuras espectrales ricas y altamente polarizadas \cite{Gajjar2018,Michilli2018}, y su detección implicó que, al menos para algunos FRBs, la emisión se extiende a banda C e incluso más allá.

Posteriormente, otros repetidores han sido observados en bandas de \textasciitilde{}2 GHz con resultados variados: por ejemplo, FRB 20200120E (en la galaxia M81) fue monitoreado con la red europea EVN hasta \textasciitilde{}1.7 GHz, y FRB 20180916B (repetidor con período de 16 días) se detectó hasta \textasciitilde{}1.5 GHz y a \textasciitilde{}110 MHz por LOFAR en el extremo bajo, pero no hay todavía reportes de detección por encima de 2 GHz para estos casos.
El telescopio esférico chino \textbf{FAST}, con su enorme colector de 500 m, ha jugado un papel crucial en la búsqueda de FRBs en banda L (1.05--1.45 GHz) y potencialmente en banda S (\textasciitilde{}2--3 GHz). FAST descubrió en 2019 el repetidor \textbf{FRB 20190520B} mencionado antes --notable por estar asociado a una fuente de radio persistente y tener un DM muy alta--, a \textasciitilde{}1.25 GHz \cite{Niu2022_FRB20190520B}.

FAST posteriormente llevó a cabo extensos seguimientos de repetidores: en particular, monitoreó \textbf{FRB 121102} en 2019, detectando la cifra extraordinaria de \textbf{1652 bursts en 47 días} (0.4--1.5 GHz), el mayor conjunto de ráfagas de un mismo FRB hasta la fecha \cite{Li_2021}.

También en 2020, FAST captó \textbf{1863 bursts de FRB 20201124A} en \textasciitilde{}1.3 GHz durante dos meses \cite{Xu_2022}, revelando que este objeto exhibe uno de los niveles de actividad más altos conocidos.

Estas observaciones demuestran que en banda L--S se concentran la mayoría de FRBs detectados gracias a la alta sensibilidad de FAST; por ahora FAST no ha reportado detecciones en su receptor de banda S (instalado para \textasciitilde{}2--3 GHz), pero hay interés en explorar esa ventana con su capacidad.

En general, la evidencia apunta a que muchas fuentes repetidoras mantienen emisión significativa hasta \textasciitilde{}2 GHz, aunque suelen mostrar \textbf{espectros relativamente estrechos} o evolutivos (algunos bursts ``aparecen'' solo en sub-bandas dentro de 1--2 GHz, indicando fenómenos de absorción o bandas preferenciales).
En las frecuencias de microondas \textbf{4--8 GHz}, además de FRB121102, se han registrado algunas detecciones de FRBs aislados por radiotelescopios capaces de hacer seguimiento multi-banda. Por ejemplo, el Deep Space Network de NASA reportó un FRB (FRB 180301) a \textasciitilde{}2.3 GHz con la antena de 70 m de Goldstone en 2018, aunque este evento luego se confirmó como un repetidor observado también por FAST \cite{Luo_2023}. Otro caso: el radiotelescopio de Effelsberg (Alemania, 100 m) participó en campañas multi-frecuencia de FRB 121102 e incluso obtuvo una detección a 4.85 GHz de un burst débil durante un esfuerzo simultáneo con óptico/rayos X \cite{cruces2020frb121102}. Estas detecciones esporádicas indican que \textbf{sí existen FRBs a unos pocos GHz}, pero son menos comunes o menos fácilmente captados, quizá porque la mayoría de FRBs conocidos fueron descubiertos a <2 GHz y su espectro puede decaer hacia frecuencias mayores.
Un territorio prácticamente inexplorado es el de las \textbf{frecuencias milimétricas} (>20--30 GHz). La gran pregunta es si FRBs producen emisión detectable en, por ejemplo, bandas de 30--100 GHz. La motivación es que la dispersión y el scattering son muchísimo menores a mm-ondas, lo que permitiría ver pulsos ultra-cortos sin ensanchamiento --pero por otro lado, la potencia de emisión puede ser muy baja a estas frecuencias si el espectro sigue bajando.

Para investigar esto, se han empezado a usar instrumentos como el \textbf{Atacama Large Millimeter/submillimeter Array (ALMA)} en modo fasado. ALMA, al operar en conjunto las 66 antenas como un solo telescopio de 85 m de diámetro equivalente, ofrece una \textit{sensibilidad inédita en el rango 30--100 GHz} junto con alta resolución temporal (hasta 100 $\mu$s con su sistema de pulsar).

En 2017 ALMA (Band-3, \textasciitilde{}95 GHz) fue utilizado en un experimento piloto apuntando al magnetar del Centro Galáctico (PSR J1745--2900) para verificar detección de pulsos en mm-onda \cite{veracasanova2025}. Ese estudio tuvo éxito: \textbf{ALMA detectó pulsos individuales} altamente polarizados de este magnetar a \textasciitilde{}3 mm de longitud de onda, demostrando que \textit{al menos pulsos de magnetar galáctico} pueden brillar en frecuencias tan altas.

Extrapolando estos resultados a FRBs (que se hipotetiza podrían ser magnetars extragalácticos), los autores estiman que ALMA, observando en sus bandas más bajas (35--50 GHz, 75--90 GHz) durante las fases de alta actividad de un repetidor cercano, \textbf{podría llegar a detectar algunos bursts por hora} \cite{veracasanova2025}.

Hasta la fecha, \textbf{no se ha confirmado ninguna detección de FRB en bandas milimétricas}, pero estos cálculos sugieren que es posible si se eligen bien los objetivos (por ejemplo, FRB 121102 durante un estallido de actividad, o FRB 20201124A que tuvo >1000 bursts en 2021).

ALMA y otros telescopios milimétricos (como el IRAM 30m, o incluso experimentos de intemperie como NOEMA) están comenzando a realizar campañas ``piggyback'' en coordinación con detecciones de radio convencionales, con la esperanza de atrapar algún burst simultáneo.

Cabe resaltar que las observaciones a frecuencias tan altas podrían eludir los efectos de \textbf{dispersión y scattering intrínsecos} de entornos densos: muchos FRBs repetidores muestran ensanchamiento por scattering a 1.4 GHz (p. ej. FRB 20121102A tiene cola de scattering de varios ms a L-band). En mm-onda, ese scattering sería despreciable, permitiendo medir la estructura temporal intrínseca del pulso, siempre y cuando la señal no decaiga por debajo de umbral de detección.
En conclusión, el ``horizonte de frecuencias'' de los FRBs se ha ido expandiendo gradualmente. Hoy sabemos que:

- La emisión de FRBs abarca \textbf{desde \textasciitilde{}100 MHz (LOFAR) hasta \textasciitilde{}8 GHz comprobados} en casos repetidores, aunque muchos FRBs no se detectan en toda esa banda debido a efectos de filtro (absorsión, auto-absorción, plasma lensing, etc.).

- \textbf{Arriba de 1 GHz}, particularmente en la ventana 1--2 GHz, se han descubierto numerosos FRBs gracias a FAST, DSN y otros, consolidando que es un régimen fértil en detecciones. FAST en 1.25 GHz ha sido especialmente productivo, revelando miles de pulsos de varios repetidores \cite{Li_2021,Xu_2022}.

- En \textbf{4--8 GHz}, sólo unos pocos repetidores han mostrado ráfagas (notablemente FRB121102) \cite{Gajjar2018}, pero esto podría deberse a que pocos telescopios han buscado extensamente en esa banda --una excepción es el experimento de vigilancia continua Breakthrough Listen en GBT.

- En \textbf{frecuencias milimétricas}, aún no hay FRBs detectados, pero los esfuerzos de instrumentos como ALMA abren una posibilidad real de conseguir las primeras detecciones en los próximos años, lo que sería revolucionario (confirmaría que la emisión de FRBs puede extenderse a decenas de GHz y permitiría estudiar sus pulsos sin efectos dispersivos) \cite{veracasanova2025}.
\subsection{Mitigación de RFI, de-dispersión y estrategias de clasificación/localización automática}
Además de los algoritmos de detección en sí, el éxito de un pipeline de FRBs depende de varias técnicas transversales: \textbf{cómo se maneja la RFI}, cómo se optimiza la \textbf{dedispersión}, cómo se \textbf{clasifican} automáticamente los candidatos y, en sistemas de múltiples telescopios, cómo se \textbf{localizan} las fuentes en el cielo. A continuación revisamos avances recientes en estos aspectos:
\begin{enumerate}
\item \textbf{Mitigación de RFI:} La RFI es uno de los mayores obstáculos en radioastronomía, pues señales humanas (telecomunicaciones, satélites, radares, etc.) pueden imitar o enmascarar pulsos astronómicos. Los métodos tradicionales usan técnicas estadístico-determinísticas: por ejemplo, \textbf{SumThreshold} y \textbf{Median filtering} son algoritmos implementados en software como \textit{PRESTO} y \textit{SIGPROC} que analizan la estadística de intensidad en cada canal de frecuencia y en ventanas de tiempo para identificar outliers (picos inusuales) y crear máscaras de datos malos.

También se emplean heurísticas como el \textbf{filtro de DM = 0} (descartar eventos que aparecen simultáneamente sin dispersar en todos los canales, ya que suelen ser interferencia local) \cite{Eatough2009}. Sin embargo, estos métodos pueden suprimir excesivamente datos válidos o dejar pasar RFI atípica.

En años recientes se han introducido herramientas de \textit{machine learning} para RFI: Akeret \textit{et al}. (2017) mostraron que una red neuronal del tipo \textbf{U-Net} (muy utilizada en segmentación de imágenes) puede ser entrenada para distinguir y eliminar trazas de RFI en dinámicos espectros. Su algoritmo de RFI basado en \textit{deep learning} alcanzó desempeños comparables a los mejores métodos clásicos pero con la promesa de adaptarse a RFI de morfologías más complejas [Akeret \textit{et al}., 2017].

Otros trabajos han empleado \textit{autoencoders} o \textit{random forests} para etiquetar segmentos de datos afectados por interferencia impulsiva vs. limpia (por ejemplo, la ``RFI-Net'' desarrollada para FAST aplicando arquitectura de Akeret \textit{et al}. 2017 con ligeras modificaciones, reduciendo el false-alarm de RFI persistente en ese telescopio).

En la práctica, muchos pipelines modernos integran ya capas de mitigación más inteligentes: DRAFTS, por ejemplo, entrena su detector con datos reales donde la RFI común (p. ej. señales fijas a ciertas frecuencias) está presente, de forma que la red aprende a \textbf{ignorar la RFI estable} (como las emisiones continuas en 1200--1300 MHz típicas en FAST) al no asociarlas a la forma dispersada de un FRB \cite{zhang2024drafts}.

Adicionalmente, se han propuesto esquemas \textit{on-the-fly} en radiotelescopios de formación de haz digital (beamforming) para comparar múltiples haces y descartar señales presentes en todos (indicador de RFI local) -- estrategia usada en arrays como CHIME/FRB.

En resumen, la tendencia actual es combinar enfoques: aplicar filtros rápidos de RFI obvios (ej. canales saturados) y luego utilizar modelos de ML para casos sutiles, logrando minimizar la pérdida de sensibilidad a la vez que se contienen los falsos disparos.
\item \textbf{Optimización de la de-dispersión:} La corrección de dispersión es fundamental pero computacionalmente costosa si se hace de forma ingenua para cientos-miles de DMs. Se han desarrollado algoritmos más eficientes como la \textbf{dedispersión en árbol} (tree dedispersion) que reutiliza cálculos entre DMs cercanos, reduciendo la complejidad \cite{Taylor2014}.

También se han implementado librerías optimizadas en GPU, como \textit{DEDISP}, para acelerar la suma dedispersada en paralelo. Barsdell \textit{et al}. (2012) demostraron que usando GPUs se podía obtener aceleraciones de orden 10--100× en la dedispersión incoherente, permitiendo búsquedas en tiempo real en muchos casos.

Hoy en día, prácticamente todos los pipelines en telescopios de gran volumen utilizan GPU o hardware específico (FPGA) para esta tarea. Por ejemplo, Heimdall fue reescrito para GPU, y el backend FRB de MeerKAT emplea FPGA+GPU para dedispersar \textasciitilde{}1000 DMs en vivo.

Otra técnica reciente es la \textbf{dedispersión dinámica}: si se conoce \textit{a priori} que se busca un repetidor con cierta DM conocida, se puede evitar un grid amplio y enfocar en rangos estrechos o incluso aplicar dedispersión coherente (si se cuenta con datos de voltaje baseband). Esto se ha hecho en seguimiento de FRB 121102, donde tras obtener su DM precisa (\textasciitilde{}560 pc/cm³), las búsquedas posteriores usaban esa DM fija mejorando S/N.

En general, la de-dispersión ya no es el cuello de botella computacional gracias a estos avances, y en pipelines como DRAFTS++ es simplemente un módulo rápido en GPU que prepara los datos para la red neuronal \cite{zhang2024drafts}.

Cabe mencionar que la \textbf{dedispersión coherente} (corrección exacta de dispersión con la transformada inversa de Fourier en baseband) se reserva para pulsos extremadamente brillantes o para análisis post-detección, ya que requiere grabar enormes volúmenes de datos de voltaje. No es factible hacerlo en búsquedas ciegas rutinarias (excepto en esfuerzos muy específicos con interferómetros como EHT o con triggers externos). Por tanto, la dedispersión incoherente optimizada sigue siendo el estándar.
\item \textbf{Clasificación automática de candidatos:} Ya discutido en secciones anteriores, pero reforzamos que hoy día es considerado \textit{best practice} incorporar un clasificador basado en aprendizaje automático en el pipeline. Proyectos como CHIME/FRB usan un random forest entrenado con features de cada candidato (S/N, número de haces en que aparece, curvas DM, etc.) para asignar un ``FRB score'' y solo reportar públicamente aquellos por encima de cierto umbral de probabilidad de ser reales \cite{CHIMEFRB_2021_Catalog1}.

Igualmente, telescopios como ASKAP aplican filtros aprendidos (por ejemplo, la red \textbf{FETCH} mencionada se integró en el backend CRAFT de ASKAP para revisar cada candidato en tiempo real). Esto ha conducido a un \textbf{drástico descenso en falsas alarmas} reportadas.

Hace unos años era común que catálogos iniciales de FRBs incluyeran algunos eventos que luego se identificaban como RFI (falsos positivos); gracias a la clasificación ML y verificación cruzada, los catálogos recientes prácticamente no tienen contaminantes.

Además, esta automatización permite operar en modo totalmente robótico: por ejemplo, Apertif/ARTS en Westerbork puede descubrir FRBs y enviar alertas VO en segundos sin intervención humana, apoyado en un clasificador de red neuronal entrenado con el método de Connor \& van Leeuwen (2018) \cite{vanLeeuwen2021}.

En resumen, la clasificación automática se ha consolidado como componente estándar, aumentando la confiabilidad de las detecciones anunciadas.
\item \textbf{Localización automática de eventos:} La localización precisa de un FRB (es decir, determinar su posición en el cielo) es esencial para asociarlo con una galaxia anfitriona o contrapartida, pero con radiotelescopios de haz único tradicionales es prácticamente imposible (sólo se conoce que ocurrió dentro del lóbulo del telescopio, típico error de varios grados). Por ello, se han desarrollado \textit{estrategias de localización interferométrica rápida}.

Una de ellas es usar \textbf{matrices de antenas} que durante la búsqueda graben datos de visibilidad de alta temporización que puedan usarse para formar imágenes post facto del instante del burst. Un ejemplo pionero es el sistema CRAFT en \textbf{ASKAP (Australia)}: este radiotelescopio interferométrico de 36 antenas normalmente promedia a tiempos de segundos, pero el modo CRAFT guarda temporalmente los datos crudos por unos segundos.

Cuando el pipeline de detección (Heimdall+FETCH en este caso) identifica un FRB en uno de los haces formados, el sistema congela y descarga las visibilidades correspondientes y luego realiza una correlación e imagen del evento, logrando obtener la posición del FRB con precisión sub-arco-minuto en minutos. Así, ASKAP logró en 2019 la \textbf{primera localización precisa de un FRB de un solo disparo}, FRB 180924, en una galaxia a z\textasciitilde{}0.3 \cite{Bannister2019}.

Otro ejemplo es \textbf{realfast (VLA)}, un backend en el Very Large Array diseñado por Law \textit{et al}. (2018): corre en paralelo al correlacionador normal y examina las visibilidades en busca de pulsos de ms, y al detectarlos, calcula en tiempo real la posición en el campo (resolución \textasciitilde{}arcosegundos). Realfast consiguió localizar, por ejemplo, bursts del repetidor FRB 180916.J0158+65 con precisión de \textasciitilde{}1" \cite{Marcote2020}.

Estas soluciones representan la \textit{vanguardia} en pipelines de FRB: combinan detección inmediata con \textbf{localización automática}, permitiendo identificar la galaxia anfitriona en cuestión de horas o menos. A día de hoy, gracias a estos avances, alrededor de 20 FRBs (mayormente repetidores pero también algunos de un solo burst) ya cuentan con localizaciones sub-arco-segundo y hosts identificados.
\end{enumerate}
En términos de software, la localización automática implica integrar el pipeline de búsqueda con las etapas de correlación interferométrica y análisis de visibilidades, algo mucho más complejo que la simple detección en una serie temporal. Sin embargo, esfuerzos como el de \textbf{Apertif/ARTS} en el WSRT (Westerbork) están implementando esquemas similares: utilizan una matriz de 12 platos con \textit{phased array feeds} para simultáneamente detectar FRBs y obtener \textit{diferencias de tiempo de arribo} entre antenas, resolviendo así posiciones por \textit{trilateración} en pocos segundos \cite{Oostrum2020}.

Por otro lado, en instrumentos de haz múltiple (no interferométricos) se pueden usar métodos de \textbf{localización por multi-beam}: p. ej., el telescopio de Parkes tiene un receptor de 13 haces; si un FRB se detecta con mayor S/N en uno de ellos, se puede estimar que ocurrió en el lóbulo de ese feed, restringiendo la posición a \textasciitilde{}14' de radio.

Esto se empleó para asociar el primer repetidor FRB 121102 a una región de error de 14' antes de su posterior localización exacta con VLBI \cite{2017Natur.541...58C}. Aunque rudimentaria, esta técnica de \textit{beam ranking} sigue siendo útil para telescopios multi-haz como FAST (19 beams) o CHIME (1024 beams formados), dando una primera indicación de dirección de llegada.
En conclusión, los desarrollos recientes apuntan a \textbf{pipelines integrales} que no sólo detectan los FRBs con alta fiabilidad sino que también \textit{caracterizan} el fenómeno en tiempo real: limpian RFI automáticamente, miden DM/anchos, clasifican como real o espurio, y si la instrumentación lo permite, \textbf{localizan} la fuente en el cielo instantáneamente.

Cada uno de estos pasos se ha visto potenciado por el empleo de aprendizaje automático y por la creciente potencia computacional disponible en las instalaciones. De cara al futuro próximo, con nuevos proyectos como el Square Kilometre Array (SKA) y el Deep Synoptic Array (DSA-200), se espera que decenas de FRBs sean descubiertos \textit{por día}.

Solo con pipelines altamente automatizados (posiblemente basados completamente en IA) será factible manejar esta avalancha de datos y aprovechar al máximo la física que estos fenómenos ofrecen (medir dispersiones para cosmología, estudiar sus posibles periodicidades, sus polarizaciones extremas, etc.).

La comunidad de radioastronomía transitoria ya se está preparando para esa nueva era, construyendo sobre los avances de PRESTO/Heimdall pero incorporando las lecciones aprendidas de proyectos como DRAFTS, FETCH, e introduciendo técnicas de \textit{learning} más sofisticadas (como \textit{learning-to-rank}, \textit{unsupervised anomaly detection} para RFI, e incluso \textbf{redes neuronales gráficas} para combinar información de múltiples telescopios).

En síntesis, el \textbf{estado del arte} de los pipelines de FRB en 2025 es uno de rápida evolución, con un claro movimiento hacia soluciones inteligentes y escalables, capaces de descubrir y distinguir eventos transitorios en medio de datos masivos ruidosos, y extender nuestras observaciones de FRBs a nuevas fronteras de frecuencia y precisión.




