\secnumbersection{MARCO CONCEPTUAL}
\setcounter{secnumdepth}{4}
\setcounter{tocdepth}{4}
\makeatletter
\renewcommand\paragraph{\@startsection{paragraph}{4}{\z@}%
  {1.5ex \@plus .5ex \@minus .2ex}%   % espacio antes
  {0.8ex \@plus .2ex}%                 % espacio después
  {\normalfont\normalsize\bfseries}}  % estilo
\makeatother

\subsection{Introducción a los Fast Radio Bursts}

\subsubsection{Descubrimiento y Primeras Detecciones}

Los Fast Radio Bursts (FRBs), o estallidos rápidos de radio, son pulsos de radio transitorios de duración brevísima (del orden de milisegundos) y extremadamente luminosos en comparación con fuentes galácticas conocidas. Desde el descubrimiento del primer FRB por Lorimer y colaboradores en 2007, identificado en datos archivados del radiotelescopio de Parkes, estos fenómenos han sido objeto de intenso estudio. El pulso detectado por Lorimer tenía un tiempo de dispersión inusualmente alto, lo que sugiere que su origen estaba a distancias cosmológicas más allá de nuestra galaxia \citep{Lorimer2007}. Posteriores detecciones de FRBs reforzaron su naturaleza extragaláctica al presentar medidas de dispersión (DM, por sus siglas en inglés) mucho mayores que las esperables por la Vía Láctea \citep{Thornton2013a}. La DM caracteriza la densidad de electrones libres a lo largo de la línea de visión y causa un retardo de tiempo dependiente de la frecuencia $\nu$ de la señal: los pulsos de radio más bajos en frecuencia llegan más tarde que los de mayor frecuencia, siguiendo una relación aproximada $t(\nu)\propto \nu^{-2}$. Esta dependencia impone una curvatura característica $\nu^{-2}$ en los perfiles de frecuencia-tiempo de los FRBs, distintiva de la dispersión en plasma interestelar \citep{LorimerKramer2004}.

\subsubsection{Propiedades Energéticas y Distancias Cosmológicas}

Cada nuevo FRB descubierto trae información tanto del medio intergaláctico atravesado (a través de la DM y de la dispersión) como del entorno local del progenitor \citep{Masui2015}. Hasta mediados de la década de 2020 se han detectado ya cientos de FRBs publicados --e incluso miles incluyendo los catálogos más recientes-- que confirman que se trata de eventos a distancias cosmológicas con energías isotrópicas típicas de $10^{38}$--$10^{40}$ J (equivalentes a $10^{31}$--$10^{33}$ erg), concentradas en unos pocos milisegundos \citep{Petroff2019,Bochenek2020}. Esta enorme energía implica fuentes astrofísicas extremas. 

\subsubsection{Repetidores vs. Eventos Únicos y Modelos de Progenitores}

Sin embargo, la naturaleza exacta de los FRBs permaneció elusiva durante años: inicialmente se propusieron modelos desde colisiones cataclísmicas (por ejemplo, la fusión de estrellas de neutrones) hasta emisiones coherentes de objetos compactos aislados \citep{Lorimer2007,Platts2019}. La posterior detección de FRBs repetidores (es decir, que emiten múltiples ráfagas no destructivas) descartó ciertos escenarios de destrucción única y apuntaló modelos con estrellas de neutrones jóvenes altamente magnetizadas (magnetars) como posibles fuentes \citep{Spitler2016,Margalit2018}.

\subsubsection{Observaciones en Diferentes Bandas de Frecuencia}

Dada la gran distancia de origen, todos los FRBs conocidos hasta la fecha se han observado a longitudes de onda de radio centimétricas y decimétricas, donde los radiotelescopios ofrecen alta sensibilidad \citep{Manchester1977}. A estas frecuencias relativamente bajas, los FRBs exhiben sus mayores luminosidades, pero también sufren en gran medida los efectos de propagación como la dispersión y el ensanchamiento por scattering (dispersión multi-trayectoria) en plasma turbulento. La observación en frecuencias más altas, en rango milimétrico (mm), podría mitigar el scattering (que escala fuertemente con $\nu^{-4}$), permitiendo detectar pulsos más intrínsecamente breves. No obstante, a mayor frecuencia también disminuye típicamente la potencia de emisión de estos transitorios y el campo de visión de los telescopios, planteando nuevos desafíos técnicos. En las secciones siguientes se abordan los fundamentos físicos y fenomenológicos de los FRBs, casos emblemáticos de fuentes repetidoras, la posible conexión con magnetars (incluyendo el caso del magnetar galáctico SGR 1935+2154), las oportunidades y retos de su detección en bandas de alta frecuencia (hasta ondas milimétricas), los métodos clásicos de búsqueda basados en dedispersión de pulsos, y las soluciones modernas de aprendizaje automático para la clasificación automática de candidatos.

\subsection{Fundamentos Físicos y Fenomenología Observacional de los FRBs}

\subsubsection{Dispersión en Plasma y Medida de DM}

Un FRB típico se manifiesta como un pulso único de radio de duración brevísima ($\sim$1--10 ms) que sobresale muy significativamente por encima del ruido de fondo. En el dominio tiempo-frecuencia, la señal presenta la firma de la dispersión en plasma: las componentes de menor frecuencia arriban con mayor retraso temporal respecto a las de mayor frecuencia, según la ley $\Delta t \propto \text{DM} \cdot \nu^{-2}$. Esta ley implica que, cuando se observa en un ancho de banda amplio, el pulso aparece como un trazo inclinado o curvo en el diagrama de frecuencia vs. tiempo, donde las bajas frecuencias se retrasan más \citep{CordesMcLaughlin2003}. Al corregir este retraso mediante un proceso de dedispersión para un cierto valor de DM, las componentes de frecuencia se alinean en el tiempo revelando el pulso comprimido. La Figura~\ref{fig:frb121102_espectro} ilustra un ejemplo de la dinámica de un FRB en frecuencia-tiempo: se muestra un pulso de FRB 121102 detectado con el telescopio Lovell a 1.4 GHz, donde tras dedispersar los datos se evidencian dos componentes de emisión separadas en el tiempo (un pulso precursor débil precediendo al pulso principal por $\sim17$ ms). Este precursor sugiere emisiones múltiples dentro de la misma ráfaga, una característica observada en varios FRBs repetidores.

\begin{figure}[h]
\centering
% TODO: Agregar figura aquí
\caption{Espectro dinámico (frecuencia vs. tiempo) de una ráfaga del repetidor FRB 121102 observado con el radiotelescopio Lovell en banda L ($\sim$1.4 GHz). Se ha aplicado dedispersión coherente con la DM del pulso principal para alinear las frecuencias. Se aprecia un pulso secundario más débil precediendo al pulso principal por $\sim17$ ms (indicativo de subestructuras o precursores en la emisión). Las franjas horizontales corresponden a canales de frecuencia filtrados por interferencia de radiofrecuencia (RFI). Adaptado de \citet{Rajwade2020}.}
\label{fig:frb121102_espectro}
\end{figure}

La medida de dispersión (DM) es una magnitud central en el estudio de FRBs. Se define como la integral de la densidad de electrones libres $n_e$ a lo largo de la línea de visión hasta la fuente: ${\rm DM} = \int_0^D n_e\, dl$, típicamente expresada en unidades de pc cm$^{-3}$. Dado que las señales de radio sufren un retraso relativo $\Delta t \approx 4.15\,{\rm ms} \times ({\rm DM}/{\rm pc\,cm^{-3}}) \times[(\nu_{\rm baja}/{\rm GHz})^{-2} - (\nu_{\rm alta}/{\rm GHz})^{-2}]$ \citep{LorimerKramer2004}, una DM anómalamente alta es evidencia de un origen extragaláctico o cosmológico. Los FRBs descubiertos hasta ahora exhiben DM de cientos hasta miles de pc cm$^{-3}$, excediendo por mucho el contenido electrónico de la Vía Láctea en sus direcciones \citep{Petroff2019}. Por ejemplo, el FRB inicial de Lorimer tenía DM $\sim375$ pc cm$^{-3}$, varias veces mayor que la contribución galáctica esperada en su coordenada, implicando una distancia de orden gigapársec si el exceso de DM proviene del medio intergaláctico \citep{Lorimer2007}. La DM se utiliza entonces como un indicador de distancia cosmológica: aunque existe dispersión en la relación DM--$z$ por la distribución inhomogénea del medio intergaláctico, a primera aproximación DMs del orden $10^3$ pc cm$^{-3}$ sugieren orígenes a redshifts $z\sim0.5$--1 \citep{Inoue2004,Tendulkar2017}.

\subsubsection{Scattering y Ensanchamiento Temporal}

Además de la dispersión, los FRBs a menudo exhiben ensanchamiento temporal por scattering multipath en el plasma, visible como una cola exponencial en el pulso dedispersado, más pronunciada a frecuencias bajas. Este efecto, causado por la difusión angular en plasmas turbulentos, atenúa la detectabilidad de pulsos breves y puede variar de microsegundos a decenas de milisegundos dependiendo de la frecuencia y la línea de visión \citep{Cordes2016}. A frecuencias de unos pocos GHz, varios FRBs muestran anchuras de pulso intrínsecas del orden de cientos de microsegundos o menos una vez corregido el scattering (e.g., FRB 121102; \citet{Hardy2017,Michilli2018}). Se ha observado una clara dependencia con la frecuencia: los pulsos tienden a ser más breves en altas frecuencias, tanto por la reducción del scattering como posiblemente por un ancho intrínseco menor \citep{Gajjar2018a,Nimmo2021}. Por ejemplo, ráfagas de FRB 121102 detectadas a 4--8 GHz tenían anchos de $\sim30$--100 $\mu$s, mucho menores que las de 1.4 GHz que típicamente son de varios milisegundos \citep{Gajjar2018a,Hessels2019}.

\subsubsection{Polarización y Rotación Faraday}

Otro aspecto observacional clave es la polarización de la emisión. Muchos FRBs muestran altos grados de polarización lineal, cercanos al 100\% en algunos casos, acompañados de medidas de rotación (RM) extremadamente grandes, lo que indica campos magnéticos intensos y plasmas denso-ionizados en las cercanías de la fuente. El caso emblemático es FRB 121102, cuyo RM fue medido en $\sim$+$1.4\times10^5$ rad m$^{-2}$ (en el marco de la fuente) --el mayor jamás registrado para una fuente astronómica extragaláctica-- y varió ligeramente con el tiempo \citep{Michilli2018}. Este valor sugiere que el FRB reside en un entorno magnetizado extremo, posiblemente un plasma denso cercano a un agujero negro o resto de supernova joven. La fuerte polarización lineal, combinada con RMs grandes, apoya la conexión con magnetars u otros objetos de campos ultraintensos \citep{Masui2015,Michilli2018}. De hecho, casi todos los FRBs con buena detección polarimétrica presentan polarización lineal significativa y, en algunos casos, componentes de polarización circular moderada, indicando que la emisión es coherente y se produce en presencia de campos ordenados.

\subsubsection{Morfología Espectro-Temporal y Sub-pulsos}

Más allá de estos rasgos generales, los FRBs muestran una morfología espectro-temporal variada. Algunas ráfagas constan de uno solo pulso amplio en banda, mientras otras exhiben sub-pulsos múltiples que suelen derivar en frecuencia --un fenómeno coloquialmente llamado ``efecto trombón triste''-- donde sub-estallidos sucesivos aparecen a frecuencias menores con retrasos crecientes \citep{Hessels2019}. Este drift en frecuencia se ha observado sobre todo en FRBs repetidores, comenzando con FRB 121102 \citep{Gajjar2018b} y posteriormente en varias fuentes repetitivas \citep{CHIME_FRB_Collaboration_2020,Josephy2019}. Por el contrario, la mayoría de FRBs de evento único descubiertos con ASKAP muestran espectros puntuales e irregularmente modulados, concentrando la energía en rangos de frecuencia estrechos (pocos cientos de MHz) con poco o nulo drift apreciable \citep{Shannon2018,Cho2020}. Esta diferencia llevó a sugerir arquetipos morfológicos distintos para FRBs repetidores vs. no repetidores \citep{Pleunis2021}. Sin embargo, las comparaciones deben tomarse con cautela, ya que cada radiotelescopio y cada algoritmo de detección introducen sesgos; estudios recientes con muestras homogéneas (como el catálogo de CHIME) aún evalúan estadísticamente estas diferencias \citep{Fonseca2020,Pleunis2021}.

\subsubsection{Incompletitud en Detección y Ráfagas Atípicas}

Un desafío importante es que los métodos tradicionales de búsqueda pueden estar dejando escapar ráfagas atípicas. Por ejemplo, \citet{Gourdji2019} analizaron pulsos débiles de FRB 121102 y hallaron que muchos eran estrechos en banda ($\sim$200 MHz de ancho efectivo a 1.4 GHz), lo cual reduce su visibilidad en búsquedas de amplio ancho de banda con dedispersión incoherente. Esto sugiere que los FRBs con espectros inusualmente limitados o estructuras complejas podrían no activarse en los algoritmos estándar, subestimando la tasa real de detección \citep{Gourdji2019}. En resumen, la fenomenología observacional de los FRBs es rica: desde pulsos únicos hasta trenes de sub-pulsos modulados, con anchos de microsegundos a milisegundos, dispersados y a veces fuertemente polarizados. Estas propiedades proporcionan pistas cruciales sobre la naturaleza de sus progenitores y entornos, como se discutirá en las próximas secciones.

\subsection{Casos Históricos y FRBs Repetidores Notables}

\subsubsection{FRB 121102: El Primer Repetidor}

Tras los primeros FRBs aislados, el hallazgo revolucionario vino en 2016 con la identificación del primer FRB repetidor, FRB 121102, por Spitler y colaboradores. FRB 121102 (también catalogado como FRB20121102A) fue inicialmente descubierto como un solo evento en 2014 con Arecibo \citep{Spitler2014}, pero un programa de seguimiento reveló múltiples estallidos recurrentes desde la misma coordenada en años posteriores \citep{Spitler2016}. Esta fuente, ubicada en la constelación de Auriga, se ha convertido en la más estudiada de los FRBs gracias a sus frecuentes repeticiones. Su importancia radica en que demostró que al menos algunos FRBs provienen de objetos persistentes y no de eventos catastróficos de un solo uso (como explosiones destructivas), apoyando escenarios de remanentes activos (como estrellas de neutrones jóvenes).

FRB 121102 presentó varias características extraordinarias. Además de su ya mencionada altísima RM \citep{Michilli2018} que implicaba un entorno magnetizado intenso, fue el primer FRB precisamente localizado en el cielo: mediante observaciones interferométricas con el Very Large Array se logró asociarlo a una diminuta galaxia enana de bajo metalicidad a $z\approx0.19$ \citep{Chatterjee2017,Tendulkar2017}. En dicha galaxia, FRB 121102 reside coincidente con una región HII de intensa formación estelar, y se identificó además una fuente de radio persistente de bajo brillo en las cercanías \citep{Chatterjee2017}. Todo ello sugiere que el FRB podría originarse en un objeto compacto (p.ej. un magnetar joven o un pulsar nebular) alojado en un entorno denso (quizá el remanente de una supernova o cerca de un agujero negro). Este cuadro pionero transformó el campo: FRB 121102 pasó a ser un laboratorio único para estudiar la variabilidad y espectro de los FRBs. Se han detectado ya cientos de pulsos de esta fuente a lo largo de múltiples campañas, abarcando frecuencias desde $\sim$600 MHz hasta 8 GHz. Notablemente, \citet{Gajjar2018a} reportaron más de una docena de ráfagas en solo 30 minutos de observación con el telescopio de Green Bank a 4--8 GHz, y mostraron que a dichas altas frecuencias los pulsos de FRB 121102 tienden a ser más breves y con subestructuras espectrales definidas. También \citet{Hessels2019} encontraron una compleja estructura tiempo-frecuencia con deriva sistemática de sub-pulsos, lo cual incluso planteó la posibilidad de usar la morfología como criterio para distinguir FRBs repetidores de no repetidores.

\subsubsection{FRB 180916: Periodicidad de 16 Días}

Otro descubrimiento trascendental fue el de FRB 180916.J0158+65 (también designado 20180916B), el segundo repetidor confirmado inicialmente anunciado por el experimento canadiense CHIME/FRB \citep{CHIME_FRB_Collaboration_2020}. Este FRB, de DM $\sim348$ pc cm$^{-3}$, es particularmente notable porque CHIME posteriormente detectó en él un comportamiento periódico en su actividad: ráfagas recurrentes agrupadas en ventanas de unos pocos días, repitiéndose cada $\sim16.35$ días \citep{CHIME_FRB_Collaboration_2020}. Fue la primera evidencia de periodicidad en los FRBs, publicada en 2020, y generó enorme interés ya que sugiere modulación por un ciclo astronómico. El origen de dicha periodicidad aún se debate; modelos propuestos incluyen un sistema binario donde el FRB (posiblemente un magnetar) orbita una estrella masiva y solo es observable en parte de la órbita \citep{Lyutikov2020}, o un magnetar aislado con precesión del eje de emisión produciendo intervalos de actividad \citep{Levin2020}. 

\subsubsection{Ciclos de Actividad y Periodicidad en Repetidores}

En cualquier caso, FRB 180916 mostró que no todos los repetidores son aleatorios: algunos pueden tener ciclos de actividad. Hasta ahora, este FRB de 16 días es el ejemplo más firme, aunque hay indicios tentativos de periodicidades más largas en otros casos. Precisamente, \citet{Rajwade2020} analizaron 5 años de datos de FRB 121102 (el primer repetidor) y sugirieron una posible modulación periódica de $\sim161$ días en sus épocas activas, con una ventana activa de $\sim54$ días seguida de $\sim107$ días ``apagado''. Este resultado fue anunciado con cautela debido a la limitada muestra de pulsos y a largos periodos sin detección, y estudios posteriores con más detecciones \citep{cruces2020frb121102} han refinado o cuestionado ese valor. No obstante, la posibilidad de que los FRBs repetidores tengan escalas de actividad cíclica sugiere que podrían ser objetos en sistemas binarios o con precesión periódica, aportando una pieza más al rompecabezas de su origen.

\subsubsection{Propiedades Distintivas de FRBs Repetidores}

A la fecha, se han confirmado observacionalmente varias decenas de FRBs repetidores \citep{Fonseca2020,CHIME2021}. La gran mayoría de FRBs descubiertos ($\approx95\%$) siguen siendo eventos no repetidos en las observaciones hasta ahora \citep{Petroff_2022}. No está claro si los FRBs de una sola aparición forman una clase distinta (por ejemplo, explosiones catastróficas que destruyen la fuente) o si simplemente sus repeticiones son muy raras o difíciles de detectar. Los repetidores bien estudiados, como FRB 121102 y FRB 180916, comparten algunas propiedades: tienden a tener múltiples componentes tiempo-frecuencia (sub-pulsos con drift), espectros más estrechos y, en promedio, anchos de pulso algo mayores que los FRBs uno-a-uno detectados con instrumental equivalente \citep{Fonseca2020,Pleunis2021}. Adicionalmente, algunos repetidores muestran cambios en sus propiedades entre distintas épocas: por ejemplo, FRB 121102 exhibió variaciones en DM y RM sutiles a lo largo de años, indicando cambios en el plasma circundante \citep{cruces2020frb121102}, y FRB 180916 mostró que sus ráfagas a 400 MHz (CHIME) y a 1.4 GHz (otras antenas) ocurrían en distintas fases del ciclo de 16 días, sugiriendo que las emisiones de baja y alta frecuencia podrían activarse en momentos diferentes dentro del periodo \citep{PastorMarazuela2021}. Todos estos hallazgos apuntan a escenarios de progenitores dinámicos --posiblemente estrellas de neutrones jóvenes con un entorno local en evolución, o interacciones binarias--. La siguiente sección profundiza en el modelo actualmente más favorecido de progenitor: los magnetars o magnetares, vinculado fuertemente a raíz de un descubrimiento notable en nuestra propia Galaxia.

\subsection{Magnetars y su Vínculo con los FRBs}

\subsubsection{Características de los Magnetars}

Los magnetars son un subtipo de estrellas de neutrones con campos magnéticos extraordinariamente intensos ($\geq10^{14}$ Gauss en superficie). Estas estrellas pueden liberar abruptamente su energía magnética en forma de poderosos destellos de rayos X/gamma y en emisiones de radio coherentes. Desde hace algunos años se ha propuesto que los magnetars jóvenes podrían ser los progenitores de muchos FRBs extragalácticos \citep{PopovPostnov2010,Lyutikov2017}. Diversas evidencias sostienen esta conexión: (i) los FRBs requieren altas energías en intervalos brevísimos, coherentes con erupciones magnetares; (ii) su localización en galaxias con intensa formación estelar sugiere objetos jóvenes \citep{Tendulkar2017,Bochenek2020}; (iii) las altas RMs medidas, como en FRB 121102, apuntan a entornos como los vientos magnetizados de un magnetar o su remanente de supernova \citep{Michilli2018}. 

\subsubsection{SGR 1935+2154: El FRB Galáctico}

Pero la prueba más contundente surgió el 28 de abril de 2020, cuando por primera vez se detectó una ráfaga de radio en nuestra Galaxia con propiedades análogas a un FRB: el magnetar SGR 1935+2154 emitió un pulso doble de $\sim$ms de duración con una fluencia colosal de $\sim1.5 \times 10^6$ Jy$\cdot$ms \citep{Bochenek2020}. Este evento, designado como FRB 200428 en el contexto extragaláctico, fue captado independientemente por dos experimentos: STARE2 en microondas \citep{Bochenek2020} y CHIME/FRB en 400--800 MHz \citep{CHIME_SGR2020}. La energía isotrópica equivalente liberada (asumiendo la distancia de 9 kpc del magnetar) fue del orden de $10^{34}$ J, unas $10^3$--$10^4$ veces mayor que cualquier pulso de radio galáctico conocido, aunque aún aproximadamente 30--40 veces más débil que el FRB extragaláctico menos energético registrado \citep{Bochenek2020}. Esto coloca al evento de SGR 1935 en el rango intermedio entre la población de ráfagas de magnetar galácticas y los FRBs cosmológicos, fortaleciendo la idea de continuidad física entre ambos fenómenos. Además, de forma crucial, la ráfaga de radio coincidió con un estallido de rayos X prácticamente simultáneo detectado por satélites \citep{Mereghetti2020,Li2021}, lo que concuerda con modelos en los que una reconexión o erupción en la magnetosfera del magnetar genera tanto un pulso de radio coherente (p.ej. mediante un mecanismo de máser de plasma) como un pulso de alta energía.

\subsubsection{Implicaciones para el Origen de FRBs}

La implicación de este descubrimiento es que magnetars activos pueden explicar al menos parte de la población de FRBs. Un magnetar joven (edad $\sim10^2$--$10^4$ años) en nuestra Galaxia emitiendo una erupción inusualmente poderosa se vería desde otras galaxias como un FRB típico \citep{Bochenek2020}. Cabe destacar que FRB 200428 (SGR 1935+2154) presentó dos picos sub-milisegundos separados por $\sim30$ ms y con drifts espectrales opuestos en banda, muy parecidos a sub-pulsos de FRBs repetidores \citep{Zhang2020}, reforzando la analogía. Tras este hallazgo, otros magnetars galácticos han sido seguidos atentamente en busca de emisiones FRB, aunque sin detecciones comparables hasta ahora.

\subsubsection{PSR J1745-2900: Emisión en Banda Milimétrica}

Los magnetars no solo pueden emitir en radio a frecuencias estándar ($\sim1$ GHz), sino potencialmente en bandas mucho más altas. De hecho, algunos magnetars galácticos se han detectado como radiosources persistentes de baja luminosidad. Un caso importante es el magnetar de nuestro centro galáctico, PSR J1745--2900, que orbita a unos pocos años luz del agujero negro central Sgr A*. Este magnetar mostró pulsos de radio tras su nacimiento en 2013, y estudios posteriores confirmaron su emisión hasta frecuencias milimétricas. Recientemente, \citet{veracasanova2025} reportaron la detección de pulsos altamente polarizados de PSR J1745--2900 usando ALMA a $\sim86$ GHz (longitud de onda $\sim3.5$ mm), representando la primera observación de ráfagas de un magnetar en el régimen mm. Se detectaron 8 pulsos individuales con energías en el rango de $10^{29}$ erg, y sorprendentemente, la distribución de energías siguió una ley de potencia con exponente $-2.4$ consistente con la observada tanto en magnetars a frecuencias cm como en FRBs repetitivos \citep{veracasanova2025}. Esto sugiere una continuidad en los mecanismos de emisión a lo largo de muchas décadas de frecuencia.

La detección de pulsos hasta $\sim300$ GHz (onda sub-mm) es un territorio aún inexplorado para los FRBs extragalácticos, pero los resultados con magnetars galácticos son prometedores. Si los FRBs son realmente magnetars, no hay una razón fundamental para que cesen de emitir en banda mm, salvo la decaída de la potencia con la frecuencia. Las ventajas de observar FRBs en alta frecuencia incluyen: (i) el scattering intergaláctico e interestelar es mucho menor a mm (permitiendo captar la forma intrínseca del pulso), y (ii) ciertos FRBs en entornos densos podrían ser opacos a bajas frecuencias pero visibles a frecuencias altas que penetran mejor \citep{Yang2020}. De hecho, se ha planteado que un FRB asociado a un magnetar en una región muy densa (como una supernova joven) podría ser indetectable por debajo de $\sim1$ GHz debido a plasmas con alta absorción libre-libre o dispersión, pero detectable en bandas mm una vez suficientemente brillante \citep{Omand2022}. Por otra parte, los desafíos son considerables: la emisión de sincrotrón coherente de magnetar/pulsar suele disminuir con la frecuencia (espectros de índice $-1$ a $-3$), por lo que las ráfagas en mm serían más débiles; además, los telescopios mm tienen campos de visión pequeños. No obstante, instrumental como el Atacama Large Millimeter/submillimeter Array (ALMA) en modo fasado (phased array) ha demostrado ser capaz de descubrir pulsos en mm. ALMA, al combinar 50+ antenas de 12 m, actúa como un único plato de diámetro efectivo $\sim84$ m, el más sensible en el rango mm. Con su alta resolución temporal (hasta $\sim100$ $\mu$s), ALMA puede buscar FRBs y pulsos de magnetar en bandas 3, 4, 6 (85--150 GHz) e incluso banda 1 (35--50 GHz) que próximamente entrará en operación con receptores dedicados.

Los estudios piloto sugieren que varias decenas de púlsares galácticos conocidos podrían ser detectados por ALMA en mm \citep{veracasanova2025}, y estiman que si se apunta ALMA hacia un FRB repetidor durante sus fases de alta actividad, se podrían registrar varios estallidos por hora si su espectro se extiende suficientemente en frecuencia. En suma, los magnetars proporcionan un nexo plausible entre fenómenos locales y cosmológicos. La asociación directa de SGR 1935+2154 con un FRB y la posibilidad de estudiar magnetars en rango milimétrico abren nuevas ventanas observacionales que pueden confirmar definitivamente el origen magnetar de los FRBs, o revelar la existencia de múltiples poblaciones de progenitores.

\subsection{Observación de FRBs en Alta Frecuencia (ondas milimétricas)}

La extensión de las búsquedas de FRBs hacia frecuencias milimétricas (30--300 GHz) representa tanto
una oportunidad emocionante como un desafío técnico. Históricamente, casi todos los FRBs se han
detectado en bandas de radio centimétricas (de 100 MHz a varios GHz) donde la emisión es más intensa
y los radiotelescopios tienen gran área colectora. Sin embargo, hay motivaciones clave para explorar
frecuencias más altas:

\subsubsection{Mitigación de Dispersión y Scattering}

La mitigación de la dispersión y scattering: Como se indicó, los retardos por dispersión decrecen
fuertemente con la frecuencia ($\Delta t \propto \nu^{-2}$), y la dispersión multi-trayectoria
(scattering) aún más abruptamente ($\tau_{\rm sc} \propto \nu^{-4}$ típicamente). A frecuencias
mm, un pulso atravesando un mismo plasma sufre un ensanchamiento temporal miles de veces
menor que a 1 GHz. Esto significa que FRBs muy distantes o en entornos densos, cuyos pulsos
en banda L (1.4 GHz) estarían ensanchados por scattering a decenas de ms y casi irreconocibles,
podrían manifestarse nítidamente a 100 GHz con anchuras intrínsecas de solo $\mu$s o ms. Además,
el barrido de dispersión dentro del ancho de banda de recepción es mínimo. Por ejemplo, en
ALMA banda 3 (centro $\sim86$ GHz, ancho 2 GHz), la demora entre el extremo inferior y superior de
banda para un FRB con DM $\sim1770$ pc cm$^{-3}$ (como FRB 121102) es de apenas $\sim46$ $\mu$s
. En tales condiciones, ni siquiera es necesario dedispersar en tiempo real para detectar
el pulso; en la búsqueda de pulsos del magnetar galáctico con ALMA, se fijó la DM conocida y se
sumaron las frecuencias sin pérdida significativa de S/N \citep{veracasanova2025}. La
contrapartida es que con un retardo tan reducido, la característica curva $\nu^{-2}$ deja de ser
evidente en un solo espectro dinámico estrecho, eliminando una de las firmas clásicas para
distinguir un pulso astrofísico de un espurio de RFI. En ALMA, por ejemplo, al no haber una
deriva detectable en frecuencia dentro de la banda, la confirmación de pulso requirió apoyarse
en la alta polarización intrínseca y en la coincidencia con la fase de pulsar esperada (en el caso
del magnetar PSR J1745--2900). Para FRBs extragalácticos, esto implicaría que detectar
un FRB a mm sin conocimiento previo de su DM sería difícil, ya que la ``huella'' de dispersión en
una pequeña banda mm sería casi imperceptible. Por tanto, las búsquedas mm podrían
necesitar ya conocer la DM (p. ej., apuntar a una fuente repetidora previamente caracterizada en
cm) o cubrir un ancho de banda muy grande para captar alguna curvatura.

\subsubsection{Entornos Densos y Opacidad}

Entornos densos y opacidad: Algunos modelos proponen que ciertos FRBs jóvenes residen en
entornos densos (como restos de supernova o vientos nebulares) que podrían absorber o
dispersar completamente la emisión de radio por debajo de una frecuencia de corte, pero dejar
pasar componentes de mayor frecuencia. Observar en mm abriría la posibilidad de detectar
FRBs ``oscuros'' a bajas frecuencias. Un caso ilustrativo es el FRB repetidor 20201124A, cuya
densa región de plasma observada sugiere que podría atenuar fuertemente su propia emisión
de baja frecuencia \citep{Hilmarsson2022}. En general, las ondas mm atraviesan columnas de
plasma con mucha menos atenuación libre-libre que ondas de 20--50 cm. Por ello, telescopios
como ALMA podrían eventualmente descubrir FRBs en galaxias muy activas o entornos que
resulten invisibles para instrumentos como CHIME o Parkes.

\subsubsection{Telescopios y Sensibilidad en Banda Milimétrica}

Telescopios y sensibilidad: El principal instrumento capaz de estas observaciones es ALMA.
Operando en modo fasado, ALMA ha demostrado una sensibilidad incomparable en 3 mm: la
detección de pulsos de $\sim1$ Jy con anchos de unos ms es factible con unas pocas decenas de
milisegundos de integración (en comparación, una antena single-dish de 12 m tardaría minutos
en alcanzar similar S/N). En la Tabla 1 de \citet{veracasanova2025} se comparan retardos de
dispersión entre banda L (1.4 GHz) y bandas mm: para DM altas ($\sim2000$), Effelsberg a 1.3 GHz
experimenta retardos de $\sim17.93$ s (!) entre 1.21--1.51 GHz, mientras ALMA a 86 GHz tendría solo
46 $\mu$s. Esto ilustra cuán limpio es el régimen mm respecto a la dispersión. No obstante,
ALMA y otros observatorios mm tienen campos de visión pequeños (de pocos minutos de arco a
3 mm), por lo que no es práctico realizar barridos ciegos de grandes áreas del cielo como se
hace a 1.4 GHz. En su lugar, la estrategia es apuntar a fuentes FRB conocidas (por ejemplo,
repetidores con coordenadas catalogadas) durante sus fases activas. Telescopios como el IRAM
30m o el futuro QTT en 110 GHz también podrían contribuir con detectores de pulsos.

\subsubsection{Resultados Iniciales y Expectativas}

Resultados iniciales y expectativas: La detección de los 8 pulsos del magnetar galáctico con
ALMA banda 3 prueba la eficacia del instrumento y la existencia de emisión pulsada a altas
frecuencias \citep{veracasanova2025}. Extrapolando esos resultados, los autores estiman que
más de 160 púlsares galácticos conocidos deberían ser visibles por ALMA en al menos una de
sus bandas. En el caso de FRBs extragalácticos, la predicción es más incierta debido a sus
desconocidos espectros de alta frecuencia. Si asumimos que un FRB repetidor mantiene un
índice espectral similar al promedio de púlsares ($\alpha \sim -1.6$; \citet{Jankowski2017}) o de
FRBs de ASKAP ($\sim -1.5$; \citet{Macquart2019}), una ráfaga de 1 Jy$\cdot$ms a 1.4 GHz tendría una
fluencia de $\sim0.1$ Jy$\cdot$ms a 86 GHz. ALMA podría detectar eso marginalmente. Sin embargo,
algunos FRBs repetidores han mostrado espectros más planos o invertidos en ciertas ráfagas
\citep{Law2017,Gajjar2018a}, lo que abre la posibilidad de ráfagas relativamente fuertes
en mm. Es revelador que durante la campaña de 2020 del repetidor FRB 180916, no se
detectaron pulsos en simultáneo con el telescopio óptico FAST a 1250 MHz mientras sí se veían
en CHIME a 600 MHz, insinuando un posible debilitamiento hacia frecuencias más altas \citep{Xu2021}. Por el contrario, FRB 121102 fue visible hasta 8 GHz. En conclusión, aunque todavía
ningún FRB extragaláctico ha sido reportado en mm, los próximos años podrían cambiar esto.
Programas de busca de FRBs con ALMA ya están en marcha, priorizando repetidores durante
ventanas pronosticadas de actividad (por ejemplo, FRB 121102 en sus intervalos ``on'' sugeridos
de $\sim160$ días, o FRB 180916 en su fase pico cada 16 días). La primera detección de un FRB en
banda mm supondría un gran avance, confirmando la extensión de su espectro y permitiendo
mediciones de DM sin degeneración con RM (ya que a altas frecuencias la rotación Faraday es
casi imperceptible mientras la dispersión sigue presente).

\subsubsection{Desafíos Técnicos y Validación de Señales}

Entre las dificultades únicas de las observaciones mm cabe mencionar: (a) la sensibilidad requerida
--los FRBs en mm podrían ser mucho más débiles, requiriendo integración muy rápida y muchos
radiotelescopios en fase--; (b) la casi nula curvatura de dispersión dentro de bandas estrechas, haciendo
difícil distinguir eventos astrofísicos de RFI sin otras pistas (como polarización o coincidencia temporal
con telescopios de menor frecuencia); y (c) la posible variación de la morfología de la señal --si la
emisión FRB proviene de pulsos coherentes de plasmas relativistas, sus propiedades espectrales
podrían cambiar con la frecuencia, por ejemplo pulsos más cortos y espectralmente más simples en
mm que en 600 MHz. La experiencia con pulsos de magnetar en ALMA sugiere que la polarización es un
criterio útil para validación: en 86 GHz, todos los pulsos detectados de PSR J1745--2900 mostraron
fracciones de polarización (lineal o circular) $>50\%$, claramente diferenciados del ruido o RFI \citep{veracasanova2025}. Asimismo, conocer de antemano la periodicidad de la fuente (si es un pulsar o
magnetar) ayuda a descartar señales espurias. En el caso de FRBs cosmológicos, que no repiten de
forma predecible, la confirmación tendría que venir de detecciones simultáneas multi-frecuencia: por
ejemplo, un FRB visto por CHIME a 600 MHz y por ALMA a 100 GHz en el mismo instante, con la misma
DM, sería evidencia incontrovertible de emisión broadband.

Las observaciones en alta frecuencia representan la frontera emergente en el estudio de
FRBs. Si bien intrincadas, prometen grandes recompensas: podrían revelar FRBs escondidos,
proporcionar pulsos libres de distorsión por plasma, y aportar restricciones severas a los modelos de
emisión (por ejemplo, si se detecta un corte espectral). Telescopios de próxima generación como el
Deep Space Network 70m (en banda Ka) o configuraciones de VLBI con antenas de plato grande a 20--
50 GHz también contribuirán. De momento, ALMA, gracias a su sensibilidad sin precedentes, se perfila
como pionero en la búsqueda de FRBs en el dominio milimétrico, ofreciendo la posibilidad tangible de
explorar una ventana frecuencial hasta ahora inexplorada para estos misteriosos estallidos
cósmicos.

\subsection{Métodos Tradicionales de Búsqueda y Detección de FRBs}

\subsubsection{Pipeline Clásico de Detección}

La detección de FRBs en vastos volúmenes de datos de radio requiere métodos eficientes para extraer
señales cortas y dispersas entre ruido e interferencia. Los métodos tradicionales se basan en aplicar la
corrección de dispersión a los datos de forma exhaustiva, seguida de filtrados y umbrales para
identificar candidatos. En la práctica, un radiotelescopio registra voltajes o potencias en muchas subbandas de frecuencia con alta resolución temporal (usualmente milisegundos o menor). Un pipeline de búsqueda de FRBs típicamente ejecuta los siguientes pasos fundamentales \citep{Petroff2015,KeanePetroff2015}:

\begin{enumerate}
\item \textbf{Dedispersión por ensayo de DM:} Dado que la DM de una posible ráfaga es desconocida de
antemano, se genera un banco de triales de DM que cubren desde 0 (ninguna dispersión) hasta
un máximo elevado (p.ej. 5000 pc cm$^{-3}$ para búsquedas a escala cosmológica). Para cada
valor de DM, se aplican retrasos apropiados a las series de cada canal de frecuencia (o subbanda) para alinearlas en el tiempo, integrando en frecuencia para obtener una serie temporal
dedispersada. Este proceso puede hacerse de forma directa (brute-force) o mediante algoritmos
optimizados como la Transformada Dispersiva Rápida (FDMT; \citet{ZackayOfek2017}) que
explotan transformadas acumulativas. Herramientas populares como HEIMDALL \citep{Champion2016} realizan esta dedispersión de manera paralela en GPUs, permitiendo explorar miles de
DM en tiempo real. Alternativamente, PRESTO \citep{Ransom2001} implementa dedispersión por subbandas de forma eficiente en CPU.

\item \textbf{Filtrado por anchura de pulso:} Los pulsos pueden sufrir dispersión intrínseca o scattering,
haciéndolos más anchos que la resolución nativa. Para maximizar la detección, se aplica una
familia de filtros de suavizado (convolución con ventanas) a cada serie temporal dedispersada
para distintas longitudes (p. ej., 1, 2, 4, 8, ..., 128 ms). Esto simula la integración de la señal en
diferentes anchos de pulso, aumentando la S/N de pulsos más anchos a costa de agregar ruido
correlacionado. Tras este paso, para cada DM se obtiene un conjunto de series filtradas.

\item \textbf{Detección de umbral:} Se recorre cada serie dedispersada y filtrada en busca de excedencias
sobre un umbral de S/N predefinido (típicamente 6$\sigma$--8$\sigma$). Cada vez que una muestra supera el
umbral, se registra un candidato con sus parámetros: DM, tiempo de llegada, S/N y anchura
(correspondiente al filtro óptimo).

\item \textbf{Agrupamiento y coincidencias:} Dado que un pulso real suele generar múltiples detecciones
cercanas (por ejemplo, en DMs adyacentes o en anchos parecidos), los candidatos brutos se
agrupan para evitar duplicados. Se agrupan por proximidad en tiempo y DM utilizando
algoritmos de friends-of-friends o kNN en el espacio (t, DM) \citep{BurkeSpolaor2011}. Así,
múltiples detecciones de un mismo evento colapsan a un único candidato representativo con la
DM y anchura que dio máxima S/N.

\item \textbf{Rechazo de interferencia (RFI):} En paralelo, se realizan filtros para descartar señales originadas
por RFI terrestre. Un criterio común es eliminar detecciones con DM cerca de 0 pc cm$^{-3}$, ya
que la RFI no astrofísica no presenta dispersión. También se aprovechan receptores multihaz: si
el telescopio tiene varios haces apuntados adyacentes (como Parkes 13-beams, ASKAP 36-
beams, outrigger arrays), una señal astrofísica puntual solo aparecerá en un haz, mientras que la
RFI tiende a ser vista en varios por igual. Así, candidatos concurrentes en muchos haces se
descartan \citep{KarakoArgaman2015}. Este enfoque fue crucial para identificar y eliminar
eventos espurios famosos como los Perytons en Parkes (señales de microondas locales que
simulaban FRBs; \citet{BurkeSpolaor2011}). Adicionalmente, se aplican máscaras de frecuencia
para excluir canales saturados por RFI persistente y se vigila la banda cero de Fourier (DC spike)
para eliminar impulsos muy anchos.
\end{enumerate}

\subsubsection{Inspección Visual y Confirmación}

Después de estos pasos, lo que queda es una lista de candidatos de FRB astrofísicos potenciales.
Históricamente, estos candidatos se sometían a inspección visual por un astrónomo para confirmar la
morfología dispersada característica en el espectrograma y descartar falsos positivos residuales.
Herramientas interactivas permiten visualizar cada candidato con su dinámica de frecuencia-tiempo, su
curva de S/N vs DM (debe mostrar un pico marcado en la DM óptima) y su perfil temporal dedispersado \citep{Keane2018}. Si la señal exhibe una clara curva $\nu^{-2}$ y no hay indicios de RFI en otros haces o
estaciones, se declara un FRB real.

\subsubsection{Software y Herramientas}

Entre los software de código abierto usados tradicionalmente destacan: HEIMDALL (para
dedispersión rápida en GPU; originalmente desarrollado por B. Barsdell), PRESTO (con el módulo
single\_pulse\_search.py para dedispersión en CPU; \citet{Ransom2001}), FDMT (implementado en la
búsqueda bonsai para CHIME; \citet{Smith2018}) y otros paquetes como ARTS/AMBER (para
Westerbork; \citet{Sclocco2016}) o AstroPulse. Cada colaboración de telescopio suele adaptar y
optimizar su cadena: por ejemplo, ASKAP desarrolló su propio pipeline en GPU llamado FREDDA
\citep{Bannister2017} y MeerKAT emplea el pipeline MeerFAST. Pese a las diferencias, todos
comparten el esquema básico descrito.

\subsubsection{Sensibilidad y Tasa de Falsas Alarmas}

Un aspecto importante es la sensibilidad y tasa de falsas alarmas. Al manejar volúmenes de datos
gigantes (un solo día de CHIME genera millones de candidatos de umbral 6$\sigma$), se debe calibrar
cuidadosamente el umbral para balancear detecciones vs. tasa de falsas detecciones. Un umbral muy
alto ($\geq10\sigma$) podría perder eventos débiles; uno bajo produce demasiados candidatos para inspeccionar.
Además, la instrumentación introduce peculiaridades: tiempos muertos, variaciones de ruido, etc., que
afectan la estadística. Hasta la fecha, se han realizado esfuerzos de comparación entre pipelines con
inyecciones simuladas de FRBs para evaluar su rendimiento \citep{Petroff2018}. No existe aún un
estándar unificado: cada sistema ha ajustado su propio nivel de confianza.

\subsubsection{Limitaciones de los Métodos Tradicionales}

A pesar de estas diferencias, los métodos tradicionales han tenido éxito descubriendo la mayoría de los
$\sim$600 FRBs conocidos hasta 2019 \citep{Petroff2019}. Por ejemplo, Parkes encontró los primeros
20 FRBs usando dedispersión en CPU con PRESTO, mientras ASKAP halló unos 20 en su búsqueda
interferométrica con algoritmos en GPU; CHIME, entre 2018--2019, descubrió 535 FRBs (incluyendo 18
repetidores) en su primer catálogo usando dedispersión rápida (bonsai) y una robusta jerarquía de
filtros en GPU \citep{CHIME2021}. Sin embargo, el factor limitante de estos métodos
tradicionales es la dependencia de la inspección humana y de reglas estáticas para reconocer patrones,
lo cual empieza a ser inviable ante la escala de datos de nuevas encuestas. Aquí es donde intervienen
las técnicas de aprendizaje automático, que han revolucionado la etapa final de clasificación de
candidatos como veremos a continuación.

\subsection{Enfoques de Aprendizaje Automático para la Detección de FRBs}

Los pipelines tradicionales (p.ej. algoritmos de dedispersión y umbralización como PRESTO o
Heimdall) suelen producir miles de candidatos por observación, de los cuales la mayoría son falsos
positivos. Esto impone una carga significativa de análisis manual para diferenciar eventos
astrofísicos reales de RFI o ruido aleatorio. En respuesta, desde mediados de la década de 2010 se han
incorporado técnicas de aprendizaje automático (machine learning) para agilizar y automatizar la
búsqueda de FRBs.

\subsubsection{Primeros Sistemas Asistidos por Machine Learning}

Primeros sistemas asistidos por ML: La primera iniciativa notable fue la de \citet{Wagstaff2016},
quienes implementaron un clasificador automático en el sistema V-FASTR del VLBA. Este modelo de
machine learning categorizaba cada candidato como ``pulso de púlsar conocido'', ``artefacto de RFI'' o
``posible nueva detección''. Gracias a ello, lograron filtrar automáticamente el 80--90\% de los
candidatos con una precisión $>98\%$, reduciendo drásticamente la carga de revisión humana a solo el 10--
20\% de los eventos más promisorios. 

\subsubsection{Clasificadores de Candidatos con Deep Learning}

En años posteriores, \citet{Connor2018} aplicaron
redes neuronales profundas para la clasificación de pulsos detectados en nuevos sondeos de FRBs.
Desarrollaron una arquitectura jerárquica de deep learning que combinaba múltiples representaciones
de los pulsos (p.ej. espectros dinámicos de distintos haces) para estimar la probabilidad de que un
evento sea un FRB real. Entrenado con miles de ejemplos de falsos positivos de radiotelescopios
como CHIME Pathfinder y Apertif, este sistema alcanzó alta exactitud y permitió priorizar en tiempo real
los eventos más probables de ser auténticos transientes. En paralelo, \citet{Agarwal2020}
desarrollaron el clasificador FETCH (Fast Extragalactic Transient Candidate Hunter), basado en aprendizaje
profundo con transfer learning. FETCH emplea redes convolucionales pre-entrenadas para distinguir
imágenes de candidatos FRB frente a RFI, usando como entrada tanto la intensidad vs. tiempo como la
intensidad vs. DM de cada candidato. Con este enfoque lograron tasas de acierto superiores al
99\% en conjuntos de prueba, demostrando que el modelo podía generalizar a distintos telescopios
(detectando consistentemente FRBs simulados inyectados en datos reales de Parkes y ASKAP por
encima de S/N$\sim10$). Tanto el trabajo de \citet{Connor2018} como el de \citet{Agarwal2020} evidenciaron que la clasificación automática de candidatos reduce drásticamente los falsos
positivos a revisar manualmente y acelera la identificación de eventos verdaderos. Sin embargo,
estos enfoques aún dependían de los pipelines clásicos para encontrar los candidatos iniciales -- es decir,
no abordan el problema de la incompletitud en la detección, donde eventos de baja S/N pueden ni
siquiera aparecer entre los candidatos generados por los algoritmos tradicionales.

\subsubsection{Detección Directa con Deep Learning}

Detección directa con deep learning: Reconociendo las limitaciones de los métodos clásicos, algunos
investigadores intentaron usar deep learning para detectar directamente las señales de FRB en los
datos crudos, en lugar de solo clasificar candidatos pre-seleccionados. Por ejemplo, \citet{Zhang2018}
re-analizaron datos del repetidor FRB 121102 mediante una red neuronal convolucional entrenada para
reconocer la huella característica de un FRB. Este algoritmo descubrió 72 nuevas ráfagas en 5 horas de
observación que no habían sido detectadas por las técnicas convencionales, incrementando en un
$\sim30\%$ el número total de pulsos conocidos de dicha fuente. Este resultado pionero demostró el
potencial de la IA para encontrar señales astrofísicas sutiles que escapan a los pipelines clásicos y
motivó nuevos esfuerzos en esta dirección. Recientemente, \citet{Liu2022} propusieron un pipeline
llamado DDSS (Dispersed Dynamic Spectrum Search) que aplica directamente un clasificador profundo
sobre las representaciones tiempo-frecuencia de las observaciones, intentando identificar en ellas las
huellas dispersas de FRBs sin etapa previa de detección clásica. No obstante, \citet{Zhang2018} y \citet{Liu2022} evidenciaron también las dificultades de la detección directa: las ráfagas débiles (de bajo
S/N) pasan inadvertidas al estar muy diluidas en las imágenes, y la variabilidad en la curvatura de la
``parábola de dispersión'' (dependiente del DM de cada evento) dificulta usar entradas de tamaño fijo en
una búsqueda ciega. En otras palabras, si el segmento de datos analizado no abarca
completamente la parábola de un FRB, la red puede no detectarlo; además, aun detectando la
presencia de una señal, un modelo puramente imagen-céntrico típicamente solo indicaría el tiempo de
arribo pero no su DM, requiriendo pasos adicionales para caracterizar el burst. Estas limitaciones
motivaron el desarrollo de enfoques híbridos que combinaran lo mejor de ambos mundos (detección
eficiente y clasificación robusta), dando paso a nuevas arquitecturas de pipelines basados en deep
learning integrales.

\subsubsection{El pipeline DRAFTS: detección con objeto y clasificación binaria}

El esfuerzo más reciente en esta línea es DRAFTS (Deep Learning-based RAdio Fast Transient Search),
presentado por \citet{zhang2024drafts} como un pipeline de nueva generación para detectar FRBs en datos
de radio mediante aprendizaje profundo. DRAFTS propone una arquitectura de dos etapas
especialmente diseñada para resolver las deficiencias de los métodos previos, abordando tanto la
incompletitud de búsqueda como la alta tasa de falsos positivos de las técnicas tradicionales. En primer
lugar, emplea un detector de objetos de visión por computador para localizar directamente las firmas
de FRB dentro de los datos dedispersados. Específicamente, DRAFTS utiliza un modelo anchor-free
basado en CenterNet \citep{Zhou2019} que analiza la matriz de tiempo vs. DM de la observación en
busca de la característica forma de ``lazo'' o bow-tie que deja una ráfaga dispersada. A diferencia
de enfoques anteriores que convertían los datos en imágenes para ingresarlos a la red, aquí el modelo
opera directamente sobre el stream numérico tiempo-DM, lo que ahorra tiempo de I/O en la inferencia
. El detector devuelve las coordenadas centrales del posible evento (es decir, estima
simultáneamente el tiempo de llegada y la medida de dispersión óptima del FRB). A
continuación, en la segunda etapa, DRAFTS extrae la porción correspondiente a la señal candidata en
los datos originales (espectro dinámico dedispersado al DM encontrado) y la alimenta a un clasificador
binario basado en ResNet. Este clasificador de imagen verifica si la señal corresponde a un FRB
real o a ruido/RFI, descartando así las detecciones espurias restantes.

Esta combinación secuencial de detección y clasificación proporciona notables ventajas. Por un lado,
al detectar en el espacio tiempo-DM, DRAFTS soluciona la incompletitud: cualquier FRB presente,
incluso débil, tiende a manifestarse como un pico localizado cuando los datos se dedispersan
correctamente, lo que el detector puede identificar con alta sensibilidad. Por otro lado, la etapa de
clasificación asegura que los falsos positivos se reduzcan drásticamente -- incluso si algún artefacto de
ruido logra disparar el detector, es muy probable que el clasificador ResNet lo etiquete como no
astrofísico, evitando un reporte erróneo. De hecho, el diseño garantiza que la misma ráfaga no sea
detectada múltiples veces y que prácticamente ninguna interferencia supere ambas etapas: si una
señal falsa es marcada en la primera fase, es filtrada en la segunda, haciendo casi innecesaria la
intervención manual humana en la validación. \citet{zhang2024drafts} reportan que este pipeline alcanza
niveles de desempeño muy superiores a los métodos convencionales en todos los aspectos clave:
precisión, exhaustividad y velocidad de detección. Entrenado con un extenso conjunto de $\sim2700$
bursts reales de FAST (incluyendo fuentes repetidoras conocidas) y datos simulados, DRAFTS demostró
en pruebas con datos reales no vistos una tasa de recuperación cercana al 100\% de los FRBs
presentes, manteniendo una tasa de falsos alarmas mínima. Notablemente, en la re-búsqueda
de los datos del repetidor FRB 20190520B (observados con el radiotelescopio FAST), DRAFTS detectó
más del triple de ráfagas comparado con el pipeline tradicional Heimdall en ese mismo conjunto de datos. Este resultado subraya cuánto puede mejorar la sensibilidad efectiva mediante técnicas de
deep learning, recuperando muchos eventos que los algoritmos clásicos pasaron por alto. Al mismo
tiempo, la combinación de detector + clasificador redujo drásticamente los falsos positivos, por lo que
prácticamente no se requieren inspecciones manuales de candidatos, incluso operando en tiempo real
. En suma, DRAFTS representa el estado del arte en pipelines de búsqueda de FRBs: una solución
híbrida que integra visión computacional y clasificación profunda para lograr detecciones más
completas, confiables y rápidas que los enfoques previos, abriendo la vía a escaneos autónomos de
próxima generación en radioastronomía \citep{zhang2024drafts}.

\subsection{Desafíos en la Detección de FRBs a Altas Frecuencias}

La extensión de las búsquedas de FRBs hacia el régimen milimétrico representa una frontera científica
prometedora, como se discutió en la sección 2.5. Sin embargo, junto con las oportunidades vienen desafíos
técnicos y metodológicos significativos que el campo apenas comienza a explorar. Esta sección presenta
el estado actual del conocimiento sobre estos desafíos observacionales e instrumentales, documentando
lo que se ha observado hasta ahora en los primeros experimentos con ALMA y otros telescopios mm.

Si bien la observación de FRBs en bandas de radio más altas ofrece ventajas como discutimos, también
conlleva dificultades únicas que han comenzado a manifestarse:

\subsubsection{Sensibilidad y Espectros de Emisión}

Dado un espectro no trivial de emisión, es posible que los FRBs
sean intrínsecamente menos enérgicos a frecuencias mm que en cm. Muchos púlsares y
magnetars siguen leyes de potencia con $\alpha \approx -1$ a $-3$ (flux $\propto \nu^\alpha$),
lo que significa que sus pulsos a 100 GHz podrían tener intensidades 1--2 órdenes de magnitud
menores que a 1 GHz \citep{Jankowski2017}. Si los FRBs comparten esta tendencia, se requerirá
integrar más tiempo o usar telescopios aún más grandes para detectarlos. Por ahora, ALMA es el
único con sensibilidad suficiente; aun así, un FRB ``típico'' de 1 Jy$\cdot$ms a 1.4 GHz podría presentarse
con solo $\sim0.1$ Jy$\cdot$ms a 90 GHz, al límite de detección en una sola antena de 12 m. Sumado a esto,
no sabemos la tasa de repetición de FRBs en altas frecuencias: podrían repetir con igual
frecuencia pero simplemente ser más débiles (lo cual es manejable), o podría suceder que la
física de emisión limite la radiación coherente a cierto rango de frecuencias (p.ej., corte alto del
espectro), reduciendo drásticamente la ocurrencia de pulsos detectables en mm. Explorar este
aspecto requiere tiempo de integración sustancial apuntando a repetidores conocidos. Si en
decenas de horas ALMA no detectase nada de, por ejemplo, FRB 121102, se podría establecer la
primera restricción de que su espectro cae abruptamente por encima de X GHz.

\subsubsection{Reducción de la Firma Dispersiva}

Como se mencionó, a mayor frecuencia la firma de dispersión
dentro de anchos de banda típicos es menos pronunciada. En un experimento ciego en banda
Ku (12--18 GHz) o W (75--110 GHz), un FRB podría aparecer casi como un pulso de banda ancha
simultáneo (pues el retardo entre sub-bandas es menor que la duración del pulso). Esto dificulta
la distinción de un FRB frente a un pulso RFI impulsivo que afecte todas las frecuencias a la vez.
La dispersión diferencial es lo que hace inconfundible a los FRBs en bandas bajas; sin ella, se
pierde una de las características definitorias. Para solventar esto, se puede: (i) requerir
detecciones paralelas en varios telescopios en diferentes bandas (por ejemplo, una detección
simultánea en ALMA y en un radiotelescopio de 1.4 GHz restablecería la certeza); (ii) apoyarse en
la polarización --la RFI suele ser no polarizada, mientras que un FRB genuino tal vez presente
polarización lineal significativa--; o (iii) utilizar la distribución de energía en DM--tiempo,
buscando ese patrón de bow-tie a microescala, aunque sea en un rango de DM muy pequeño
(esto es retador, pero en principio detectable si la resolución temporal es suficiente para ver que
la S/N maximiza exactamente a la DM correcta y decae a su alrededor). En cualquier caso, la
mitigación de RFI en alta frecuencia requerirá criterios adicionales. Por ejemplo, ALMA se
apoya en su correlación interferométrica: señales de RFI local no correlacionadas entre antenas
se atenúan al formar el haz fasado, y además cada antena dispone de detectores de espectro
RFI. Aun así, pulsos cortos de origen instrumental podrían colarse. La experiencia con el
experimento Phased ALMA demuestra que examinar la polarización y la coincidencia con la
rotación del pulsar (si se trata de un magnetar conocido) fue crucial para validar las detecciones
\citep{veracasanova2025}.

\subsubsection{Morfología de Señales en Banda Milimétrica}

Otra incógnita es cómo lucen los FRBs a escalas temporales
ultra cortas. Si la emisión viene de regiones muy compactas, a menor longitud de onda
podríamos resolver sub-estructuras temporales más finas, incluso del orden de microsegundos.
De hecho, ya se han medido microestructuras de $\sim30$ $\mu$s en repetidores a 1.4 GHz (FRB
121102; \citet{Michilli2018}) y hasta de 3--4 $\mu$s en FRB 180916 a 600 MHz \citep{Nimmo2021}. A
100 GHz, un pulso sin scattering podría revelar variabilidad intrínseca a microsegundos o
nanosegundos si existiese --similar a los giant pulses del púlsar del Cangrejo que llegan a ns--.
Pero capturar esa morfología exige muestreos muy rápidos y gran ancho de banda sin
dispersión. ALMA actualmente alcanza resolución de hasta 31.25 $\mu$s en modo pulsar \citep{Torne2021}; futuras mejoras podrían bajar a 1--10 $\mu$s. Si se descubriera un FRB repetidor brillante en
mm, se podrían buscar estas subestructuras: su presencia o ausencia daría información sobre el
mecanismo de emisión (por ejemplo, pulsos a ns indicarían regiones emisoras extremadamente
pequeñas, del orden de metros, incompatibles con ciertos modelos). Por otro lado, las ráfagas
en mm podrían carecer de algunas complejidades presentes a 600 MHz, como los múltiples subpulsos con drift. Es posible que a frecuencias altas solo se vea la última porción del espectro de
emisión coherente, quizá un único pulso más simple. Esto complicaría usar la morfología como
sello de repetidor o no repetidor.

\subsubsection{Limitaciones Instrumentales}

Finalmente, los retos prácticos de operar en mm no son menores: la
atenuación atmosférica, el ruido térmico más elevado de los receptores a esas frecuencias y la
necesidad de un calibrado y sincronización precisos de múltiples antenas. Todo ello conlleva que
la disponibilidad de tiempo para experimentos FRB en telescopios como ALMA sea limitada,
debiendo competir con otros programas de tiempo. Aun así, la comunidad FRB está invirtiendo
en este frente, pues los posibles descubrimientos (por ejemplo, correlaciones simultáneas con
rayos X, detecciones en óptico de destellos asociados, etc.) serían revolucionarios.

\medskip

La detección de FRBs en altas frecuencias milimétricas se encuentra en sus albores. Los
progresos teóricos y experimentales de la última década --desde establecer la naturaleza extragaláctica
de los FRBs, pasando por identificar repetidores y un magnetar galáctico progenitor, hasta desarrollar
sofisticadas herramientas de detección automáticas-- han sentado las bases para esta nueva fase. El
estado del arte actual nos muestra un campo vibrante: conocemos ya del orden de $10^3$ FRBs,
entendemos mejor sus distribuciones de energía y periodicidad, sospechamos fuertemente de los
magnetars como origen común en muchos casos, y disponemos de tecnología como redes neuronales
y telescopios fasados que nos permiten explorar dominios antes inaccesibles. Los próximos años, con la
consolidación de catálogos masivos (p.ej. CHIME, DSA-2000) y la incursión en frecuencias extremas
(desde 50 MHz hasta 300 GHz), probablemente convertirán a los FRBs en una herramienta cosmológica
y astrofísica madura --usándolos para sondear el medio intergaláctico, explorar física de plasma
relativista e incluso, quizá, para detectar estructuras de materia oscura a través de dispersiones
inusuales--. Cada nueva ventana frecuencial abierta, como la de mm, añade una pieza al rompecabezas
y nos acerca un poco más a desvelar la naturaleza última de estos misteriosos destellos de radio
extragalácticos.


