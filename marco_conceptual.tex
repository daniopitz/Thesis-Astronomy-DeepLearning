\secnumbersection{MARCO CONCEPTUAL}
\setcounter{secnumdepth}{4}
\setcounter{tocdepth}{4}
\makeatletter
\renewcommand\paragraph{\@startsection{paragraph}{4}{\z@}%
  {1.5ex \@plus .5ex \@minus .2ex}%   % espacio antes
  {0.8ex \@plus .2ex}%                 % espacio después
  {\normalfont\normalsize\bfseries}}  % estilo
\makeatother

\subsection{Introducción a los Fast Radio Bursts}

\subsubsection{Descubrimiento y Naturaleza de los FRBs}

Los Fast Radio Bursts (FRBs) son pulsos de radio transitorios de duración milisegundo y extremadamente luminosos, detectados por primera vez en 2007 por Lorimer et al. en datos del radiotelescopio de Parkes \citep{Lorimer2007}. Su medida de dispersión (DM) anómalamente alta reveló un origen extragaláctico a distancias cosmológicas \citep{Thornton2013a}. La firma característica de los FRBs es la dispersión en plasma interestelar: los pulsos de baja frecuencia arriban más tarde que los de alta frecuencia siguiendo $t(\nu)\propto \nu^{-2}$, generando una curvatura distintiva en los espectrogramas frecuencia-tiempo (Figura~\ref{fig:lorimer_burst}).

\begin{figure}[htbp]
\centering
\includegraphics[width=0.8\textwidth]{Marco_conceptual/20140512_Lorimer-Burst.png}
\caption{Espectrograma del ``Lorimer Burst'' (FRB 010724), primer FRB reportado. Se observa el barrido dispersivo característico con dependencia $\sim\nu^{-2}$: frecuencias altas ($\sim$1.5 GHz) arriban antes que bajas ($\sim$1.2 GHz). Panel insertado: pulso dedispersado de $\sim$5 ms. Adaptado de \citet{Lorimer2007}.}
\label{fig:lorimer_burst}
\end{figure}

\subsubsection{Propiedades Energéticas y Poblaciones}

Establecida la firma dispersiva característica de los FRBs, su naturaleza física plantea interrogantes fundamentales sobre los progenitores. Los FRBs se originan a distancias cosmológicas con energías isotrópicas de $10^{38}$--$10^{40}$ J concentradas en milisegundos \citep{Petroff2019,Bochenek2020}, implicando fuentes astrofísicas extremas. Se han catalogado cientos de eventos, incluyendo repetidores (múltiples ráfagas de la misma fuente) y aparentes eventos únicos. Los repetidores descartaron modelos cataclísmicos y favorecen objetos persistentes como estrellas de neutrones altamente magnetizadas (magnetars) \citep{Spitler2016,Margalit2018}.

\subsubsection{Motivación para Observaciones en Alta Frecuencia}

Comprendiendo ya la naturaleza energética de los FRBs y la dicotomía repetidores/únicos, surge naturalmente la pregunta sobre estrategias observacionales óptimas. Históricamente, los FRBs se han observado en bandas centimétricas (0.1--10 GHz) donde exhiben mayor luminosidad pero sufren fuertemente dispersión y scattering ($\propto \nu^{-4}$). Las observaciones milimétricas (30--300 GHz) mitigan estos efectos, permitiendo detectar pulsos intrínsecamente breves, aunque presentan desafíos de sensibilidad y campos de visión reducidos. Las secciones siguientes abordan: fundamentos físicos de los FRBs, casos emblemáticos de repetidores, conexión con magnetars (incluyendo SGR 1935+2154), detección en alta frecuencia, métodos tradicionales de búsqueda, y soluciones modernas de aprendizaje automático.

\subsection{Fundamentos Físicos y Fenomenología Observacional de los FRBs}

Para comprender los desafíos de detección que motivan esta tesis, es esencial caracterizar en detalle la fenomenología observacional de los FRBs. Las siguientes subsecciones describen las propiedades físicas clave que determinan tanto las estrategias de búsqueda tradicionales como las adaptaciones necesarias para alta frecuencia.

\subsubsection{Dispersión en Plasma y Medida de DM}

Un FRB típico se manifiesta como un pulso único de radio de duración brevísima ($\sim$1--10 ms) que sobresale muy significativamente por encima del ruido de fondo. En el dominio tiempo-frecuencia, la señal presenta la firma de la dispersión en plasma: las componentes de menor frecuencia arriban con mayor retraso temporal respecto a las de mayor frecuencia, según la ley $\Delta t \propto \text{DM} \cdot \nu^{-2}$. Esta ley implica que, cuando se observa en un ancho de banda amplio, el pulso aparece como un trazo inclinado o curvo en el diagrama de frecuencia vs. tiempo, donde las bajas frecuencias se retrasan más \citep{CordesMcLaughlin2003}. Al corregir este retraso mediante un proceso de dedispersión para un cierto valor de DM, las componentes de frecuencia se alinean en el tiempo revelando el pulso comprimido.


Más allá de la fenomenología cualitativa, la DM proporciona información cuantitativa crucial. La medida de dispersión (DM) es una magnitud central en el estudio de FRBs. Se define como la integral de la densidad de electrones libres $n_e$ a lo largo de la línea de visión hasta la fuente: ${\rm DM} = \int_0^D n_e\, dl$, típicamente expresada en unidades de pc cm$^{-3}$. Dado que las señales de radio sufren un retraso relativo $\Delta t \approx 4.15\,{\rm ms} \times ({\rm DM}/{\rm pc\,cm^{-3}}) \times[(\nu_{\rm baja}/{\rm GHz})^{-2} - (\nu_{\rm alta}/{\rm GHz})^{-2}]$ \citep{LorimerKramer2004}, una DM anómalamente alta es evidencia de un origen extragaláctico o cosmológico. Los FRBs descubiertos hasta ahora exhiben DM de cientos hasta miles de pc cm$^{-3}$, excediendo por mucho el contenido electrónico de la Vía Láctea en sus direcciones \citep{Petroff2019}. Por ejemplo, el FRB inicial de Lorimer tenía DM $\sim375$ pc cm$^{-3}$, varias veces mayor que la contribución galáctica esperada en su coordenada, implicando una distancia de orden gigapársec si el exceso de DM proviene del medio intergaláctico \citep{Lorimer2007}. La DM se utiliza entonces como un indicador de distancia cosmológica: aunque existe dispersión en la relación DM--$z$ por la distribución inhomogénea del medio intergaláctico, a primera aproximación DMs del orden $10^3$ pc cm$^{-3}$ sugieren orígenes a redshifts $z\sim0.5$--1 \citep{Inoue2004,Tendulkar2017}.

\subsubsection{Scattering y Ensanchamiento Temporal}

Además de la dispersión, los FRBs exhiben ensanchamiento temporal por scattering multipath en plasma turbulento, más pronunciado a bajas frecuencias ($\tau_{\rm sc} \propto \nu^{-4}$) \citep{Cordes2016}. A frecuencias altas, los pulsos son significativamente más breves: FRB 121102 mostró anchos de $\sim30$--100 $\mu$s a 4--8 GHz versus varios milisegundos a 1.4 GHz \citep{Gajjar2018a,Hessels2019,Nimmo2021}, motivando observaciones en bandas milimétricas para captar morfologías intrínsecas.

\subsubsection{Polarización y Rotación Faraday}

Complementando las firmas de dispersión y scattering, la polarización proporciona pistas cruciales sobre los entornos de los progenitores. Muchos FRBs exhiben alta polarización lineal ($\sim100\%$) y medidas de rotación Faraday (RM) extremas, indicando campos magnéticos intensos en entornos densos. FRB 121102 registró RM $\sim1.4\times10^5$ rad m$^{-2}$, el mayor para una fuente extragaláctica \citep{Michilli2018}, sugiriendo proximidad a magnetars o restos de supernova. Esta firma polarimétrica apoya la conexión con objetos de campos ultraintensos y proporciona un criterio de validación crucial para detectores automáticos en alta frecuencia.

\subsubsection{Morfología Espectro-Temporal y Sub-pulsos}

Los FRBs exhiben morfología espectro-temporal diversa: desde pulsos simples de banda ancha hasta complejos sub-pulsos con deriva en frecuencia (``efecto trombón triste''), donde componentes sucesivas aparecen a frecuencias decrecientes \citep{Hessels2019,Pleunis2021}. La Figura~\ref{fig:morfologias_frb} ilustra cuatro arquetipos morfológicos generales identificados por CHIME/FRB, relevantes para el diseño de detectores automáticos que deben generalizar sobre esta variabilidad. El drift es particularmente característico de repetidores como FRB 121102 \citep{Gajjar2018b}, mientras que eventos aparentemente únicos tienden a mostrar espectros más estrechos.

\begin{figure}[htbp]
\centering
\includegraphics[width=0.92\textwidth]{Marco_conceptual/morfologias.jpg}
\caption{Cuatro arquetipos morfológicos de FRBs (CHIME/FRB, 400--800 MHz): (1) banda ancha simple, (2) banda estrecha simple, (3) temporalmente compleja, y (4) deriva descendente (``efecto trombón triste''). Cada panel muestra espectro dinámico (abajo) y perfil temporal (arriba). Crédito: \citet{Pleunis2021}.}
\label{fig:morfologias_frb}
\end{figure}

Para ilustrar concretamente esta diversidad morfológica, un ejemplo detallado se observa en FRB 121102, cuyas múltiples ráfagas revelan patrones intrincados de deriva en frecuencia y sub-estructuras temporales, como se ilustra en la Figura~\ref{fig:frb121102_waterfall}.

\begin{figure}[htbp]
\centering
\includegraphics[width=0.95\textwidth]{Marco_conceptual/waterfall-FRB121102.jpg}
\caption{Espectros dinámicos de 18 ráfagas de FRB 121102 (DM = 560.5 pc cm$^{-3}$) en tres campañas: Arecibo (1.2--1.7 GHz), Green Bank (1.7--2.3 GHz), y GB-BL (4--8 GHz). Se observa deriva descendente característica (``efecto trombón triste'') y variabilidad morfológica entre ráfagas. A frecuencias más altas (4--8 GHz), los pulsos son más breves ($\sim$30--100 $\mu$s), demostrando reducción del scattering. Adaptado de \citet{Hessels2019}.}
\label{fig:frb121102_waterfall}
\end{figure}

\subsubsection{Incompletitud en Detección y Ráfagas Atípicas}

Esta diversidad morfológica plantea un desafío crítico para la detección: los métodos tradicionales pueden perder ráfagas atípicas. Pulsos estrechos en banda ($\sim$200 MHz) o morfologías complejas escapan a algoritmos estándar, subestimando tasas de detección \citep{Gourdji2019}. Esta incompletitud motiva enfoques de aprendizaje profundo para detección directa, tema central de esta tesis. La fenomenología observacional descrita --diversidad morfológica, dispersión, polarización-- proporciona pistas sobre progenitores y desafía el diseño de detectores automáticos robustos.

\subsection{Casos Históricos y FRBs Repetidores Notables}

Habiendo establecido las propiedades físicas generales de los FRBs, examinamos ahora casos emblemáticos que han transformado nuestra comprensión del fenómeno y que sirven como casos de validación para pipelines automáticos de detección.

\subsubsection{FRB 121102: El Primer Repetidor}

El descubrimiento de FRB 121102 como primer repetidor \citep{Spitler2016} demostró que al menos algunos FRBs provienen de objetos persistentes y no de eventos catastróficos destructivos. Localizado en una galaxia enana a $z\approx0.19$ con región HII de formación estelar \citep{Chatterjee2017,Tendulkar2017}, se ha convertido en el FRB más estudiado con cientos de pulsos detectados entre 0.6--8 GHz. Sus características extraordinarias incluyen: altísima RM ($\sim1.4\times10^5$ rad m$^{-2}$), morfología compleja con deriva de sub-pulsos, y pulsos progresivamente más breves a altas frecuencias ($\sim30$--100 $\mu$s a 4--8 GHz) \citep{Gajjar2018a,Hessels2019}. FRB 121102 es un caso de validación crítico para esta tesis (Componente 1).

\subsubsection{FRB 180916 y Periodicidad}

Más allá de las ráfagas múltiples, algunos repetidores exhiben periodicidad en su actividad. FRB 180916 (CHIME/FRB) reveló periodicidad de $\sim16.35$ días, la primera evidencia de ciclos de actividad en FRBs \citep{CHIME_FRB_Collaboration_2020}. Indicios tentativos de periodicidad $\sim161$ días también se reportaron para FRB 121102 \citep{Rajwade2020,cruces2020frb121102}. Estas modulaciones temporales sugieren sistemas binarios o precesión de magnetars, y son relevantes para estrategias de observación dirigida.

\subsubsection{Propiedades Distintivas de Repetidores}

Generalizando más allá de estos casos individuales, se han confirmado decenas de repetidores, aunque $\approx95\%$ de FRBs descubiertos permanecen como eventos únicos \citep{Petroff_2022}. Los repetidores tienden a exhibir: múltiples sub-pulsos con drift, espectros más estrechos, y variaciones temporales en DM/RM indicando evolución del plasma circundante \citep{Fonseca2020,Pleunis2021,cruces2020frb121102}. Estas propiedades apuntan a progenitores dinámicos --posiblemente magnetars jóvenes-- cuya identificación se aborda en la siguiente sección.

\subsection{Magnetars y su Vínculo con los FRBs}

\subsubsection{Magnetars como Progenitores: Evidencia de SGR 1935+2154}

Los magnetars son estrellas de neutrones con campos magnéticos extraordinariamente intensos ($\geq10^{14}$ G) capaces de liberar energía magnética en destellos de rayos X/gamma y emisiones de radio coherentes. Diversas evidencias sostienen su conexión con FRBs: (i) requerimientos energéticos compatibles, (ii) localización en regiones de formación estelar sugiriendo objetos jóvenes \citep{Tendulkar2017,Bochenek2020}, y (iii) altas RMs indicando entornos magnetizados extremos \citep{Michilli2018,PopovPostnov2010,Lyutikov2017}.

La prueba definitiva llegó el 28 de abril de 2020: el magnetar galáctico SGR 1935+2154 emitió un pulso tipo-FRB de fluencia $\sim1.5 \times 10^6$ Jy$\cdot$ms, detectado por STARE2 y CHIME/FRB \citep{Bochenek2020,CHIME_SGR2020}. Con energía $\sim10^{34}$ J (distancia 9 kpc), este evento (FRB 200428) se ubica entre ráfagas típicas de magnetar y FRBs cosmológicos, estableciendo continuidad física entre ambos. Crucialmente, el pulso de radio coincidió con un estallido de rayos X simultáneo (Figura~\ref{fig:sgr1935_radio_xray}), confirmando que reconexión magnetosférica puede generar emisión multi-longitud de onda \citep{Mereghetti2020,Li2021}.

\begin{figure}[htbp]
\centering
\includegraphics[width=0.85\textwidth]{Marco_conceptual/FRB-magnetar.png}
\caption{Coincidencia radio--rayos X del evento FRB 200428 de SGR 1935+2154 (28 abril 2020). Perfil temporal de rayos X (curva azul, INTEGRAL) con dos picos principales. Líneas rojas verticales: tiempos de arribo de pulsos de radio (CHIME/FRB, 400--800 MHz). Esta correlación temporal multi-longitud de onda estableció definitivamente el vínculo magnetar-FRB. Adaptado de \citet{Mereghetti2020,CHIME_SGR2020,Bochenek2020}.}
\label{fig:sgr1935_radio_xray}
\end{figure}

\subsubsection{Implicaciones y Emisión en Alta Frecuencia}

El evento de SGR 1935+2154 tiene implicaciones profundas tanto para la comprensión de progenitores como para estrategias observacionales. Este descubrimiento establece que magnetars jóvenes ($\sim10^2$--$10^4$ años) pueden explicar al menos parte de los FRBs: una erupción magnetar vista desde otra galaxia aparecería como FRB típico \citep{Bochenek2020}. FRB 200428 presentó morfología similar a repetidores (dos picos separados $\sim30$ ms con drifts espectrales) \citep{Zhang2020}, reforzando la analogía.

Crucialmente para esta tesis, los magnetars emiten en bandas milimétricas. \citet{veracasanova2025} detectaron 8 pulsos del magnetar galáctico PSR J1745--2900 con ALMA a $\sim86$ GHz (Banda 3), la primera observación de ráfagas de magnetar en régimen mm. Las energías ($\sim10^{29}$ erg) siguieron ley de potencia con exponente $-2.4$, consistente con magnetars en cm y FRBs repetitivos, sugiriendo continuidad de mecanismos de emisión a través de décadas de frecuencia. Este resultado valida la viabilidad de observar FRBs extragalácticos en bandas milimétricas (Componente 2 de esta tesis).

Las ventajas de observar FRBs en mm incluyen: (i) scattering reducido ($\propto \nu^{-4}$) captando morfologías intrínsecas, y (ii) penetración en entornos densos opacos a bajas frecuencias \citep{Yang2020,Omand2022}. Los desafíos incluyen espectros típicamente decrecientes (índice $-1$ a $-3$) y campos de visión limitados. ALMA, operando en modo fasado (phased array) con diámetro efectivo $\sim84$ m y resolución temporal $\sim100$ $\mu$s, es el instrumento más sensible para búsquedas en bandas 3--6 (85--150 GHz). Los estudios piloto sugieren viabilidad de detectar decenas de púlsares y repetidores FRB durante fases activas \citep{veracasanova2025}. En suma, los magnetars proporcionan el nexo entre fenómenos locales y cosmológicos, motivando la extensión de detectores automáticos a alta frecuencia abordada en esta tesis.

\subsection{Observación de FRBs en Alta Frecuencia: Oportunidades y Desafíos}

La evidencia de emisión de magnetars en banda milimétrica motiva examinar las oportunidades y desafíos de extender las búsquedas de FRBs hacia el régimen milimétrico (30--300 GHz). Históricamente, los FRBs se han detectado en bandas centimétricas (100 MHz - 10 GHz) donde la emisión es intensa pero sufre fuertemente efectos de propagación. Esta sección integra las ventajas físicas de alta frecuencia, los desafíos algorítmicos que motivan el Componente 2 de esta tesis, y el estado actual del instrumental disponible.

\subsubsection{Ventajas Físicas: Mitigación de Efectos de Propagación}

Los retardos por dispersión ($\Delta t \propto \nu^{-2}$) y scattering ($\tau_{\rm sc} \propto \nu^{-4}$) decrecen drásticamente con la frecuencia. A frecuencias mm, un pulso sufre ensanchamiento miles de veces menor que a 1 GHz, permitiendo captar morfologías intrínsecas. En ALMA banda 3 (86 GHz, ancho 2 GHz), la demora por DM $\sim1770$ pc cm$^{-3}$ es apenas $\sim46$ $\mu$s, posibilitando integración directa en frecuencia sin dedispersión \citep{veracasanova2025}.

Adicionalmente, altas frecuencias penetran entornos densos (restos de supernova, vientos nebulares) opacos a bajas frecuencias, abriendo la posibilidad de detectar FRBs ``oscuros'' invisibles para CHIME o Parkes \citep{Yang2020,Omand2022,Hilmarsson2022}.

\subsubsection{Desafíos: Sensibilidad, Firma Dispersiva y Morfología}

Sin embargo, las ventajas vienen acompañadas de desafíos significativos. Primero, los espectros típicamente decrecientes (índice $\alpha \approx -1$ a $-3$, flux $\propto \nu^\alpha$) implican que pulsos a 100 GHz podrían ser 1--2 órdenes de magnitud más débiles que a 1 GHz \citep{Jankowski2017}. Un FRB de 1 Jy$\cdot$ms a 1.4 GHz tendría solo $\sim0.1$ Jy$\cdot$ms a 90 GHz.

Segundo, y crítico para algoritmos de detección, la firma dispersiva se comprime drásticamente: la curva $\nu^{-2}$ se vuelve casi imperceptible, eliminando el indicador clave para distinguir FRBs de RFI. Para FRBs extragalácticos sin DM conocido, detectar en mm sin firma dispersiva clara requiere estrategias alternativas: (i) validación por polarización (FRBs muestran $>50\%$ polarización vs. RFI no polarizada \citep{veracasanova2025}), (ii) coherencia multi-antena en arrays interferométricos, o (iii) detecciones simultáneas multi-frecuencia.

Tercero, se desconoce la morfología temporal en mm. Pulsos podrían revelar microestructuras de $\mu$s-ns (ya medidas 3--4 $\mu$s en FRB 180916 a 600 MHz \citep{Nimmo2021}) o, inversamente, carecer de complejidades presentes a bajas frecuencias (drift, sub-pulsos múltiples).

\subsubsection{Instrumental: ALMA y Capacidades Actuales}

ALMA, operando en modo fasado (diámetro efectivo $\sim84$ m, resolución $\sim100$ $\mu$s), es el instrumento más sensible en 3 mm, capaz de detectar pulsos de $\sim1$ Jy en decenas de ms. Sin embargo, campos de visión pequeños (pocos arcmin) hacen imprácticos barridos ciegos. La estrategia es apuntar a repetidores conocidos durante fases activas (ej. FRB 121102 en intervalos ``on'' $\sim160$ días, FRB 180916 cada 16 días).

\subsubsection{Resultados Iniciales con ALMA}

La detección de 8 pulsos del magnetar galáctico PSR J1745--2900 con ALMA banda 3 prueba la eficacia del instrumento y existencia de emisión pulsada en mm \citep{veracasanova2025}. Extrapolando, se estima que $>160$ púlsares galácticos serían visibles por ALMA. Para FRBs extragalácticos, asumiendo índices espectrales típicos ($\alpha \sim -1.5$ \citep{Jankowski2017,Macquart2019}), una ráfaga de 1 Jy$\cdot$ms a 1.4 GHz tendría $\sim0.1$ Jy$\cdot$ms a 86 GHz, marginalmente detectable. Aunque ningún FRB extragaláctico ha sido reportado en mm, programas de búsqueda están en marcha priorizando repetidores durante ventanas de actividad.

\subsubsection{Limitaciones Instrumentales y Criterios de Validación}

Las limitaciones prácticas incluyen: atenuación atmosférica, ruido térmico elevado, calibrado preciso multi-antena, y disponibilidad limitada de tiempo de telescopio. La experiencia con ALMA demuestra que validación requiere criterios alternativos: polarización (todos los pulsos del magnetar mostraron $>50\%$ polarización, claramente diferenciados de RFI \citep{veracasanova2025}), coherencia multi-antena, o detecciones simultáneas multi-frecuencia (ej. CHIME a 600 MHz + ALMA a 100 GHz con misma DM como evidencia incontrovertible).

\medskip

En suma, las observaciones en mm representan una frontera emergente prometedora: podrían revelar FRBs escondidos, proporcionar pulsos sin distorsión por plasma, y aportar restricciones severas a modelos de emisión. ALMA se perfila como pionero, ofreciendo la posibilidad de explorar esta ventana frecuencial inexplorada. Los desafíos identificados --compresión de firma dispersiva, sensibilidad reducida, morfología desconocida-- motivan las estrategias metodológicas adaptadas del Componente 2 de esta tesis.

\subsection{Métodos Tradicionales de Búsqueda y Detección de FRBs}

Para contextualizar las soluciones basadas en aprendizaje profundo que constituyen el núcleo de esta tesis, es esencial comprender primero los métodos tradicionales de detección que han dominado el campo y cuyas limitaciones motivan el desarrollo de nuevos enfoques.

\subsubsection{Pipeline Clásico de Detección}

La detección de FRBs en vastos volúmenes de datos de radio requiere métodos eficientes para extraer
señales cortas y dispersas entre ruido e interferencia. Los métodos tradicionales se basan en aplicar la
corrección de dispersión a los datos de forma exhaustiva, seguida de filtrados y umbrales para
identificar candidatos. En la práctica, un radiotelescopio registra voltajes o potencias en muchas subbandas de frecuencia con alta resolución temporal (usualmente milisegundos o menor). Un pipeline de búsqueda de FRBs típicamente ejecuta los siguientes pasos fundamentales \citep{Petroff2015,KeanePetroff2015}:

\begin{enumerate}
\item \textbf{Dedispersión por ensayo de DM:} Dado que la DM de una posible ráfaga es desconocida de
antemano, se genera un banco de triales de DM que cubren desde 0 (ninguna dispersión) hasta
un máximo elevado (p.ej. 5000 pc cm$^{-3}$ para búsquedas a escala cosmológica). Para cada
valor de DM, se aplican retrasos apropiados a las series de cada canal de frecuencia (o subbanda) para alinearlas en el tiempo, integrando en frecuencia para obtener una serie temporal
dedispersada. Este proceso puede hacerse de forma directa (brute-force) o mediante algoritmos
optimizados como la Transformada Dispersiva Rápida (FDMT; \citet{ZackayOfek2017}) que
explotan transformadas acumulativas. Herramientas populares como HEIMDALL \citep{Champion2016} realizan esta dedispersión de manera paralela en GPUs, permitiendo explorar miles de
DM en tiempo real. Alternativamente, PRESTO \citep{Ransom2001} implementa dedispersión por subbandas de forma eficiente en CPU.

\item \textbf{Filtrado por anchura de pulso:} Una vez dedispersados los datos, el siguiente paso optimiza la detección para pulsos de diferentes anchos. Los pulsos pueden sufrir dispersión intrínseca o scattering,
haciéndolos más anchos que la resolución nativa. Para maximizar la detección, se aplica una
familia de filtros de suavizado (convolución con ventanas) a cada serie temporal dedispersada
para distintas longitudes (p. ej., 1, 2, 4, 8, ..., 128 ms). Esto simula la integración de la señal en
diferentes anchos de pulso, aumentando la S/N de pulsos más anchos a costa de agregar ruido
correlacionado. Tras este paso, para cada DM se obtiene un conjunto de series filtradas.

\item \textbf{Detección de umbral:} Se recorre cada serie dedispersada y filtrada en busca de excedencias
sobre un umbral de S/N predefinido (típicamente 6$\sigma$--8$\sigma$). Cada vez que una muestra supera el
umbral, se registra un candidato con sus parámetros: DM, tiempo de llegada, S/N y anchura
(correspondiente al filtro óptimo).

\item \textbf{Agrupamiento y coincidencias:} La etapa de umbralización genera numerosas detecciones redundantes que deben consolidarse. Dado que un pulso real suele generar múltiples detecciones
cercanas (por ejemplo, en DMs adyacentes o en anchos parecidos), los candidatos brutos se
agrupan para evitar duplicados. Se agrupan por proximidad en tiempo y DM utilizando
algoritmos de friends-of-friends o kNN en el espacio (t, DM) \citep{BurkeSpolaor2011}. Así,
múltiples detecciones de un mismo evento colapsan a un único candidato representativo con la
DM y anchura que dio máxima S/N.

\item \textbf{Rechazo de interferencia (RFI):} A pesar del agrupamiento, la mayoría de candidatos restantes son aún artefactos instrumentales o terrestres. En paralelo, se realizan filtros para descartar señales originadas
por RFI terrestre. Un criterio común es eliminar detecciones con DM cerca de 0 pc cm$^{-3}$, ya
que la RFI no astrofísica no presenta dispersión. También se aprovechan receptores multihaz: si
el telescopio tiene varios haces apuntados adyacentes (como Parkes 13-beams, ASKAP 36-
beams, outrigger arrays), una señal astrofísica puntual solo aparecerá en un haz, mientras que la
RFI tiende a ser vista en varios por igual. Así, candidatos concurrentes en muchos haces se
descartan \citep{KarakoArgaman2015}. Este enfoque fue crucial para identificar y eliminar
eventos espurios famosos como los Perytons en Parkes (señales de microondas locales que
simulaban FRBs; \citet{BurkeSpolaor2011}). Adicionalmente, se aplican máscaras de frecuencia
para excluir canales saturados por RFI persistente y se vigila la banda cero de Fourier (DC spike)
para eliminar impulsos muy anchos.
\end{enumerate}

Estos cinco pasos fundamentales constituyen el núcleo de prácticamente todos los sistemas de búsqueda de FRBs operativos hasta la fecha. La Figura~\ref{fig:traditional_pipeline} presenta un esquema detallado de este pipeline tradicional, mostrando el flujo completo desde los datos crudos hasta la lista final de candidatos.

\begin{figure}[htbp]
\centering
\includegraphics[width=0.95\textwidth]{Marco_conceptual/PipelineClassic.png}
\caption{Pipeline tradicional de detección de FRBs. Flujo secuencial: (1) preprocesamiento/RFI, (2) dedispersión sobre parrilla DM (0--5000 pc cm$^{-3}$), (3) búsqueda de pulsos (filtros boxcar), (4) umbralización (S/N$>$6--8$\sigma$), y (5) clasificación/validación. Implementaciones modernas incorporan ML (ej. FETCH) en la etapa final. Adaptado de \citet{Wang2025CRACO}.}
\label{fig:traditional_pipeline}
\end{figure}

\subsubsection{Inspección Visual y Confirmación}

Tras el pipeline automático, el paso final tradicional era la validación humana. Históricamente, candidatos requerían inspección visual para confirmar morfología dispersada ($\nu^{-2}$) y descartar RFI residual \citep{Keane2018}. Este cuello de botella humano motiva la automatización mediante aprendizaje profundo abordada en las siguientes secciones.

\subsubsection{Software y Herramientas}

Antes de examinar las limitaciones globales, es útil conocer las implementaciones concretas existentes. El pipeline descrito ha sido implementado en diversos paquetes de software optimizados para diferentes instrumentos. Software tradicional incluye HEIMDALL (GPU, dedispersión rápida), PRESTO (CPU, \citet{Ransom2001}), FDMT/bonsai (CHIME, \citet{Smith2018}), FREDDA (ASKAP, \citet{Bannister2017}), y MeerFAST (MeerKAT). Todos comparten el esquema básico: dedispersión exhaustiva, filtrado adaptado, y umbralización, con variantes de implementación específicas por instrumento.

\subsubsection{Limitaciones de los Métodos Tradicionales}

Pese a la diversidad de implementaciones, todos los pipelines tradicionales comparten limitaciones estructurales fundamentales. Los métodos tradicionales descubrieron $\sim$600 FRBs hasta 2019 (Parkes, ASKAP, CHIME con 535 en su primer catálogo) \citep{Petroff2019,CHIME2021}. Sin embargo, enfrentan limitaciones críticas: (i) calibración de umbral S/N requiere balance entre sensibilidad y tasa de falsas alarmas (CHIME genera millones de candidatos 6$\sigma$ diarios), (ii) incompletitud ante morfologías atípicas (pulsos estrechos en banda, estructuras complejas), y (iii) dependencia de inspección humana inviable ante escalas de datos actuales. Estas limitaciones motivan técnicas de aprendizaje automático que revolucionan la clasificación y detección directa de candidatos.

\subsection{Enfoques de Aprendizaje Automático para la Detección de FRBs}

Las limitaciones de los métodos tradicionales --en particular, la alta tasa de falsos positivos y la incompletitud ante morfologías atípicas-- han impulsado la adopción progresiva de técnicas de aprendizaje automático que automatizan y mejoran la detección. Esta sección traza la evolución desde sistemas asistidos por ML hasta pipelines integrales como DRAFTS, fundamento del trabajo desarrollado en esta tesis.

\subsubsection{Primeros Sistemas Asistidos por Machine Learning}

Primeros sistemas asistidos por ML: La primera iniciativa notable fue la de \citet{Wagstaff2016},
quienes implementaron un clasificador automático en el sistema V-FASTR del VLBA. Este modelo de
machine learning categorizaba cada candidato como ``pulso de púlsar conocido'', ``artefacto de RFI'' o
``posible nueva detección''. Gracias a ello, lograron filtrar automáticamente el 80--90\% de los
candidatos con una precisión $>98\%$, reduciendo drásticamente la carga de revisión humana a solo el 10--
20\% de los eventos más promisorios. 

\subsubsection{Clasificadores de Candidatos con Deep Learning}

Construyendo sobre estos primeros éxitos, la comunidad comenzó a explorar arquitecturas de aprendizaje profundo más sofisticadas. En años posteriores, \citet{Connor2018} aplicaron
redes neuronales profundas para la clasificación de pulsos detectados en nuevos sondeos de FRBs.
Desarrollaron una arquitectura jerárquica de deep learning que combinaba múltiples representaciones
de los pulsos (p.ej. espectros dinámicos de distintos haces) para estimar la probabilidad de que un
evento sea un FRB real. Entrenado con miles de ejemplos de falsos positivos de radiotelescopios
como CHIME Pathfinder y Apertif, este sistema alcanzó alta exactitud y permitió priorizar en tiempo real
los eventos más probables de ser auténticos transientes. Simultáneamente y con enfoque complementario, \citet{Agarwal2020}
desarrollaron el clasificador FETCH (Fast Extragalactic Transient Candidate Hunter), basado en aprendizaje
profundo con transfer learning. FETCH emplea redes convolucionales pre-entrenadas para distinguir
imágenes de candidatos FRB frente a RFI, usando como entrada tanto la intensidad vs. tiempo como la
intensidad vs. DM de cada candidato. Con este enfoque lograron tasas de acierto superiores al
99\% en conjuntos de prueba, demostrando que el modelo podía generalizar a distintos telescopios
(detectando consistentemente FRBs simulados inyectados en datos reales de Parkes y ASKAP por
encima de S/N$\sim10$). Tanto el trabajo de \citet{Connor2018} como el de \citet{Agarwal2020} evidenciaron que la clasificación automática de candidatos reduce drásticamente los falsos
positivos a revisar manualmente y acelera la identificación de eventos verdaderos. Sin embargo,
estos enfoques aún dependían de los pipelines clásicos para encontrar los candidatos iniciales -- es decir,
no abordan el problema de la incompletitud en la detección, donde eventos de baja S/N pueden ni
siquiera aparecer entre los candidatos generados por los algoritmos tradicionales.

\subsubsection{Detección Directa con Deep Learning}

Si bien los clasificadores mejoraron drásticamente la precisión, aún dependían de pipelines tradicionales para generar candidatos iniciales, heredando así su incompletitud. Reconociendo esta limitación, algunos investigadores intentaron usar deep learning para detectar directamente las señales de FRB en los
datos crudos, en lugar de solo clasificar candidatos pre-seleccionados. Por ejemplo, \citet{Zhang2018}
re-analizaron datos del repetidor FRB 121102 mediante una red neuronal convolucional entrenada para
reconocer la huella característica de un FRB. Este algoritmo descubrió 72 nuevas ráfagas en 5 horas de
observación que no habían sido detectadas por las técnicas convencionales, incrementando en un
$\sim30\%$ el número total de pulsos conocidos de dicha fuente. Este resultado pionero demostró el
potencial de la IA para encontrar señales astrofísicas sutiles que escapan a los pipelines clásicos y
motivó nuevos esfuerzos en esta dirección. Construyendo sobre este éxito, \citet{Liu2022} propusieron recientemente un pipeline
llamado DDSS (Dispersed Dynamic Spectrum Search) que aplica directamente un clasificador profundo
sobre las representaciones tiempo-frecuencia de las observaciones, intentando identificar en ellas las
huellas dispersas de FRBs sin etapa previa de detección clásica. No obstante, \citet{Zhang2018} y \citet{Liu2022} evidenciaron también las dificultades de la detección directa: las ráfagas débiles (de bajo
S/N) pasan inadvertidas al estar muy diluidas en las imágenes, y la variabilidad en la curvatura de la
``parábola de dispersión'' (dependiente del DM de cada evento) dificulta usar entradas de tamaño fijo en
una búsqueda ciega. En otras palabras, si el segmento de datos analizado no abarca
completamente la parábola de un FRB, la red puede no detectarlo; además, aun detectando la
presencia de una señal, un modelo puramente imagen-céntrico típicamente solo indicaría el tiempo de
arribo pero no su DM, requiriendo pasos adicionales para caracterizar el burst. Estas limitaciones
motivaron el desarrollo de enfoques híbridos que combinaran lo mejor de ambos mundos (detección
eficiente y clasificación robusta), dando paso a nuevas arquitecturas de pipelines basados en deep
learning integrales.

\subsubsection{El pipeline DRAFTS: detección con objeto y clasificación binaria}

Las experiencias con detección directa demostraron tanto el potencial como las limitaciones de enfoques puramente imagen-céntricos. Esto motivó el desarrollo de arquitecturas híbridas que combinan lo mejor de ambos mundos: detección eficiente en espacio comprimido y clasificación robusta. El esfuerzo más reciente en esta línea es DRAFTS (Deep Learning-based RAdio Fast Transient Search),
presentado por \citet{zhang2024drafts} como un pipeline de nueva generación para detectar FRBs en datos
de radio mediante aprendizaje profundo. DRAFTS propone una arquitectura de dos etapas
especialmente diseñada para resolver las deficiencias de los métodos previos, abordando tanto la
incompletitud de búsqueda como la alta tasa de falsos positivos de las técnicas tradicionales. En primer
lugar, emplea un detector de objetos de visión por computador para localizar directamente las firmas
de FRB dentro de los datos dedispersados. Específicamente, DRAFTS utiliza un modelo anchor-free
basado en CenterNet \citep{Zhou2019} que analiza la matriz de tiempo vs. DM de la observación en
busca de la característica forma de ``lazo'' o bow-tie que deja una ráfaga dispersada. A diferencia
de enfoques anteriores que convertían los datos en imágenes para ingresarlos a la red, aquí el modelo
opera directamente sobre el stream numérico tiempo-DM, lo que ahorra tiempo de I/O en la inferencia
. El detector devuelve las coordenadas centrales del posible evento (es decir, estima
simultáneamente el tiempo de llegada y la medida de dispersión óptima del FRB). 

Habiendo identificado candidatos potenciales, la segunda etapa valida su autenticidad. DRAFTS extrae la porción correspondiente a la señal candidata en
los datos originales (espectro dinámico dedispersado al DM encontrado) y la alimenta a un clasificador
binario basado en ResNet. Este clasificador de imagen verifica si la señal corresponde a un FRB
real o a ruido/RFI, descartando así las detecciones espurias restantes.

¿Por qué esta arquitectura de dos etapas es superior? Esta combinación secuencial de detección y clasificación proporciona notables ventajas. Por un lado,
al detectar en el espacio tiempo-DM, DRAFTS soluciona la incompletitud: cualquier FRB presente,
incluso débil, tiende a manifestarse como un pico localizado cuando los datos se dedispersan
correctamente, lo que el detector puede identificar con alta sensibilidad. Por otro lado, la etapa de
clasificación asegura que los falsos positivos se reduzcan drásticamente -- incluso si algún artefacto de
ruido logra disparar el detector, es muy probable que el clasificador ResNet lo etiquete como no
astrofísico, evitando un reporte erróneo. De hecho, el diseño garantiza que la misma ráfaga no sea
detectada múltiples veces y que prácticamente ninguna interferencia supere ambas etapas: si una
señal falsa es marcada en la primera fase, es filtrada en la segunda, haciendo casi innecesaria la
intervención manual humana en la validación. 

Los resultados empíricos confirman estas ventajas teóricas. \citet{zhang2024drafts} reportan que este pipeline alcanza
niveles de desempeño muy superiores a los métodos convencionales en todos los aspectos clave:
precisión, exhaustividad y velocidad de detección. Entrenado con un extenso conjunto de $\sim2700$
bursts reales de FAST (incluyendo fuentes repetidoras conocidas) y datos simulados, DRAFTS demostró
en pruebas con datos reales no vistos una tasa de recuperación cercana al 100\% de los FRBs
presentes, manteniendo una tasa de falsos alarmas mínima. Notablemente, en la re-búsqueda
de los datos del repetidor FRB 20190520B (observados con el radiotelescopio FAST), DRAFTS detectó
más del triple de ráfagas comparado con el pipeline tradicional Heimdall en ese mismo conjunto de datos. Este resultado subraya cuánto puede mejorar la sensibilidad efectiva mediante técnicas de
deep learning, recuperando muchos eventos que los algoritmos clásicos pasaron por alto. Al mismo
tiempo, la combinación de detector + clasificador redujo drásticamente los falsos positivos, por lo que
prácticamente no se requieren inspecciones manuales de candidatos, incluso operando en tiempo real. 

En suma, DRAFTS representa el estado del arte en pipelines de búsqueda de FRBs: una solución
híbrida que integra visión computacional y clasificación profunda para lograr detecciones más
completas, confiables y rápidas que los enfoques previos, abriendo la vía a escaneos autónomos de
próxima generación en radioastronomía \citep{zhang2024drafts}. La Figura~\ref{fig:drafts_architecture} ilustra esquemáticamente la arquitectura de este pipeline innovador, cuya evolución hacia DRAFTS++ constituye el núcleo de esta tesis.

\begin{figure}[htbp]
\centering
\includegraphics[width=0.95\textwidth]{WorkFlow-original.png}
\caption{Arquitectura del pipeline DRAFTS. Sistema de dos etapas: \textbf{Etapa 1:} Detector CenterNet analiza espacio DM--tiempo identificando coordenadas (DM, tiempo) de candidatos. \textbf{Etapa 2:} Clasificador ResNet18 evalúa autenticidad de candidatos extraídos en formato frecuencia--tiempo. Salida: probabilidad FRB real vs. RFI/ruido. Esta arquitectura híbrida recupera $\sim$3× más ráfagas que métodos tradicionales con mínimos falsos positivos. Adaptado de \citet{zhang2024drafts}.}
\label{fig:drafts_architecture}
\end{figure}

\medskip

Este marco conceptual ha trazado el camino desde los fundamentos físicos de los FRBs hasta el estado del arte en detección mediante aprendizaje profundo, integrando casos históricos, conexión con magnetars, y desafíos específicos de alta frecuencia. La fenomenología observacional diversa, las limitaciones de métodos tradicionales, el éxito de DRAFTS en frecuencias centimétricas, y los desafíos únicos del régimen milimétrico (compresión de firma dispersiva, validación por polarización, limitaciones instrumentales) establecen el contexto completo que motiva el desarrollo de DRAFTS++: un pipeline operativo, robusto y extensible capaz de operar efectivamente tanto en regímenes de dispersión prominente como comprimida.



