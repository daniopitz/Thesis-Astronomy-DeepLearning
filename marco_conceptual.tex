\secnumbersection{MARCO CONCEPTUAL}

\subsection{Fenomenología de FRBs y propagación en plasmas}

\subsubsection{Contexto físico-observacional de FRBs}

En radioastronomía, además de las fuentes permanentes o de 
emisión continua, existe una variedad de fenómenos transitorios 
caracterizados por emisiones de corta duración. Estos eventos incluyen, 
por ejemplo, los pulsos emitidos por púlsares y magnetares (señales periódicas o esporádicas de
origen galáctico) y estallidos únicos o poco frecuentes como los Rotating Radio Transients 
(RRATs) y las recientemente descubiertas ráfagas rápidas de radio (\textit{Fast Radio Bursts}, FRBs). 
Todos ellos son ejemplos de transientes rápidos de radio, es decir, pulsos de radio de duración breve 
(desde microsegundos hasta segundos) que suelen aparecer de manera impredecible. La detección y estudio
de estos transientes proporcionan información sobre procesos astrofísicos extremos, pues su emisión 
implica mecanismos energéticos violentos y propagación a través de medios plasmáticos que dispersan 
y atenúan la señal. En particular, los FRBs se han convertido en uno de los focos de investigación 
más activos en la última década debido a su lejanía y potencia inusual \citep{Petroff_2022}.

Las FRBs son pulsos de radio de duración típicamente del orden de milisegundos, 
que destacan por su intensa luminosidad y alta medida de dispersión (Dispersion Measure, DM), lo que 
indica un origen extragaláctico. El descubrimiento inicial de un FRB en 2007, conocido como la ráfaga 
de Lorimer, marcó un hito al detectar un pulso de $\sim$5 ms cuya intensidad y retardo por dispersión 
sugerían que provenía de fuera de nuestra galaxia \citep{Lorimer2007}. Lorimer et al. (2007) reportaron 
este evento excepcional en la banda de 1.4\,GHz, con un DM de 375\,pc\,cm$^{-3}$, mucho mayor al esperado
 por el contenido de electrones en la Vía Láctea, lo que implicaba un origen cosmológico 
 \citep{Lorimer2007,CordesMcLaughlin2003}. Desde entonces, numerosos radiotelescopios y 
 búsquedas ciegas han descubierto centenares de FRBs; catálogos recientes han consolidado 
 varios miles de eventos y fuentes \citep{CHIME2021,CHIMEFRB_2021_Catalog1,Petroff2019}.

\subsubsection{La ley de dispersión y el "bow-tie" en tiempo--DM}

Las FRBs típicamente muestran un espectro de frecuencia amplio 
(decenas a cientos de MHz de ancho de banda) y una alta brillantitud, emitiendo una energía del orden 
de $10^{38}$--$10^{40}$\,erg en pocos milisegundos. Presentan una marcada dispersión temporal: las 
frecuencias más bajas del pulso llegan más tarde que las altas, siguiendo la ley de dispersión en 
plasma $\Delta t \propto \mathrm{DM}\,\nu^{-2}$. Este retardo se manifiesta como una curvatura 
característica en datos frecuencia-tiempo, cuantificada por la DM, columna integrada de electrones 
libres a lo largo de la línea de visión \citep{LorimerKramer2004}. Las FRBs descubiertas hasta ahora 
tienen DM que van desde $\sim$50 hasta varios miles de pc\,cm$^{-3}$, excediendo lo atribuible a la 
Vía Láctea. Muchas muestran altos grados de polarización lineal, e índices espectrales diversos. 
Un subconjunto (p.ej., FRB\,121102, FRB\,180916, FRB\,20190520B) presenta actividad repetitiva y 
entornos densos.

\subsubsection{Medida de dispersión (DM), retrasos de llegada y su estimación}

\subsubsection{Ensanchamiento por dispersión/escattering y consecuencias en S/N}

\subsubsection{Implicancias para estrategias de búsqueda}

\subsection{Polarización y medio magneto-iónico (RM)}

\subsubsection{Fundamentos de Stokes I,Q,U,V y Rotación Faraday (RM)}

\subsubsection{Definición de Stokes y PA($\lambda^2$); extracción de RM y su interpretación}

\subsubsection{Casos de RM extremos en FRBs; cuándo/por qué se reduce la depolarización}

\subsubsection{Valor diagnóstico de la polarización para distinguir orígenes/medios}

\subsubsection{Representaciones polarimétricas en detección de FRBs: potencial para discriminación en mm-wave}

\subsection{Régimen milimétrico (30--100 GHz): ``compresión dispersiva''}

\subsubsection{Escalamiento del retraso ($\propto\nu^{-2}$) y pérdida de contraste en tiempo--DM}

A frecuencias de decenas a centenas de GHz, $\Delta t\propto \nu^{-2}$ reduce el 
contraste dispersivo en banda; la señal aparece casi simultánea en frecuencia, 
dificultando discriminación frente a ruido/RFI \citep{LorimerKramer2004,CordesChatterjee2019}. 
La sensibilidad instrumental y la tasa de datos exigen procesamiento de baja latencia.

\subsubsection{Requisitos instrumentales: resolución temporal y ancho de banda}

\subsubsection{Consecuencias para S/N y selección del plan DM (pasos/alcance)}

\subsubsection{Evidencias de pulsos/magnetar en mm que motivan el régimen}

Se han detectado pulsos de un magnetar en el Centro Galáctico con ALMA en modo 
\textit{phased}, demostrando viabilidad de detectar transientes a $\sim$100\,GHz y 
resaltando el valor de la polarización en la discriminación de señales 
\citep{Matthews2018,veracasanova2025}. Esto abre la puerta a búsquedas de FRBs repetidoras en mm 
\citep{veracasanova2025}.


\subsection{Representaciones y operaciones núcleo (conceptuales)}

\subsubsection{Waterfall crudo vs dedispersado; artefactos frecuentes}

\subsubsection{Mapas tiempo--DM: geometría del bow-tie, centro (t,DM), contrastes}

\subsubsection{Plan de dedispersión (rango/paso); costo vs fidelidad}

\subsubsection{Métricas físicas básicas: SNR, anchura de pulso, boxcar y plausibilidad}

\secnumbersection{ESTADO DEL ARTE}

\subsection{Pipelines clásicos de búsqueda de pulsos}

\subsubsection{PRESTO: rfifind, DDplan, prepsubband, single\_pulse\_search.py; flujo estándar}

Detectar FRBs en voluminosos datos de radiotelescopios es un reto significativo. 
Los métodos clásicos se basan en identificar pulsos dispersos mediante de\-dispersión 
y umbrales de relación señal-ruido (SNR) \citep{CordesMcLaughlin2003}. En términos generales, 
el flujo tradicional consta de: (i) mitigación inicial de RFI, (ii) de\-dispersión exhaustiva 
sobre una malla de DM, (iii) filtrado por \textit{boxcars}/\textit{downsampling} para abarcar 
anchos de pulso, (iv) agrupado de candidatos y reglas de descarte, y (v) clasificación manual o 
automática. Herramientas como PRESTO y Heimdall implementan variantes optimizadas de este procedimiento
\citep{Ransom_2003,Barsdell_2012,Heimdall_Use}.

\subsubsection{Heimdall: dedispersión (direct/tree/sub-band) acelerada en GPU}

\subsubsection{Fortalezas: madurez, comunidad amplia, integración con formatos}

\subsubsection{Limitaciones: falsos positivos, tuning manual, escalabilidad en grandes DM-planes y datos largos; contraste reducido a mm}

La eficacia depende de la calidad del enmascaramiento de RFI y de la elección de parámetros 
(umbral SNR, tamaños de filtros, paso de DM). La de\-dispersión exhaustiva tiene alto costo 
computacional por redundancias. Existen optimizaciones (p.ej., FDMT) y aceleración en GPU, pero el 
escalado a encuestas masivas es complejo \citep{Zackay_2014_FDMT,Barsdell_2012}. Además, un mismo 
pulso puede generar duplicados a DMs cercanas. En la práctica, proyectos como CHIME/FRB producen 
grandes volúmenes de candidatos diarios, de los que solo una fracción es real \citep{CHIME2021}.

\subsection{Pipelines con deep learning específicos para FRBs}

\subsubsection{DRAFTS: integra CenterNet (detección anchor-free sobre tiempo--DM) + ResNet (clasificación binaria)}

Inicialmente, redes neuronales se aplicaron a la \textit{clasificación} de candidatos 
generados por métodos clásicos (p.ej., FETCH) reduciendo la inspección manual 
\citep{Agarwal_2020,Petroff_2022}. Sin embargo, estos enfoques dependen de la etapa de búsqueda previa.
 Otros intentos exploraron detección directa de firmas dispersivas en datos sin de\-dispersar, con 
 dificultades para señales débiles y variabilidad de curvatura \citep{Zhang_2020}.

El estado del arte lo representa \textbf{DRAFTS} (\textit{Deep-learning RAdio Fast Transient Search}) 
\citep{zhang2024drafts}: un pipeline de \textit{deep learning} que integra detección y 
clasificación de forma unificada. Su objetivo es mejorar eficiencia, completitud y velocidad, 
reduciendo falsos positivos frente a enfoques clásicos.

DRAFTS genera una representación 2D (tiempo en $x$, DM en $y$) donde un FRB aparece como región 
compacta con firma de \textit{bow-tie}. Emplea un detector \textit{anchor-free} (\textbf{CenterNet}) 
para localizar centros y tamaños de objetos \citep{Zhou_2019_CenterNet}, con un backbone tipo 
\textbf{ResNet} \citep{He_2015_ResNet}. La red infiere directamente el tiempo de llegada y la 
DM del pulso \citep{zhang2024drafts}.

Cada detección se recorta, se de\-dispersa a su DM y se clasifica (FRB vs no-FRB) mediante una 
CNN (p.ej., ResNet), entrenada con bursts reales y casos negativos de RFI/ruido 
\citep{Agarwal_2020,zhang2024drafts}.

En datos reales (FAST), DRAFTS detectó sustancialmente más ráfagas que pipelines clásicos 
(p.ej., Heimdall) manteniendo bajo falso positivo, con altas tasas de recuperación y precisión, 
y tiempo de proceso cercano al tiempo real en GPU \citep{zhang2024drafts,Heimdall_Use}. 
Entre las mejoras: evita cómputos redundantes sobre DMs similares, minimiza duplicados y 
aprende variabilidad morfológica relevante.

\subsubsection{CenterNet / Objects as Points: calorímetros de centros + regresión de tamaño/desplazamiento}

\subsubsection{ResNet: aprendizaje residual, backbone ampliamente adoptado}

\subsubsection{Consideraciones para mm-wave: menor bow-tie $\rightarrow$ ajustes de entrada/plan DM/umbrales}

Aumentar cobertura (DM/anchos/sub-bandas) incrementa fuertemente el costo; se requieren estrategias 
focalizadas y aceleración eficiente \citep{zhang2024drafts,Zackay_2014_FDMT}.

La RFI y su variabilidad entre instrumentos exigen estrategias dinámicas 
(p.ej., IQRM, morfología TF, SK) y robustez de dominio 
\citep{Morello_2021_IQRM,Offringa2010,Offringa2012}.

Modelos entrenados en L-banda pueden degradar su rendimiento en mm; 
se requiere reentrenamiento/adaptación de dominio \citep{zhang2024drafts}.

\subsection{Resultados recientes relevantes en mm-wave / ALMA}

\subsubsection{Pulsos del magnetar del GC con ALMA (phased): detecciones y polarización; potencial para FRBs repetidores}

\subsubsection{Detecciones de magnetar hasta 225--291 GHz (IRAM/APEX/others): precedentes a alta frecuencia}

\subsubsection{Primeros sondeos/sensibilidades y lecciones de campañas GMVA+ALMA}

\subsection{Estándares de datos y metadatos para pipelines de FRBs}

\subsubsection{FITS: HDUs, tablas, convenciones de encabezado y WCS}

\subsubsection{PSRFITS: modos SEARCH vs FOLD, polarización y compatibilidad PSRCHIVE}

\subsubsection{SIGPROC Filterbank: espectros dinámicos binarios y compatibilidad}

\subsubsection{MJD y tiempos absolutos; campos de cabecera críticos para pipelines}

