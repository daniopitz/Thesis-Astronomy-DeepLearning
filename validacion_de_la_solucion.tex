\secnumbersection{VALIDACIÓN DE LA SOLUCIÓN}

La validación de la solución propuesta constituye una etapa crítica que permite demostrar la efectividad de los dos bloques desarrollados en diferentes entornos y condiciones. Esta validación se estructura en dos componentes principales: la validación del bloque de bajas frecuencias (DRAFTS++) y la validación del bloque de altas frecuencias (High Frequency Detection).

\subsection{VALIDACIÓN DEL BLOQUE DE BAJAS FRECUENCIAS (DRAFTS++)}

La validación de DRAFTS++ se realizó mediante un proceso incremental que permitió verificar cada componente del sistema antes de su implementación completa. Este enfoque metodológico aseguró la robustez y confiabilidad del pipeline desarrollado.

\subsubsection{Validación Inicial con Dataset de Entrenamiento}

El proceso de validación inició con el dataset FAST REX, el mismo conjunto de datos utilizado para el entrenamiento de los modelos de detección. Esta etapa fue fundamental para validar todas las nuevas funcionalidades implementadas en DRAFTS++, incluyendo los nuevos tipos de visualización y el manejo de archivos de entrada, antes incluso del desarrollo del sistema de chunking.

\subsubsection{Validación de Continuidad Temporal y Time Domain}

Una vez establecida la funcionalidad básica, se procedió a evaluar una característica crítica para cualquier pipeline de detección: la continuidad temporal y la precisión en el dominio del tiempo. Para esta validación se utilizó el pulsar de prueba B0355+54\_FB\_20220918, seleccionado por sus características ideales para este propósito:

\begin{itemize}
    \item \textbf{Brightness}: Pulsar sumamente brillante que facilita la detección
    \item \textbf{Periodo de rotación}: 0.156 segundos
    \item \textbf{Duración del archivo}: 117.23 segundos (1 minuto 57 segundos)
    \item \textbf{Pulsos esperados}: 752 pulsos teóricos
\end{itemize}

Los resultados obtenidos fueron altamente satisfactorios:
\begin{itemize}
    \item \textbf{Pulsos detectados}: 732 de 752 esperados (97.3\% de eficiencia)
    \item \textbf{Clasificación}: 718 clasificados como BURSTS, 14 como NO BURSTS
\end{itemize}

Este resultado confirmó la capacidad del sistema para mantener la continuidad temporal entre ventanas de procesamiento y validó la precisión de las redes de detección pre-entrenadas de DRAFTS en condiciones controladas.

\subsubsection{Validación con Datos del FRB 121102}

Para evaluar el rendimiento del sistema en condiciones reales y con archivos de gran tamaño, se utilizó el dataset del FRB 121102, basado en el trabajo de \cite{cruces2020frb121102}. Este dataset presentó desafíos computacionales significativos que requirieron la implementación del sistema de chunking y la optimización del consumo de recursos.

Los resultados obtenidos superaron las expectativas establecidas en la literatura:
\begin{itemize}
    \item \textbf{Bursts de literatura detectados}: 24/24 (100\% de detección)
    \item \textbf{Nuevos bursts confirmados}: 2 adicionales
    \item \textbf{Candidatos adicionales}: 15 candidatos pendientes de confirmación
\end{itemize}

Esta validación confirmó exitosamente:
\begin{itemize}
    \item El manejo eficiente de archivos FITS y Filterbank
    \item El control total de parámetros por parte del usuario
    \item La continuidad temporal en archivos de gran tamaño
    \item La precisión de candidatos en las redes de detección
    \item La eficiencia del sistema de chunking y overlap
    \item La generación de outputs y visualizaciones apropiadas
\end{itemize}

\subsection{VALIDACIÓN DEL BLOQUE DE ALTAS FRECUENCIAS}

La validación del bloque de altas frecuencias se centró en el dataset del observatorio ALMA, específicamente los datos del magnetar del Centro Galáctico PSR J1745-2900, basado en el trabajo de \cite{veracasanova2025}.

\subsubsection{Validación con DRAFTS++ Adaptado}

La primera línea de validación consistió en aplicar DRAFTS++ tal como estaba configurado después de las validaciones anteriores. Los resultados obtenidos fueron mixtos:

\begin{itemize}
    \item \textbf{Pulsos originales detectados}: Algunos de los 8 pulsos reportados en la literatura fueron detectados
    \item \textbf{Pulsos adicionales}: No se detectaron los pulsos adicionales reportados posteriormente por otros investigadores
\end{itemize}

Estos resultados indicaron que, aunque DRAFTS++ funcionaba correctamente en su dominio de bajas frecuencias, requería adaptaciones específicas para el dominio de altas frecuencias.

\subsubsection{Validación con Detección por SNR + Clasificación Binaria}

La segunda línea de validación implementó un enfoque híbrido combinando detección por relación señal-ruido (SNR) con clasificación binaria. Los resultados obtenidos fueron significativamente mejores:

\begin{itemize}
    \item \textbf{Pulsos confirmados detectados}: 52/52 pulsos (100\% de detección)
    \item \textbf{Candidatos adicionales}: 273 candidatos sin confirmar
\end{itemize}

Este enfoque demostró la efectividad de combinar métodos tradicionales de detección con técnicas de machine learning para el dominio de altas frecuencias.

\subsection{ANÁLISIS COMPARATIVO DE RESULTADOS}

Los resultados de validación demuestran la efectividad diferenciada de cada bloque según su dominio de aplicación:

\begin{table}[ht]
    \centering
    \caption{\label{table:resultados_validacion} Resumen de Resultados de Validación por Bloque.} Fuente: Elaboración Propia.
    \begin{tabular}{p{4cm} p{3cm} p{3cm} p{3cm}}
        \toprule
        \textbf{Métrica} & \textbf{Bajas Frecuencias} & \textbf{Altas Frecuencias} & \textbf{Total} \\
        \midrule
        Eficiencia de Detección & 97.3\% & 100\% & 98.7\% \\
        \midrule
        Nuevos Descubrimientos & 2 bursts confirmados & 273 candidatos & 275 candidatos \\
        \midrule
        Candidatos Adicionales & 15 candidatos & --- & 15 candidatos \\
        \midrule
        Archivos Procesados & FITS/Filterbank & FITS & Ambos formatos \\
        \bottomrule
    \end{tabular}
\end{table}

\subsection{CONCLUSIONES DE LA VALIDACIÓN}

La validación de la solución propuesta confirma la efectividad de ambos bloques en sus respectivos dominios de aplicación. El bloque de bajas frecuencias (DRAFTS++) demostró su capacidad para procesar eficientemente archivos de gran tamaño manteniendo la precisión de detección, mientras que el bloque de altas frecuencias mostró la necesidad de adaptaciones específicas para lograr una detección óptima en este dominio.

Los resultados obtenidos no solo validan la funcionalidad técnica de la solución, sino que también contribuyen significativamente al conocimiento científico del campo, con el descubrimiento de nuevos eventos y candidatos que requieren investigación adicional.
