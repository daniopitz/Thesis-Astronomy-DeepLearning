\secnumbersection{CONCLUSIONES}

\subsection{Resumen de Contribuciones}

El presente trabajo abordó la transformación de DRAFTS desde prototipo de investigación hacia sistema operacional completo (DRAFTS++), mediante dos componentes complementarios que exploran tanto las capacidades como las limitaciones de la ingeniería de software aplicada a la detección de transientes de radio en diferentes regímenes espectrales.

\subsubsection{Componente 1: Infraestructura de Software para Descubrimiento Científico}

El primer componente demuestra que la excelencia en ingeniería de software constituye un habilitador directo de descubrimiento científico. La refactorización arquitectónica profunda de DRAFTS -- desde scripts desacoplados hacia un pipeline modular con procesamiento por chunks, gestión inteligente de memoria RAM/VRAM, continuidad temporal quirúrgica y trazabilidad completa -- representa la condición necesaria para procesar archivos masivos multi-gigabyte en entornos observacionales reales.

Los resultados cuantifican este impacto: Recall perfecto del 100\% (26/26 eventos detectados, incluyendo 24 reportados en literatura y 2 nuevos confirmados), procesamiento sin fallos de observaciones completas, y descubrimiento de 2 nuevos bursts del FRB 121102 confirmados independientemente por astrónomos colaboradores. Este descubrimiento científico directo, que expande el censo conocido de este repetidor icónico, evidencia que una infraestructura de software robusta constituye un componente crítico de la capacidad científica observacional. En regímenes donde la física subyacente está bien comprendida, como la dispersión interestelar en bajas frecuencias, la excelencia en ingeniería de software puede directamente expandir las fronteras del descubrimiento científico.

\subsubsection{Componente 2: Límites Fundamentales y Estrategias Especializadas}

El segundo componente revela que, sin comprensión física profunda del dominio, la ingeniería de software alcanza límites fundamentales irreductibles. La validación experimental en el régimen milimétrico (~86 GHz) cuantifica estos límites mediante dos líneas de investigación contrastantes.

La Línea 1 establece el límite de la adaptación paramétrica: el pipeline clásico alcanza un Recall máximo del 87.5\% con Precision del 36.8\%, y falla completamente para una fracción de señales. Esta limitación no constituye un defecto arquitectural o de implementación, sino una consecuencia directa de la física subyacente: los modelos fueron entrenados en firmas dispersivas desarrolladas (bow-tie) que colapsan en alta frecuencia debido a la dependencia $\Delta t_{\mathrm{ms}} \propto \nu^{-2}$.

La Línea 2 valida una estrategia alternativa basada en comprensión de dominio: al reconocer que el problema es físico (compresión dispersiva) y no arquitectural, la estrategia híbrida SNR-threshold + clasificación CNN, inspirada en pipelines clásicos establecidos como PRESTO, elimina la dependencia del bow-tie visual y recupera Recall del 100\% (16/16 pulsos de referencia). Más significativamente, esta estrategia permitió el descubrimiento de 44 nuevos pulsos confirmados del magnetar PSR J1745-2900, expandiendo el censo conocido en un factor de 5.5×.

La comparación entre ambas líneas evidencia que el éxito no proviene de mayor sofisticación arquitectural, sino de comprensión física del problema: en regímenes donde el retardo dispersivo es comparable a la resolución temporal, la propuesta de candidatos debe operar sobre el dominio temporal directamente, no sobre representaciones bidimensionales DM-tiempo. Esta comprensión requiere conocimiento astronómico profundo que trasciende las capacidades de la ingeniería de software aislada.

\subsection{Contribuciones Metodológicas y Técnicas}

\subsubsection{Arquitectura de Software para Radioastronomía Productiva}

El trabajo establece patrones arquitectónicos transferibles para pipelines astronómicos:

\begin{enumerate}
    \item \textbf{Sistema de chunking con solapamiento controlado}: Validado mediante Recall 100\% sin eventos perdidos en bordes, resuelve el problema fundamental de memoria en observaciones largas manteniendo continuidad temporal quirúrgica.
    
    \item \textbf{Gestión inteligente de memoria con fallback automático}: Planificación dinámica de recursos (RAM/VRAM) permite procesamiento adaptativo sin intervención manual ni crashes, validado en archivos multi-gigabyte.
    
    \item \textbf{Ingesta multi-formato con análisis automático}: Elimina configuración manual codificada, permitiendo operación sobre datos heterogéneos (PSRFITS, FITS, SIGPROC Filterbank) de múltiples observatorios sin ajustes específicos.
    
    \item \textbf{Trazabilidad temporal sub-milisegundo}: Timestamps precisos calculados mediante resolución instrumental validada permiten localización exacta de eventos, crítico para correlación multi-época y multi-instrumento.
\end{enumerate}

Estos patrones no son específicos de DRAFTS++; constituyen \textit{principios de diseño generalizables} para cualquier pipeline astronómico que procese datos temporales masivos.

\subsubsection{Límites de Transferibilidad de Modelos entre Regímenes Espectrales}

El trabajo cuantifica rigurosamente los límites de transferencia de detectores CNN entre regímenes espectrales:

\begin{itemize}
    \item \textbf{Límite cuantificado}: CenterNet entrenado en bajas frecuencias transfiere parcialmente a alta frecuencia (87.5\% máximo), pero con degradación inevitable de precisión y fallas persistentes para subclases de señales.
    
    \item \textbf{Causa raíz identificada}: No es problema de arquitectura CNN ni cantidad de datos de entrenamiento, sino \textit{colapso físico de la firma visual} que el modelo aprendió a reconocer ($\Delta t_{\mathrm{ms}} \propto \nu^{-2}$ comprime bow-tie hasta invisibilidad).
    
    \item \textbf{Solución validada}: Clasificador ResNet18 operando sobre patches normalizados transfiere exitosamente (scores ~1.00 para pulsos genuinos), demostrando que representaciones independientes de régimen espectral permiten reutilización de modelos pre-entrenados sin re-entrenamiento costoso.
\end{itemize}

Esta caracterización rigurosa de transferibilidad informa estrategias futuras: cuando la física cambia fundamentalmente, los modelos deben adaptarse no mediante ajustes paramétricos sino mediante cambio de representación de entrada.

\subsection{Descubrimientos Científicos y Contribución Observacional}

El trabajo produjo descubrimientos científicos concretos validados independientemente:

\begin{itemize}
    \item \textbf{FRB 121102 (Effelsberg, ~1.4 GHz)}: 2 nuevos bursts confirmados al 100\% por astrónomos colaboradores (DM~564 pc cm$^{-3}$, SNR 6.3$\sigma$ y 12.0$\sigma$), expandiendo el censo de eventos de este repetidor icónico.
    
    \item \textbf{PSR J1745-2900 (ALMA, 86 GHz)}: 44 nuevos pulsos confirmados del magnetar del Centro Galáctico, ampliando en factor ×5.5 los eventos conocidos de literatura y demostrando capacidad de descubrimiento en régimen milimétrico.
    
    \item \textbf{Candidatos prometedores}: 15 candidatos adicionales FRB 121102 + 101 candidatos PSR J1745-2900 pendientes de validación polarimétrica completa, representando potencial para expansión futura del censo mediante análisis Stokes Q/U/V.
\end{itemize}

Estos eventos constituyen \textbf{datos observacionales nuevos} disponibles para la comunidad científica, con potencial para estudios de periodicidad, variabilidad temporal, y caracterización espectral de fuentes repetidoras.

\subsection{Conocimiento de Dominio: La Relevancia de Métodos Clásicos}

Un hallazgo metodológico fundamental del presente trabajo es la relevancia persistente de pipelines clásicos establecidos en el diseño de sistemas modernos basados en aprendizaje profundo. El desarrollo de la estrategia híbrida SNR-threshold (Línea 2) no emergió de experimentación exhaustiva con arquitecturas CNN avanzadas, sino del análisis sistemático de cómo pipelines tradicionales como PRESTO \cite{2011ascl.soft07017R} abordan la detección de transientes en condiciones donde la firma dispersiva es débil o inexistente.

\subsubsection{Principios Metodológicos Transferidos desde PRESTO}

El análisis de PRESTO reveló tres principios arquitectónicos fundamentales cuya aplicación resultó determinante para el éxito de la estrategia híbrida:

\begin{enumerate}
    \item \textbf{Independencia de representaciones visuales bidimensionales}: PRESTO opera directamente sobre series temporales mediante cálculo de relación señal-ruido con filtrado adaptado, evitando dependencia de patrones espaciales bidimensionales. Esta aproximación demuestra robustez inherente ante compresión dispersiva, ya que no requiere la presencia de estructuras visuales específicas en el espacio DM-tiempo.
    
    \item \textbf{Dedispersión coherente posterior a propuesta}: Tras identificación de candidatos por umbral SNR, PRESTO aplica dedispersión local optimizando medida de dispersión para maximizar coherencia temporal. DRAFTS++ adapta este principio mediante estimación de $\mathrm{DM}^*$ óptimo por máxima coherencia en rejilla local, seguida de clasificación del patch dedispersado.
    
    \item \textbf{Complementariedad de enfoques}: La estrategia híbrida combina la robustez de propuesta clásica basada en SNR con el poder discriminativo de clasificadores CNN pre-entrenados, aprovechando las fortalezas de ambas metodologías sin requerir re-entrenamiento de modelos existentes.
\end{enumerate}

\subsubsection{Implicaciones para Investigación Futura}

Los resultados sugieren que la investigación en aplicaciones de aprendizaje profundo para astronomía debe incorporar sistemáticamente el conocimiento acumulado en métodos clásicos establecidos. Los pipelines tradicionales representan décadas de conocimiento de dominio sobre física de señales astronómicas, destilado en aproximaciones algorítmicas específicamente diseñadas para abordar limitaciones instrumentales y físicas fundamentales. Este conocimiento debe informar activamente el diseño de sistemas modernos, complementando las capacidades de modelos de aprendizaje profundo en lugar de ser descartado como tecnológicamente obsoleto.

\subsection{Limitaciones y Fronteras del Trabajo}

\subsubsection{Limitaciones Reconocidas del Componente 1}

\begin{itemize}
    \item \textbf{Dependencia de modelos pre-entrenados}: DRAFTS++ integra pero no re-entrena los modelos CenterNet/ResNet18 existentes. El rendimiento está fundamentalmente limitado por calidad de entrenamiento original del prototipo DRAFTS.
    
    \item \textbf{Validación limitada a repetidores conocidos}: Los descubrimientos científicos (2 nuevos bursts FRB 121102) provienen de fuentes repetidoras con múltiples observaciones. Sensibilidad en bursts únicos aislados requiere validación adicional con datasets más extensos.
    
    \item \textbf{Falsos positivos no eliminados}: Precision moderada (63.4\% FRB 121102, 37.3\% PSR J1745-2900) refleja filosofía de diseño (sensibilidad sobre especificidad), pero requiere validación manual posterior de candidatos. Reducir FPs sin perder recall genuino es desafío abierto.
\end{itemize}

\subsubsection{Limitaciones Reconocidas del Componente 2}

\begin{itemize}
    \item \textbf{Dataset de referencia reducido}: Validación con N=8 pulsos confirmados (PSR J1745-2900) limita potencia estadística para caracterizar rigurosamente comportamiento en alta frecuencia. Datasets más extensos permitirían análisis distribucional completo.
    
    \item \textbf{Líneas 3 y 4 no completadas}: Representaciones 2D alternativas (espectrogramas polarimétricos, mapas tiempo-ancho, tiempo-RM, coherencia espectral) y estrategias de Zhang (DM-expand, fishing DM$\approx$0) permanecen en desarrollo. Su viabilidad y rendimiento comparativo con Línea 2 requiere investigación futura.
    
    \item \textbf{Ausencia de validación polarimétrica}: Los 101 candidatos prometedores PSR J1745-2900 carecen de análisis Stokes Q/U/V completo. Discriminación FRB/RFI mediante polarización extrema (>50\% típica de FRBs vs. RFI no polarizada) refinaría precisión significativamente.
\end{itemize}

\subsubsection{Fronteras Metodológicas: Límites de la Ingeniería de Software}

El análisis comparativo entre componentes revela una frontera metodológica fundamental en la aplicación de ingeniería de software a problemas de radioastronomía: existen regímenes donde el factor limitante del rendimiento no es la sofisticación de la implementación computacional, sino la comprensión física insuficiente del dominio. Esta distinción tiene implicaciones significativas para la dirección de investigación futura:

\begin{enumerate}
    \item \textbf{Representaciones discriminativas en alta frecuencia}: Las Líneas 3 y 4 exploran representaciones bidimensionales alternativas (espectrogramas polarimétricos, mapas tiempo-ancho, tiempo-RM, coherencia espectral) diseñadas para preservar separabilidad entre señales astrofísicas y RFI en ausencia del bow-tie dispersivo. El desarrollo de estas representaciones requiere investigación física fundamental sobre qué características espectrales, polarimétricas o temporales distinguen intrínsecamente pulsos cósmicos de interferencia terrestre en regímenes comprimidos. Esta investigación trasciende las capacidades de la ingeniería de software y requiere avances en comprensión astronómica del dominio.
    
    \item \textbf{Arquitecturas específicas de dominio}: Si bien ResNet18 transfiere exitosamente a alta frecuencia operando sobre patches normalizados, su arquitectura fue originalmente diseñada para clasificación de objetos naturales (ImageNet) y no para señales astronómicas. Arquitecturas específicamente optimizadas para el dominio (e.g., redes convolucionales temporales unidimensionales, transformers adaptados para series de tiempo astronómicas) podrían mejorar el rendimiento, pero su diseño óptimo requiere desarrollo de teoría de señales astronómicas más formal y completa.
    
    \item \textbf{Integración multi-mensajero}: La detección de transientes en la era de astronomía multi-mensajero requiere correlación coherente entre observaciones de radiotelescopios, detectores de ondas gravitacionales, telescopios ópticos y detectores de neutrinos. El diseño de pipelines que exploten esta coherencia multi-canal requiere modelos físicos unificados de fuentes transitorias que actualmente se encuentran en desarrollo activo en astrofísica teórica.
\end{enumerate}

Los resultados sugieren que el progreso futuro en detección de transientes en regímenes desafiantes (alta frecuencia, relaciones señal-ruido reducidas, morfologías no estándar) requiere avances en comprensión astrofísica fundamental en igual medida que mejoras en capacidades computacionales. La ingeniería de software puede construir infraestructura robusta, escalable y eficiente, pero no puede sustituir el conocimiento físico del dominio.

\subsection{Trabajo Futuro: Líneas Concretas de Investigación}

\subsubsection{Corto Plazo (3-6 meses): Completar Componente 2}

\begin{enumerate}
    \item \textbf{Implementación completa Línea 3 (Representaciones 2D)}:
    \begin{itemize}
        \item Desarrollo de generadores de espectrogramas polarimétricos (Stokes IQUV multicanal)
        \item Implementación de mapas tiempo-ancho con filtrado multi-escala
        \item Construcción de mapas tiempo-RM para análisis de rotación Faraday
        \item Evaluación cuantitativa comparativa vs. Línea 2 (baseline SNR-threshold)
    \end{itemize}
    
    \item \textbf{Validación polarimétrica exhaustiva}:
    \begin{itemize}
        \item Análisis Stokes Q/U/V completo de 101 candidatos prometedores PSR J1745-2900
        \item Discriminación FRB/RFI mediante polarización extrema (umbral >50\%)
        \item Refinamiento de métricas Precision mediante validación polarimétrica rigurosa
    \end{itemize}
    
    \item \textbf{Extensión a datasets HF adicionales}:
    \begin{itemize}
        \item Validación en observaciones ALMA de otros magnetares conocidos
        \item Evaluación en bandas milimétricas intermedias (40-50 GHz, Band 1-2)
        \item Caracterización sistemática de Recall/Precision vs. frecuencia central
    \end{itemize}
\end{enumerate}

\subsubsection{Mediano Plazo (6-12 meses): Extensión Científica}

\begin{enumerate}
    \item \textbf{Búsqueda sistemática de nuevos eventos}:
    \begin{itemize}
        \item Procesamiento de archivos históricos ALMA/Effelsberg con DRAFTS++
        \item Campaña de re-detección en observaciones archivadas de repetidores conocidos
        \item Validación científica independiente de candidatos emergentes por comités de astrónomos
    \end{itemize}
    
    \item \textbf{Caracterización temporal multi-época}:
    \begin{itemize}
        \item Análisis de periodicidad en eventos recién descubiertos (44 pulsos PSR J1745-2900)
        \item Estudios de variabilidad temporal y tasas de actividad
        \item Correlación con fenómenos multi-mensajero (rayos-X, óptico)
    \end{itemize}
    
    \item \textbf{Integración con observatorios en tiempo real}:
    \begin{itemize}
        \item Adaptación de DRAFTS++ para operación en pipelines de comensales (e.g., ALFABURST en Arecibo, UTMOST en Molonglo)
        \item Implementación de sistemas de alerta temprana para triggers de seguimiento multi-frecuencia
        \item Demostración operacional en campañas observacionales reales (no solo re-análisis post-facto)
    \end{itemize}
\end{enumerate}

\subsubsection{Largo Plazo (1-2 años): Investigación Fundamental}

\begin{enumerate}
    \item \textbf{Arquitecturas específicas de dominio}:
    \begin{itemize}
        \item Diseño de redes neuronales 1D optimizadas para series temporales astronómicas (vs. CNNs 2D adaptadas de visión)
        \item Exploración de transformers con atención temporal para capturar dependencias largas en datos dispersados
        \item Modelos generativos (VAEs, GANs) para síntesis de señales realistas de entrenamiento en regímenes con datos escasos
    \end{itemize}
    
    \item \textbf{Teoría de representaciones discriminativas HF}:
    \begin{itemize}
        \item Investigación física fundamental sobre características espectrales/temporales/polarimétricas que distinguen FRBs de RFI en alta frecuencia
        \item Desarrollo de espacios de características invariantes a régimen espectral basados en primeros principios físicos
        \item Formalización matemática de "bow-tie artificial" en espacios de representación de alta dimensión
    \end{itemize}
    
    \item \textbf{Integración multi-mensajero}:
    \begin{itemize}
        \item Diseño de pipelines unificados que correlacionen señales de radio, ondas gravitacionales (LIGO/Virgo), neutrinos (IceCube), rayos gamma (Fermi)
        \item Desarrollo de modelos físicos de fuentes transitorias que predigan firmas multi-mensajero coherentes
        \item Implementación de sistemas de decisión probabilística Bayesiana para clasificación multi-canal
    \end{itemize}
\end{enumerate}

\subsection{Reflexiones Finales}

El presente trabajo ilustra principios fundamentales sobre la naturaleza de la investigación interdisciplinaria en astronomía computacional: el progreso científico en este dominio requiere excelencia simultánea en ingeniería de software y comprensión física del dominio, siendo ambas componentes necesarias e insustituibles.

El Componente 1 demuestra que la ingeniería de software rigurosa -- incorporando modularidad, gestión eficiente de recursos, procesamiento en streaming y trazabilidad completa -- constituye un habilitador directo de descubrimiento científico y no meramente infraestructura técnica de soporte. Los 2 nuevos bursts del FRB 121102 confirmados existen como contribución científica observable precisamente porque la infraestructura de software desarrollada permitió su procesamiento y detección.

El Componente 2 demuestra que la sofisticación arquitectural, en ausencia de comprensión física profunda del dominio, alcanza límites fundamentales irreductibles. La Línea 1 cuantifica este límite (Recall máximo 87.5\%); la Línea 2 evidencia que su superación no requiere arquitecturas CNN más complejas, sino comprensión causal de las limitaciones del enfoque original (colapso del bow-tie dispersivo por física de la dispersión interestelar).

\subsubsection{Implicaciones para Práctica Profesional}

Los resultados sugieren implicaciones específicas para la práctica profesional en astronomía computacional:

\begin{enumerate}
    \item Los ingenieros informáticos que desarrollan sistemas para astronomía deben cultivar comprensión suficiente de la física subyacente para distinguir problemas de implementación (resolubles mediante mejor arquitectura de software) de problemas de representación (que requieren estrategias alternativas informadas por conocimiento de dominio).
    
    \item Los astrónomos que emplean herramientas computacionales deben reconocer que la calidad de la ingeniería de software constituye infraestructura científica crítica que expande directamente las capacidades observacionales, no meramente soporte técnico auxiliar.
    
    \item La colaboración interdisciplinaria efectiva requiere comunicación bidireccional: ingenieros que comprendan física suficiente para formular preguntas relevantes, y astrónomos que valoren ingeniería suficiente para reconocer cuándo limitaciones técnicas restringen capacidades científicas.
\end{enumerate}

DRAFTS++ representa un caso de estudio de esta colaboración exitosa: ingeniería de software informada por comprensión física, física observacional habilitada por infraestructura de ingeniería. El desarrollo futuro de sistemas de detección de transientes de radio dependerá críticamente de mantener y profundizar esta integración interdisciplinaria.

\subsection{Síntesis de Contribuciones}

El presente trabajo transforma DRAFTS desde prototipo de investigación hacia sistema operacional completo (DRAFTS++), estableciendo contribuciones específicas en tres dimensiones:

\subsubsection{Contribuciones Observacionales}

\begin{itemize}
    \item Descubrimiento de 46 eventos confirmados independientemente (2 bursts FRB 121102 + 44 pulsos PSR J1745-2900), expandiendo censos conocidos de fuentes repetidoras icónicas
    \item Identificación de 116 candidatos prometedores adicionales (15 FRB 121102 + 101 PSR J1745-2900) pendientes de validación polarimétrica mediante análisis Stokes Q/U/V
\end{itemize}

\subsubsection{Contribuciones Metodológicas}

\begin{itemize}
    \item Cuantificación rigurosa de límites de transferibilidad de modelos CNN entre regímenes espectrales, estableciendo que la adaptación paramétrica alcanza Recall máximo del 87.5\% debido a colapso físico de firmas dispersivas
    \item Validación de estrategia híbrida SNR-threshold + clasificación CNN como solución efectiva para alta frecuencia, logrando Recall del 100\% mediante eliminación de dependencia en representaciones visuales bidimensionales
    \item Demostración de relevancia persistente de conocimiento acumulado en pipelines clásicos (PRESTO) para informar diseño de sistemas modernos basados en aprendizaje profundo
\end{itemize}

\subsubsection{Contribuciones Técnicas}

\begin{itemize}
    \item Patrones arquitectónicos transferibles para pipelines astronómicos productivos: procesamiento por chunks con solapamiento controlado, gestión inteligente de memoria con fallback automático, ingesta multi-formato con análisis automático, trazabilidad temporal sub-milisegundo
    \item Sistema operacional validado mediante procesamiento sin fallos de archivos multi-gigabyte y Recall perfecto (100\%) sin eventos perdidos en bordes de chunks
\end{itemize}

\subsubsection{Contribución Conceptual}

Más allá de resultados específicos, el trabajo establece comprensión matizada sobre la naturaleza de investigación interdisciplinaria en astronomía computacional: cuándo la ingeniería de software puede resolver problemas directamente mediante mejor implementación, cuándo requiere informarse activamente de conocimiento de dominio físico, y cuándo los límites fundamentales del rendimiento demandan investigación astrofísica nueva. Esta caracterización evita tanto determinismo tecnológico ingenuo como pesimismo sobre límites inherentes, proporcionando marco conceptual para guiar investigación futura en el campo.
