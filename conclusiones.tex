\secnumbersection{CONCLUSIONES}

\textit{DRAFTS++ demuestra que la ingeniería informada por física transforma un prototipo de detección de FRBs en un sistema operacional end-to-end que, además de procesar a escala con robustez, produce descubrimiento científico verificable.}

\subsection{Síntesis frente a Objetivos}

El sistema opera E2E sin intervención y procesa archivos multi-gigabyte con recuperación íntegra de eventos (Recall 100\% en FRB 121102 del telescopio Effelsberg); valida robustez temporal en fuentes periódicas (732/752 pulsos en B0355+54, 97.3\%); y extiende sensibilidad al régimen milimétrico (~86 GHz ALMA) con un enfoque híbrido inspirado en PRESTO (Recall 100\% y 44 pulsos nuevos confirmados del magnetar PSR J1745-2900). La mejora sostenida se evidencia también en curvas Precision-Recall y AUPRC, métricas adecuadas para clases desbalanceadas en detección de transientes astronómicos. El descubrimiento de 46 eventos confirmados independientemente (2 FRBs + 44 pulsos de magnetar) más 116 candidatos prometedores demuestra que el sistema amplía genuinamente conocimiento astronómico, no solo replica análisis previos.

\subsection{Significado e Impacto}

Operacionalmente, DRAFTS++ pasa de código de investigación a infraestructura científica reutilizable para re-análisis a escala y campañas near-real-time. Científicamente, amplía el censo (2 bursts nuevos de FRB 121102 y 44 pulsos del magnetar del Centro Galáctico) y fija un baseline cuantificado para transferencia a alta frecuencia. En HF, la caracterización rigurosa reveló que la existencia de pulsos persistentemente indetectables (Línea 1: Recall máximo 87.5\%) expone limitaciones arquitecturales irreductibles cuando el bow-tie colapsa, motivando estrategias especializadas; la estrategia híbrida (Línea 2: Recall 100\%) superó este límite apoyándose en estadísticos temporales en lugar de firmas 2D comprimidas. La mejora se refleja también en PR/AUPRC, métricas idóneas en datos desbalanceados.

\subsection{Calidad del Software y Reproducibilidad}

Evaluamos con ISO/IEC 25010 (2011, actualizado 2023): rendimiento, confiabilidad (0 pérdidas en bordes), mantenibilidad modular y portabilidad multi-instrumento; artefactos FAIR disponibles en \url{https://github.com/Kodamonkey/DRAFTS-UC} y orientados a ACM Artifact Badging.

\subsection{Limitaciones y Próximos Pasos}

\begin{itemize}
    \item HF con N reducido → ampliar dataset y polarimetría Q/U/V.
    \item Generalización → validar en VLA/MeerKAT/CHIME.
    \item Tiempo real → integración operacional (streaming y telemetría).
\end{itemize}

\subsection{Trabajo Futuro}

El desarrollo de DRAFTS++ estableció una arquitectura sólida y validó empíricamente dos estrategias para detección en alta frecuencia (Líneas 1 y 2). Sin embargo, durante el diseño del Componente 2 se propusieron \textbf{cuatro líneas metodológicas complementarias}, de las cuales \textbf{solo dos fueron implementadas y validadas} en esta memoria debido a restricciones de tiempo y alcance. Las \textbf{Líneas 3 y 4}, descritas arquitectónicamente en el capítulo de propuesta, quedaron como \textbf{trabajo futuro prioritario de nivel posgrado}.

\subsubsection{Línea 3: Representaciones 2D Alternativas}

\textit{Estado: Propuesta conceptual no implementada}

En frecuencias milimétricas donde la firma ``bow-tie'' se comprime, se proponen representaciones 2D alternativas que generen patrones discriminativos artificiales:

\paragraph{Propuestas de Representaciones}

\textbf{Espectrograma Polarimétrico (Stokes IQUV):} Imagen multicanal (I, Q, U, V) explotando polarización extrema de FRBs ($>50\%$) versus RFI no polarizada. FRBs generan trazas coherentes en canales I/Q/U, mientras RFI aparece solo en I.

\textbf{Mapa Tiempo-Ancho:} Respuesta SNR a diferentes ventanas de integración. Pulsos astrofísicos producen patrón ``triángulo invertido'' (máxima SNR en $w \approx \tau_{\text{pulso}}$), distintivo de RFI breve o extendida.

\textbf{Mapa Tiempo-RM:} Explora espacio de rotación Faraday. FRBs con RM elevada producen franjas verticales; RFI aparece cerca de RM=0.

\textbf{Coherencia Espectral:} Autocorrelación por frecuencia. FRBs broadband persisten a lags grandes; RFI narrowband decae abruptamente.

\textbf{Bow-tie Artificial:} Combinación multi-representación mediante concatenación $\mathbf{I}_{\mathrm{combined}} = \mathrm{concat}[\mathbf{I}_{\mathrm{width}}, \mathbf{I}_{\mathrm{RM}}, \mathbf{I}_{\mathrm{coherence}}]$, generando firma multidimensional equivalente al bow-tie clásico.

\subsubsection{Línea 4: Estrategias Avanzadas de Zhang}

\textit{Estado: Propuesta conceptual no implementada}

Se proponen dos estrategias que preservan la filosofía DM-centrada adaptándose a compresión dispersiva:

\paragraph{Estrategias Propuestas}

\textbf{Estrategia 1 - Expansión de Rejilla DM:} Forzar apertura morfológica del bow-tie mediante rangos DM ampliados ($\gamma > 1$) y pasos más gruesos, permitiendo que modelos entrenados en bow-ties desarrollados recuperen sensibilidad en regímenes comprimidos. Requiere módulo de cálculo dinámico de parámetros $\gamma$ y $d_{\min}$ según frecuencia central, ancho de banda y resolución temporal.

\textbf{Estrategia 2 - Fishing a DM$\approx$0:} Detección primaria permisiva en DM mínimo seguida de validación física rigurosa: (i) ajuste de DM por maximización SNR en sub-bandas ($\mathrm{DM}_{\mathrm{best-fit}} = \arg\max_{\mathrm{DM}} \sum_{f} \mathrm{SNR}_f(\mathrm{DM})$), (ii) verificación de coherencia entre sub-bandas ($\mathrm{Consistency} > \alpha_{\mathrm{consistency}}$), y (iii) consistencia temporal cross-chunk. Requiere validador DM-aware multi-criterio integrado.

\subsubsection{Recomendaciones para Implementación Futura}

\begin{itemize}
    \item \textbf{Priorizar Línea 3:} Análisis polarimétrico IQUV es el discriminador más robusto en alta frecuencia según evidencia empírica de ALMA.
    \item \textbf{Datasets polarimétricos:} Requiere acceso a datos Stokes IQUV calibrados con ground truth curado.
    \item \textbf{Validación multi-instrumento:} Extender validación a VLA, MeerKAT, CHIME para generalización robusta.
    \item \textbf{Integración tiempo real:} Desarrollar conectores VOEvent para alertas multi-mensajero.
\end{itemize}

\subsection{Cierre}

La ruta física → algoritmos → arquitectura habilitó descubrimiento reproducible; proponemos este estándar para software astronómico productivo.
